%%%%%%%%%%%%%%%%%%%%%%%%%%%%%%%%%%%%%%%%%%%%%%%%%%%
%
%  New template code for TAMU Theses and Dissertations starting Fall 2012.  
%  For more info about this template or the 
%  TAMU LaTeX User's Group, see http://www.howdy.me/.
%
%  Author: Wendy Lynn Turner 
%	 Version 1.0 
%  Last updated 8/5/2012
%
%%%%%%%%%%%%%%%%%%%%%%%%%%%%%%%%%%%%%%%%%%%%%%%%%%%
%%%%%%%%%%%%%%%%%%%%%%%%%%%%%%%%%%%%%%%%%%%%%%%%%%%%%%%%%%%%%%%%%%%%%
%%                           ABSTRACT 
%%%%%%%%%%%%%%%%%%%%%%%%%%%%%%%%%%%%%%%%%%%%%%%%%%%%%%%%%%%%%%%%%%%%%

\chapter*{ABSTRACT}
\addcontentsline{toc}{chapter}{ABSTRACT} % Needs to be set to part, so the TOC doesnt add 'CHAPTER ' prefix in the TOC.

\pagestyle{plain} % No headers, just page numbers
\pagenumbering{roman} % Roman numerals
\setcounter{page}{2}

\indent 
We have implemented a new high-order low-order (HOLO) algorithm for solving
thermal radiative transfer problems.  The low-order (LO) system is based on spatial
and angular moments of the transport equation and a linear-discontinuous
finite-element spatial representation, producing equations similar to the standard
S$_2$ equations.  
The LO solver is fully implicit in time and efficiently resolves the non-linear
temperature dependence at each time step.   The HO solver
utilizes exponentially-convergent Monte Carlo (ECMC) to give a globally accurate solution
for the angular intensity to a fixed-source, pure absorber transport problem.  This
global solution is used to compute consistency terms, which require the HO and LO solutions to converge towards the same solution.  The use of ECMC
allows for the efficient reduction of statistical noise in the MC solution, reducing
inaccuracies introduced through the LO consistency
terms.  We compare results with an
implicit Monte Carlo (IMC) code for one-dimensional, gray test problems and 
demonstrate the efficiency of ECMC over standard Monte Carlo in this HOLO algorithm.

\pagebreak{}
