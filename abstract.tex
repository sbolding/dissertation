%%%%%%%%%%%%%%%%%%%%%%%%%%%%%%%%%%%%%%%%%%%%%%%%%%%
%
%  New template code for TAMU Theses and Dissertations starting Fall 2012.  
%  For more info about this template or the 
%  TAMU LaTeX User's Group, see http://www.howdy.me/.
%
%  Author: Wendy Lynn Turner 
%	 Version 1.0 
%  Last updated 8/5/2012
%
%%%%%%%%%%%%%%%%%%%%%%%%%%%%%%%%%%%%%%%%%%%%%%%%%%%
%%%%%%%%%%%%%%%%%%%%%%%%%%%%%%%%%%%%%%%%%%%%%%%%%%%%%%%%%%%%%%%%%%%%%
%%                           ABSTRACT 
%%%%%%%%%%%%%%%%%%%%%%%%%%%%%%%%%%%%%%%%%%%%%%%%%%%%%%%%%%%%%%%%%%%%%

\chapter*{ABSTRACT}
\addcontentsline{toc}{chapter}{ABSTRACT} % Needs to be set to part, so the TOC doesnt add 'CHAPTER ' prefix in the TOC.

\pagestyle{plain} % No headers, just page numbers
\pagenumbering{roman} % Roman numerals
\setcounter{page}{2}

\indent 
We have implemented a new high-order low-order (HOLO) algorithm for solving
thermal radiative transfer (TRT) problems.  Within each discrete time step, fixed-point iterations are performed between a
high-order (HO) exponentially-convergent Monte Carlo (ECMC) solver and a low-order (LO)
system of equations.  The LO system is based on spatial
and angular moments of the transport equation and a linear-discontinuous
finite-element (LDFE) spatial representation, producing equations similar to the standard
S$_2$ equations.  
The LO solver is fully implicit in time and efficiently converges the non-linear
temperature dependence with Newton's method.   The HO solver
provides a globally accurate solution
for the angular intensity to a fixed-source, pure absorber transport problem.  This
global solution is used to compute consistency terms in the LO equations that require the HO and LO solutions
to converge towards the same solution. The use of ECMC
allows for the efficient reduction of statistical noise in the solution.
  
We investigated several extensions of this algorithm.
 A parametric closure
of the LO system was used for the spatial variable, based on local
relations computed with the HO solver.
The spatial closure improves
consistency between the two solvers compared to a standard LDFE spatial
discretization of the LO system.  The ECMC algorithm has been
extended to integrate the angular intensity in time,
with a consistent time closure of the LO radiation equations.  The
time closure increases accuracy in optically-thin problems compared to a backward Euler discretization.  
Finally, we have applied standard source iteration and Krylov procedures to
iteratively solve the LO equations, with linear diffusion synthetic acceleration.  

Herein, we present results for one-dimensional, gray test problems.  Results demonstrate
several desirable properties of this algorithm: the HOLO method preserves the equilibrium diffusion limit, prevents violation
of the maximum principle, and can provide high-fidelity MC solutions to the TRT equations
with minimal statistical noise.  We have compared results with an implicit Monte Carlo
(IMC) code and compared the efficiency of ECMC to standard Monte Carlo in this HOLO
algorithm.  Our HOLO algorithm is more accurate and more efficient than standard IMC.  The
extent to which this is so is problem-dependent.

\pagebreak{}
