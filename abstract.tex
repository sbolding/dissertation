%%%%%%%%%%%%%%%%%%%%%%%%%%%%%%%%%%%%%%%%%%%%%%%%%%%
%
%  New template code for TAMU Theses and Dissertations starting Fall 2012.  
%  For more info about this template or the 
%  TAMU LaTeX User's Group, see http://www.howdy.me/.
%
%  Author: Wendy Lynn Turner 
%	 Version 1.0 
%  Last updated 8/5/2012
%
%%%%%%%%%%%%%%%%%%%%%%%%%%%%%%%%%%%%%%%%%%%%%%%%%%%
%%%%%%%%%%%%%%%%%%%%%%%%%%%%%%%%%%%%%%%%%%%%%%%%%%%%%%%%%%%%%%%%%%%%%
%%                           ABSTRACT 
%%%%%%%%%%%%%%%%%%%%%%%%%%%%%%%%%%%%%%%%%%%%%%%%%%%%%%%%%%%%%%%%%%%%%

\chapter*{ABSTRACT}
\addcontentsline{toc}{chapter}{ABSTRACT} % Needs to be set to part, so the TOC doesnt add 'CHAPTER ' prefix in the TOC.

\pagestyle{plain} % No headers, just page numbers
\pagenumbering{roman} % Roman numerals
\setcounter{page}{2}

\indent 
We have implemented a new high-order low-order (HOLO) algorithm for solving
thermal radiative transfer problems.  Within each discrete time step, fixed-point iterations are performed between a
high-order (HO) exponentially-convergent Monte Carlo (ECMC) solver and a low-order (LO)
system of equations.  The LO system is based on spatial
and angular moments of the transport equation and a linear-discontinuous
finite-element (LDFE) spatial representation, producing equations similar to the standard
S$_2$ equations.  
The LO solver is fully implicit in time and efficiently resolves the non-linear
temperature dependence at each time step with Newton's method.   The HO solver
utilizes ECMC to give a globally accurate solution
for the angular intensity to a fixed-source, pure absorber transport problem.  This
global solution is used to compute consistency terms that require the HO and LO solutions
to converge towards the same solution, neglecting spatial discretization inconsistencies. The use of ECMC
allows for the efficient reduction of statistical noise in the Monte Carlo (MC) solution,
reducing statistical inaccuracies introduced through the LO consistency
terms. 
  
We have also investigated several extensions of this algorithm. A parametric closure is used to more
accurately close the LO system in the spatial variable, based on relations between the HO
moments and point solutions at edges. The spatial closure improves consistency between the two solvers compared to
a standard LDFE spatial discretization of the LO system.  Additionally, the ECMC algorithm has been extended to include
integration of the time variable for the angular intensity with a consistent parametric time closure of the LO
radiation equations.  The Monte Carlo integration of the time variable increases accuracy in optically thin problems compared to a
backward Euler discretization and improves statistical efficiency compared to IMC.
However, the required number of Monte Carlo histories is increased to sufficiently sample the
higher-dimensional trial space, compared to the time-discrete HOLO
algorithm.  We have also developed a method to delay ECMC stagnation when the solution is
not accurately represented by the trial space, while producing a physical, positive solution.  Finally, we have applied standard source iteration and Krylov procedures to iteratively solve the LO
equations, within each Newton iteration, accelerated with linear diffusion synthetic
acceleration (DSA).  

Herein, we present results for one-dimensional, gray test problems.  Results demonstrate
several desirable properties of this algorithm: the HOLO method preserves the equilibrium diffusion limit, prevents violation
of the maximum principle, and can provide high-fidelity MC solutions to the thermal
radiative transfer equations
with minimal statistical noise.  We have compared results with an implicit Monte Carlo
(IMC) code and compared the efficiency of ECMC to standard Monte Carlo in this HOLO
algorithm.  Our HOLO algorithm is more accurate and more efficient than standard IMC.  The
extent to which this is so is problem-dependent.

\pagebreak{}
