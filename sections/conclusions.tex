\chapter{ \uppercase{Conclusions and Future Work} }


\section{CONCLUSIONS}

We have been able to produce solutions for Marshak wave test problems using
a new HOLO method that are in agreement with IMC.  Unlike IMC, our method requires no effective scattering
events to be included in the MC simulation, which limits the run time of particle
tracking, while adding the cost of a LO newton solver. Average LO iteration counts
did not significantly increase as the time step size was increased. The LDFE spatial representation
mitigates issues with teleportation error, producing results with spatial accuracy
comparable to IMC with source tilting.  The ECMC approach, with initial guesses based on the
previous radiation intensity, results in efficient reduction of statistical error and
allows for particles to be distributed to largely varying regions of the problem.
The LO solver resolves the non-linearities in the equations resulting in a fully
implicit time discretization.
The LO solver
can accurately and efficiently resolve the solution in diffusive regions, while the HO
transport solver provides the accuracy of a full transport treatment where necessary. 

The primary difficulty to overcome in the ECMC algorithm is when the solution cannot
be accurately represented by the trial space, e.g., in optically thick cells where
the solution is driven negative.   We are currently developing an approach to allow
the ECMC iterations to continue converging globally when there are such regions
present.  It is necessary to ensure the
 closure in the LO system is consistent with the HO
representation for the solution in such regions.  The ability to represent the solution accurately in
rapidly varying regions of the problem will be key for generalization of this method
to higher dimensions.  A formulization of the ECMC method
that allows for time-continuous MC transport (similar to IMC) is also currently being
investigated.  This may reduce some of the loss of accuracy in optically thin regions
due to the time discretization of the transport equation in the HO solver.
However, greater time accuracy is not of primary concern as this method is intended
for use in problems dominated by large absorption opacities, where the LO
acceleration is critical. Inclusion of Compton scattering in this algorithm, which would introduce
additional non-linear dependence through energy exchange with the
material, is a topic for future research.


Future work will also explore the accuracy of the HOLO method, in particular,
analyzing the optimal number of batches and the benefit of adaptive refinement.  This will likely
require the use of manufactured solutions.  The sensitivity of the method to mesh sizes and time step sizes will be
investigated more thoroughly.  
Ultimately, we plan to extend this method to multiple spatial dimensions for the 
case of multigroup TRT equations.  For TRT problems, it is important that
the LO spatial discretization satisfies the equilibirium diffusion limit.  To extend
to higher dimensions, our LDFE representation may require the use of a higher-degree
spatial representation for the LO system to achieve the diffusion
limit. Further asymptotic
analysis on the method will be applied before implementation. It may be necessary to use a different LO system (e.g., the non-linear diffusion
acceleration approach in~\cite{rmc}), if the S$_2$-like equations become too
inefficient or difficult to implement in higher dimensions.  Alternatively, a
variable Eddington Tensor approach may provide more stability in rapidly variable
regions of the problem while still allowing for a consistent, LDFE solution that is efficiently solvable.

Poor statistics for the face tallies may result in this trial space producing less
accurate results compared to the standard LDFE solution, at least for sufficiently fine meshes where LD
can accurately represent the solution.  Although the closure will be applied everywhere,
we expect the greatest improvement in accuracy for cells where the LDFE trial space
produces a negative solution.


\section*{ACKNOWLEDGEMENTS}

This research was supported with funding received from the DOE Office of Nuclear
Energy's Nuclear Energy University Programs, the DOE National
Nuclear Security Administration, under Award Number(s) DE-NA0002376, and under Los Alamos National Security,
LLC, for the National Nuclear Security Administration of the U.S. Department of
Energy under contract DE-AC52-06NA25396. 

\section{Future Work} 

SPATiAL CLOSURE:
Additionally, because these closure
parameters are modifications to standard spatial closures, it should be possible to filter bad
values of $\gamma^{HO}_i$.  Future studies will investigate the stablity of
this method more rigorously using a linear fourier stability analysis.

