
\section{Preservation of the Discrete Maximum Principle}

An important property of a discretization of the TRT equations is preservation of the
discrete maximum principle (MP).  Violation of the maximum principle results in
the material temperature being artificially higher than the boundary conditions and
sources should physically allow. As discussed in Sec.~\ref{sec:intro}, IMC can violate the MP due to the approximate
linearization of the emission source in the time discretization; it is not truly implicit in time. 
We expect our
method, with a truly implicit time discretization, to preserve the MP with sufficient
convergence of the nonlinear emission source~\cite{larsen_mpv}.

To numerically demonstrate that our method preserves the MP, we have simulated problems similar to those in~\cite{wollaber2013discrete}.
We modify the Marshak wave problem in Sec.~\ref{sec:marshak???} to produce a problem which
results in MP violations for IMC at various time step sizes.  The material specifications
are given in Table~\ref{tab:mpv_prob}. The domain width is 1.0 cm wih $N_c=150$ spatial mesh cells.  The radiation and material energies are initially in
equilibrium at $0.01$ keV, before an isotropic boundary source of $1$ keV is applied at
the left boundary at $t=0$. The simulation is ran unitl $t=0.1$ sh. 

The material and radiation temperature are given for an IMC simulation in Figure~\ref{fig:imc_mpv1}.  Figure~\ref{fig:imc_mpv} depicts the material temperature for various time step sizes and a ifxed
mesh size of 150 equally spaced cells. All IMC simulations used 100,000 histories per time
step. As demonstrated in Fig.~\ref{fig:imc_mpv1}, the material temperature exceeds the
specified boundary temperature and is artificially hotter than the radiation temperature.
This artificial ``temperature spike'' also leads to a slower propagation of the wave.
As shown in Fig.~\ref{fig:imc_mpv}, as larger timestep sizes are taken the nonphysical results worsen.

The simulations are repeated with the same specifications for the HOLO method. As seen in
Figure~\ref{fig:holo_mpv}, the TRT solution does not violate the maximum principle. It was
necessary to use a damped Newton's method to converge the solutions.  A damping factor of
0.5 was used for all these simulations.  All solution obey the maximum principle.
Table~\ref{tab:blah} demonstrates the LO Newton iteration counts for the HOLO metthod.





The spatial and temporal discretization affects the appearance of MP violations for
IMC~\cite{wollaber2013discrete}. In particular, if time steps are too large or spatial
mesh cells are too small, IMC will demonstrate MP violations.  Here, we have kept the
spatial mesh size fixed and increased time step to make MP violations appear.

8 $\mu$ cells, 3 batches of 6,000 particles each.

The radiation boundary source temperature is at $1$ keV. The fact that the material
has exceeded teh boundary condition is referred to as a MP violation.


The damped newton is necessary.

NO FIXUP APPLIED, NEWTON CONVERGENCE OF 10e-06.  MODIFIED MARSHAK WAVE PROBLEM.

ACCURACY ALSO AN ISSUE.  even for short time step at early time steps it does appear

\begin{table}[H]
        \caption{\label{tab:mpv_prob}Problem specifications for maximum principle
        violation. Absorption cross section has form $\sigma_a = \sigma_{a,0}/T^3$.}
\centering
        \begin{tabular}{|c|c|} \hline \\
            $\sigma_{a,0}$ (cm$^-1$ keV$^3$)  & 4.0  \\
            $\sigma_s$ (cm$^{-1}$) & 0.0 \\
            $\rho$ (g cm$^-3$) & 1.0  \\
            $c_v$ (jks/keV-g) & 0.0081181  \\ 
        \end{tabular}
\end{table}



   An isotropic incident intensity of 0.150 keV is applied
at $x=0$; the incident intensity on the right boundary is $2.5\times10^{-5}$ keV.
The material properties are $\rho = 1$ g cm$^{-3}$ and $c_v = 0.013784$ jks/keV-g. The
absorption cross section varies as $\sigma(T) = 0.001\;\rho\; T^{-3}$ (cm$^{-1}$).
The simulation was advanced until $t=5$~sh~(1~sh~$\equiv$~10$^{-8}$~s) with a fixed time step size of 0.001 sh. For comparison purposes, we
have not used adaptive mesh
refinement, only performed one HOLO iteration per time
step, and use a fixed 3 HO batches with equal number of histories per batch. A
relative tolerance of $10^{-6}$ for the change in $\phi(x)$ and $T(x)$ was used for
the LO newton solver for all results. Radiation energy
distributions are plotted as an equivalent temperature given by
$T_r=(\phi/(ac))^{0.25}$.  Cell-averaged quantities are plotted.
Although isotropic scattering can be included in the LO solver with this method~\cite{ans_2014}, we have only
considered problems with $\sigma_s = 0$ here.  
