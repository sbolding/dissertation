\section{MC Estimation of the Spatial Closure}

The HO MC solution can be used to estimate additional information in closing
the LO system.  A half-range balance equation for $\mu>0$ is formed by adding the
exact $L$ and
$R$ moment equations given by Eq.~\eqref{??????}~and~\eqref{??}, i.e.,
\begin{equation}
    \overline\mu^+_{i+1/2}\phi_{i+1/2}^+ - \overline\mu^+_{i-1/2}\phi_{i+1/2}^+ +
    \frac{\sigma_{a,i}}h_i}{2} \phi_i^+ = \frac{h_i}{2} q_i,
\end{equation}
where $q$ is a general, isotropic source.  The angular consistency terms can be estimated
with the HO solution and the inflow term $\phi_{i-1/2}^+$ is upwinded from the previous
cell.  An additional equation is needed to eliminate $\phi_{i+1/2}^+$ to produce an
equation for a single unknown $\phi_{i+1/2}^+$.  We choose a parameteric combination of
\begin{equation}
    \phi_{i+1/2}^+ = f(\gamma_i, \phi_i^+, \phi_x^+)
\end{equation}
where $\gamma$ is some constant and $f$ is some function of the input variables.  We will
explore two different closure options: a scaled slope, i.e.,
\begin{equation}
    \phi_{i\pm1/2}^\pm = \phi_i^+ \pm \gamma_i \phi_x^+
\end{equation}
and a scaled average
\begin{equation}
    \phi_{i\pm1/2}^\pm = \gamma_i \phi_i^+ \pm \phi_x^+,
\end{equation}
where a value of $\gamma_i = 1$ produces the standard linear discontinuous expressions for
the extrapolated outflows.  The 
closures are only used to eliminate the extrapolated face values, not the inflow values
for the particular direction equation.

We now use the HO solution to estimate $\gamma_i$.  For example, 
\begin{equation}
    \gamma_i^{+,HO} = \frac{\phi_{i+1/2}^+ - \phi_x^+}{\phi_i^+}
\end{equation}
in the scaled slope case.  For this closure, as the slope goes to zero this expression
becomes undefined.  In cells where the slope is $O(10^{-13} \psi_i)$, we use $\gamma_i=1$.
No problems have been observed with the fact that at relatively modest slopes $\gamma$
becomes very large.  The main reason for using this closure is it allows for values of
$\gamma$ that are equivalent to step and lumping.

The expression for the outflow face term is substituted in each equation, using the
$\gamma$ estimated from a HO solution.
For instance, the positive balance equation becomes
\begin{equation}
    \overline\mu^+_{i+1/2}\left( \gamma_i^{+,HO} \phi_i^+ + \phi_x^+ \right) - \overline\mu^+_{i-1/2}\phi_{i+1/2}^+ +
    \frac{\sigma_{a,i}}h_i}{2} \phi_i^+ = \frac{h_i}{2} q_i,
    \equation{eqn:clsd_posbal}
\end{equation}
noting that $\phi_i^+$ and $\phi_x^+$ remain as unknowns.

Our solution is in terms of the $L$ and $R$ moments, however.
The MC solution provides evaluation of the face values. 



By introducing a LDD trial space into the LO equations, we introduce an additional
unknown in each half-range equation.  This extra unknown is eliminated as a function
of the interior moments.  
However, in the HOLO context, the HO solution used a lagged source $q_i$.

In theory, if the problem were linear, or the nonlinear problem was fully converged,
then the HO and LO solutions would produce exactly the same moments.  There are
several issues with ECMC that cause this to not be true, even for a linear problem.
With ECMC, global energy balance is not preserved, but local energy balance is also
not preserved.  There are source biasing techniques for standard MC (systematic
sampling) that would exactly preserve the zeroth moment of the source~\cite{shultis_mc}). 
It is a requirement that the HO solution satisfies the zeroth moment equation. If the
closure relation also uses the first moment, then it must also satisfy the first
spatial moment equation.  However, because we have to reconstruct the bilnear moment
of $x$ and $\mu$, the consistency terms do not exactly preserve the first moment
equation.  One final reason is that the analog treatment of absorption (at low
weights as discussed in Sec.~\ref{???}) results in the fact that the product of
$\phi_i$ (as estimated via path-length estimators) and the removal cross section 
flux does not exactly equal the absorption density of weight within that cell, due to
statistical noise.  
