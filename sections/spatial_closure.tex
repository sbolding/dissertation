\section{MC Estimation of the Spatial Closure}

By introducing a LDD trial space into the LO equations, we introduce an additional
unknown in each half-range equation.  This extra unknown is eliminated as a function
of the interior moments.  

In theory, if the problem were linear, or the nonlinear problem was fully converged,
then the HO and LO solutions would produce exactly the same moments.  There are
several issues with ECMC that cause this to not be true, even for a linear problem.
With ECMC, global energy balance is not preserved, but local energy balance is also
not preserved.  There are source biasing techniques for standard MC (systematic
sampling) that would exactly preserve the zeroth moment of the source~\cite{shultis_mc}). 
It is a requirement that the HO solution satisfies the zeroth moment equation. If the
closure relation also uses the first moment, then it must also satisfy the first
spatial moment equation.  However, because we have to reconstruct the bilnear moment
of $x$ and $\mu$, the consistency terms do not exactly preserve the first moment
equation.  One final reason is that the analog treatment of absorption (at low
weights as discussed in Sec.~\ref{???}) results in the fact that the product of
$\phi_i$ (as estimated via path-length estimators) and the removal cross section 
flux does not exactly equal the absorption density of weight within that cell, due to
statistical noise.  
