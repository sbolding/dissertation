
\section{Estimating the Spatial Closure from the HO Solution}



\section{MC Estimation of the Spatial Closure}

This sections describes an alternative spatial closure to the LO equations based on 
a parametric estimate from the HO solution. 
After algebraically manipulating the time-discretized moment equations to form angular
consistency parameters, there is still more unknowns than equations, for each cell, due to
the .  A half-range balance equation for $\mu>0$ is formed by adding the
exact $L$ and
$R$ moment equations given by Eq.~\eqref{??????}~and~\eqref{??}, i.e.,
\begin{equation}
    \overline\mu^+_{i+1/2}\phi_{i+1/2}^+ - \overline\mu^+_{i-1/2}\phi_{i-1/2}^+ +
    {\sigma_{a,i}h_i} \phi_i^+ = \frac{h_i}{2} q_i,
\end{equation}
where $q$ is a general, isotropic source.  The angular consistency terms can be estimated
with the HO solution and the inflow term $\phi_{i-1/2}^+$ is upwinded from the previous
cell.  An additional equation is needed to eliminate $\phi_{i+1/2}^+$ to produce an
equation for a single unknown $\phi_{i}^+$.  The outflow can be eliminated as a parametric combination of the
other variables, i.e.,
\begin{equation}
    \phi_{i+1/2}^+ = f(\gamma^{HO}_i, \phi_i^+, \phi_x^+, \phi_{i-1/2}^+),
\end{equation}
where $\gamma^{HO}_i$ is some constant estimated with the HO solution and $f$ is some function of some number of input
variables. If the problem is linear, i.e., $q$ is known a priori, then application of this
closure will ensure that the HO and LO equations exactly preserve the same moments.  As
the problem is typically non-linear (e.g., scattering or thermal emission are included in
$q$), then this will not be strictly true.


We will
explore two different closure options: a scaled slope, i.e.,
\begin{equation}
    \phi_{i\pm1/2}^\pm = \phi_i^+ \pm \gamma_i \phi_x^+
\end{equation}
and a scaled average
\begin{equation}
    \phi_{i\pm1/2}^\pm = \gamma_i \phi_i^+ \pm \phi_x^+,
\end{equation}
where a value of $\gamma_i = 1$ produces the standard linear discontinuous expressions for
the extrapolated outflows.  The 
closures are only needed to eliminate the extrapolated face values, not the inflow values
for the particular direction equation.

We now use the HO solution to estimate $\gamma_i$.  For example, 
\begin{equation}
    \gamma_i^{+,HO} = \frac{\phi_{i+1/2}^+ - \phi_x^+}{\phi_i^+}
\end{equation}
in the scaled slope case.  For this closure, as the slope goes to zero this expression
becomes undefined.  In cells where the slope is $O(10^{-13} \psi_i)$, we use $\gamma_i=1$.
No problems have been observed with the fact that at relatively modest slopes $\gamma$
becomes very large because the solution is changing minimally in such sections. 
The main reason for using this closure is it allows for values of $\gamma$ that are equivalent to step and lumping.

The expression for the outflow face term is substituted in each equation, using the
$\gamma$ estimated from a HO solution.
For instance, the positive balance equation becomes
\begin{equation}
    \overline\mu^+_{i+1/2}\left( \gamma_i^{+,HO} \phi_i^+ + \phi_x^+ \right) - \overline\mu^+_{i-1/2}\phi_{i+1/2}^+ +
    \frac{\sigma_{a,i}h_i}{2} \phi_i^+ = \frac{h_i}{2} q_i,
    \label{eqn:clsd_posbal}
\end{equation}
noting that $\phi_i^+$ and $\phi_x^+$ remain as unknowns. The MC solution can provides the
face values.

Our solution is in terms of the $L$ and $R$ moments, however.

\subsection{LO Equations Trial Space}

The
linear shape on the interior of the cell is still used to represent the temperature
unknowns, scattering source, and emission source.  The choice of this 
trial space should allow for higher consistency between the HO and LO
equations by using MC to compute outflows from space-angle cells, rather than an LD
extrapolation of the solution.    For the HO solution, a discontinuous outflow for a space-angle cell should produce greater
angular accuracy on faces.  In optically thick cells, this will mitigate observed issues with the spatial slope being
artificially high to account for the discrepancy in angular shape between the
internal and face solutions. The outflow will be estimated using a
face-based tally of the MC solution.  The error equation can be modified to estimate the outflow for a given batch. 
For the LO equations, a parametric relation between the outflow and internal moments, as
estimated by the latest HO solution,
will be used to eliminate the extra  unknowns.  This will produce a consistent spatial closure between the HO and LO
solutions upon convergence. The closure relation will be averaged
over each half-range, similar to the angular consistency parameters.   Poor statistics for the face tallies may result in this trial space producing less
accurate results compared to the standard LDFE solution, at least for sufficiently fine meshes where LD
can accurately represent the solution.  Although the closure will be applied everywhere,
we expect the greatest improvement in accuracy for cells where the LDFE trial space
produces a negative solution.

By introducing a LDD trial space into the LO equations, we introduce an additional
unknown in each half-range equation.  This extra unknown is eliminated as a function
of the interior moments.  
However, in the HOLO context, the HO solution used a lagged source $q_i$.

In theory, if the problem were linear, or the nonlinear problem was fully converged,
then the HO and LO solutions would produce exactly the same moments.  There are
several issues with ECMC that cause this to not be true, even for a linear problem.
With ECMC, global energy balance is not preserved, but local energy balance is also
not preserved.  There are source biasing techniques for standard MC (systematic
sampling) that would exactly preserve the zeroth moment of the source~\cite{shultis_mc}). 
It is a requirement that the HO solution satisfies the zeroth moment equation. If the
closure relation also uses the first moment, then it must also satisfy the first
spatial moment equation.  However, because we have to reconstruct the bilnear moment
of $x$ and $\mu$, the consistency terms do not exactly preserve the first moment
equation.  One final reason is that the analog treatment of absorption (at low
weights as discussed in Sec.~\ref{???}) results in the fact that the product of
$\phi_i$ (as estimated via path-length estimators) and the removal cross section 
flux does not exactly equal the absorption density of weight within that cell, due to
statistical noise.  
