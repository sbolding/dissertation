%%%%%%%%%%%%%%%%%%%%%%%%%%%%%%%%%%%%%%%%%%%%%%%%%%%
%
%  New template code for TAMU Theses and Dissertations starting Fall 2012.  
%  For more info about this template or the 
%  TAMU LaTeX User's Group, see http://www.howdy.me/.
%
%  Author: Wendy Lynn Turner 
%	 Version 1.0 
%  Last updated 8/5/2012
%
%%%%%%%%%%%%%%%%%%%%%%%%%%%%%%%%%%%%%%%%%%%%%%%%%%%

%%%%%%%%%%%%%%%%%%%%%%%%%%%%%%%%%%%%%%%%%%%%%%%%%%%%%%%%%%%%%%%%%%%%%%
%%                           SECTION I
%%%%%%%%%%%%%%%%%%%%%%%%%%%%%%%%%%%%%%%%%%%%%%%%%%%%%%%%%%%%%%%%%%%%%


\pagestyle{plain} % No headers, just page numbers
\pagenumbering{arabic} % Arabic numerals
\setcounter{page}{1}

\chapter{\uppercase {Introduction}}


Accurate solutions to the thermal radiative transfer (TRT) equations are important to
simulations in the
high-energy, high-density physics regime, e.g., for inertial
confinement fusion and astrophysics.  Moment-based hybrid Monte Carlo (MC)
methods have demonstrated great potential for accelerated
solutions to TRT problems~\cite{rmc,bolding_nse,holo_rh}.   These nonlinear acceleration methods iterate between a
high-order (HO) transport equation and a low-order (LO) system formulated with angular moments
and a fixed spatial mesh.  Physics operators that
are too expensive for the HO solver to resolve directly, e.g., photon absorption and emission,
are moved to the LO system. The lower-rank LO equations can be solved with Newton
methods to allow for non-linearities in the LO equations to be efficiently
resolved.  The high-order (HO) problem is defined by the radiation transport equation with
sources computed with the previous LO solution. A MC transport solution to the HO
problem is used to construct consistency terms that appear in the LO equations. These consistency terms preserve the accuracy of the HO
solution in the next LO solve.


We have developed a new high-order low-order (HOLO) algorithm for solving
thermal radiative transfer (TRT) problems. This algorithm has several desirable
properties that improve on current computational methods for solving TRT
problems.  In particular, our HOLO method utilizes an exponentially-convergent
Monte Carlo (ECMC) algorithm to solve the associated radiation transport
equation.  The ECMC method significantly decreases the statistical noise
associated with MC transport calculations for TRT problems. 
The method is exceptionally efficient in time-dependent TRT problems because information about the
intensity from the previous time step can be used as an accurate initial guess for
the new end of time step intensity.   However, implementation
of ECMC is non-trivial, requiring a finite-element representation of the solution in
all phase-space variables that are being sampled with MC. 

In conjunction with the ECMC
algorithm, we use a nonlinear
low-order (LO) system that is fully-implicity in time and resolves the nonlinear dependencies in the system to be
efficiently resolved.  The lower-dimensional equations are formed such that the LO
system can preserve the
accuracy of the ECMC treatment of particle transport, upon convergence of the
outer iterations between the HO and LO solvers. Our LO system and approach to enforcing consistency contrast from the formulation other moment-based acceleration methods, e.g., those 
in~\cite{rmc,willert,park}. The LO equations are formed with finite-element (FE) based spatial
moments and angular moments over each half-range.  A linear-discontinuous representation
is used to discretize the temperature field, and different closures of the LO
equations have been investigated.

 The HOLO algorithm has been
developed and implemented for a simplified model with one spatial dimension and
frequency-integrated equations. Throughout this work, we compare our
method to the implicit Monte Carlo (IMC) method~\cite{fnc}, which is the standard MC
transport solution method to the TRT equations.  Results are compared to the IMC
method for one-dimensional, gray test problems and also
demonstrate the efficiency of ECMC over standard Monte Carlo in this HOLO algorithm.
The HOLO method contains many of the desired qualities, such as
preserving the equilibrium diffusion limit, preserving the maximum principle, and in
particular, providing high-fidelity MC solution to the TRT equations in an efficient
manner.

Sufficient MC histories must be performed to eliminate statistical
noise in the consistency terms that can contaminate the LO solution.
Exponentially-convergent Monte Carlo (ECMC)\cite{jake,ans_2014} provides an algorithm that can efficiently
reduce statistical noise to the same order as the HOLO iteration error with
significantly less particle histories than standard MC, 
with less computational cost, per history, in particle tracking than IMC because no scattering events occur. In particular, ECMC is
exceptionally efficient in time-dependent TRT problems because information about the
intensity from the previous time step can be used as an accurate initial guess for
the new end of time step intensity. Additionally, no particle histories are required
in regions where the radiation and material energy field are in equilibrium, similar to~\cite{rmc}.  However, implementation
of ECMC is non-trivial, requiring a finite-element representation of the solution in
all phase-space variables that are being sampled with MC.  The fundamental transport of particles is the same
as standard Monte Carlo transport codes, but the source will now contain positive and
negative weight particles.

We have also investigated several extensions and improvements of this method.  First, an alternative
solution method to the LO equations is explored, using an iterative source iteration
method with linear diffusion synthetic acceleration.  The goal is to present a solution
method for the S$_2$-like equations that is more extendable to higher dimensions.
Additionally, we have investaged solution methods to resolve issues when the optically
thick mesh cells present intensity gradients that are too difficult to resolve with the
LDFE interior mesh representation.   In the HO equations, we add artificial sources that
make the solution more easily representable by the chosen mesh resolution, without
altering the zeroth moment of the transport equation, neglecting statistical noise.  Finally, higher accuracy in the time
variable is investigated by including the time variable in the ECMC transport
algorithm.  This can produce more accuracy in the transport of particles across optically
thin regions, although a projection error in the intensity is still incurred between time
steps. The equations are closed in time consistently with a parametric modification
to standard time discretizations, as estimated by the ECMC solution.  This closure produces LO equations that have the same numerical difficulty to solve as the BE,
fully-discrete LO equations, but
have the potential to preserve the accuracy of the MC integration in time, upon non-linear
convergence of the system.  

In addition to estimating the angular
consistency terms, the HO intensity estimates a relation between volume and face-averaged
intensities to eliminate the remaining unknowns from the equations.  The goal is for the 
LO moments to reproduce the HO moments more accurately than the LDFE discretization, although
additional statistical noise is introduced through face-based tallies.
A projection of the intensity
that is separate from the linear extrapolation allows for an accuracy projection of the MC
solution at faces, which is incorporated through the HO spatial closure of 
the LO equations discussed in Sec.~\ref{sec:spat_clos}.  Additionally, the independent solution at faces allows for face-averaged
consistency terms (e.g., Eq.~\eqref{eq:ang_cons_face}) to be directly evaluated. 

The exponentially-convergent Monte
Carlo (ECMC) algorithm was modified to include integration
of the time variable; this includes the introduction of a step,
doubly-discontinuous (SDD) trial space representation in time.
A new parametric closure of the LO equations, introducing
additional time-closure consistency terms, was derived to cap-
ture the time accuracy of the HO ECMC simulations. The LO
equations can preserve the accuracy of MC radiation trans-
port treatment in time, with the same numerical expense as
Backward Euler (BE) time-discretized S$_2$ equations.
 The main interest is in increasing accuracy in resolution of radiation wavefronts in optically
thin regions, where particles transport a long distance over a time step. In such regions, the MC
integration of the time variable in the transport equation, as done by IMC, can produce greater
accuracy than a backward
Euler (BE) discretization, which propagates radiation energy through space artificially fast.
A potential application where this accuracy is important is stellar atmosphere calculations. 





\section{Outline of the Disertation}


In the remainder of this chapter, a brief background on TRT problems is given, before the
simplified model that will be used for the remainder of the work is derived.  Then,
related work on MC and HOLO solution methods for TRT are discussed.
 Herein, a brief description of thermal radiative transfer and the
model problem are given, followed by a discussion of the standard Monte Carlo
solution method and other related research.  Finally, a detailed outline of the
dissertation is given.

In Chapter~\ref{chp:holo} an overview of the outer HOLO algorithm and how the HO and LO
systems interact is given.  For the majority of this work, time-discretized equations will
be assumed; in Chapter~\ref{chp:time} a higher-accuracy MC time treatment is investigated. Chapter~\ref{chp:lo} gives a detailed derivation of the LO
equations and how they are solved.  Then, Chapter~\ref{chp:ho} details the ECMC algorithm
that is used to solve the HO transport problem.
Some of the work discussed in this dissertation was previously published
in~\cite{bolding_nse}.



\section{Thermal Radiative Transfer Background}

Thermal radiative transfer (TRT) physics are used to model the time-dependent coupling between a photon
radiation field and a high-temperature material, which is typically a plasma.  The desired
transient unknowns are the spatial
energy-density distributions of the radiation and material.  As photons transport through
the medium, they interact through scattering and absorption by the material, depositing
momentum and energy.  The
material is heated through absorption of photons and is cooled by emission of thermal
x-ray photons
into the radiation field.  The emission process is a strongly nonlinear
function of temperature~\cite{mihalas}.  Additionally, the  material properties are
typically a function of temperature, in particular the absorption cross section.  The
temperature-dependent material properties and
absorption and reemission physics lead to systems that require accurate modeling of
photon transport through a mix of
streaming and optically-thick, diffusive regions. 

Accurate modeling of TRT physics becomes relevant in the high-energy,
high-density physics regime.   Radiative transfer is a dominant form of heat transfer in
high-temperature systems, where the material temperature is $O(10^6)$ K or
higher. Typical computational applications of TRT include simulation of inertial confinement fusion and
astrophysics phenomena.  In most applications where TRT is important, the fluid
material is typically in motion and exchanges momentum with the radiation field. In this work, we neglect
motion of the material, which would require inclusion of hydrodynamics in our
model~\cite{mihalas}.  However, our LO equations are well-suited for
coupling to material motion via typical operator-splitting methods for
radiation-hydrodynamic systems~\cite{radhydro_code,os_rh}.

\subsection{The Equations of Thermal Radiative Transfer}

First, the photon radiation field, with the appropriate units used throughout this work, is
characterized.  Photons transporting through a material are described by the particle position vector
$\mathbf{r}$ (cm), direction vector $\mathbf\Omega$ (str, i.e., steradians),
time $t$ (sh, where $1\text{ sh}\equiv10^{-8}$ s), and frequency $\nu$ (Hz).  The primary
radiation unknown is the angular intensity $I \phsp$ (jk cm$^{-2}$ s$^{-1}$
Hz$^{-1}$ str$^{-1}$), which represents a distribution
function of energy
contained in the radiation field, per unit of
phase space.  We use the energy unit jerks (jk), where 1 $\text{jk}= 10^9$ joules. The intensity can be related to the volumetric density of photons
$N \phsp$ (photons cm$^{-3}$ Hz$^{-1}$ str$^{-1}$) via the relation
\begin{equation}\label{eq:intens_dens}
    I\phsp = c h \nu N \phsp,
\end{equation}
where $c$ is the speed of light (cm sh$^{-1}$) and $h$ is Planck's constant. The
angular intensity is a useful quantity because it is directly related to reaction rates.

The governing conservation equation for the radiation field is a transport equation given
by~\cite{mihalas,lewis,wollaber_thesis}
 \begin{multline}
     \frac{1}{c} \pderiv{I\phsp}{t} + \mathbf{\Omega}\cdot \Del I \phsp + \sigma_t
     (\mathbf{r},\nu) I \phsp = \\ \int \limits_0^\infty \int \limits_{4\pi}
     \sigma_s(\mathbf{\Omega'} \ra \mathbf{\Omega},\nu'\ra\nu)
     \phi(\mathbf{r}',\nu',t) \dd \Omega' \dd \nu' +
     \sigma_a(\mathbf{r},\nu)B_\nu(\mathbf{r},\nu,T),
 \end{multline}
where
\begin{equation}
    B_\nu(\mathbf{r},\nu,T) = \frac{2 h \nu^3}{c^2} \frac{1}{e^{h\nu/T} - 1}
\end{equation}
is the black-body Planckian emission
spectrum at temperature $T$ (keV)~\cite{mihalas}, and the macroscopic scattering, absorption, and total cross sections are $\sigma_s$,
$\sigma_a$, and $\sigma_t$, respectively. 
 The scattering source includes
integration over all possible incoming angles $\mathbf\Omega'$ in differential
solid angle $\dd \Omega'$.  The absorption cross section $\sigma_a$ is typically
a strong function of temperature, i.e., $\sigma_a\equiv \sigma_a(T)$.  Following standard notation, we
report temperatures in units of keV as an effective energy, obtained by multiplying by the Boltzmann
constant $k_B$~\cite{mihalas}.  Thus, all material temperatures are $T \equiv T_{K}
k_B$, where $k_B$ is the Boltzmann constant (keV K$^{-1}$) and $T_{K}$ is the temperature
in kelvin.

The material is characterized by the material internal
energy as a function of position.  The internal energy $e$ is related to the material
temperature $T$ through an equation of state.   In this work, a perfect gas equation of state is
assumed~\cite{toro}, which produces the relation $\rho c_v T = e$, where $\rho$ is the
material mass density and $c_v$ is the specific heat.  Thus, we will use $T(\mathbf{r},t)$ as the
primary unknown to describe the material energy distribution.  The material energy conservation equation is
\begin{equation}
    \rho(\mathbf{r}) c_v(\mathbf{r}) \pderiv{T(\mathbf{r},t)}{t} = \int\limits_0^\infty
    \left( \int\limits_{4\pi} \sigma_a I \phsp \dd \Omega - \sigma_a 4\pi B_\nu(\mathbf{r},\nu,T) \right) \dd \nu
 \end{equation}
 In derivation of the above equations, the conditions of local thermodynamic equilibrium
were assumed, i.e., the emission source is described point-wise by the
Planck function at the temperature at that position, and the material is well-described by
the local temperature~\cite{mihalas,wollaber_thesis}. The emission source is a non-linear function of temperature and is
proportional to $T^4$ after integration over frequency.  

\subsection{Derivation of 1D Grey Model}

At this point, we introduce the simplified equations that will be used in the remainder of
this work.  First, the solutions are assumed to only vary in one spatial dimension using
Cartesian coordinates, referred to as the 1D slab geometry~\cite{lewis}.   The
position is described by a single coordinate $x$ and the direction of particle travel
is described by $\mu$, which is the cosine of the angle between the particle direction and the
positive $x$ axis. The angular
intensity is assumed symmetric in angle azimuthally about the $x$ axis.  
To
simplify the equations, the equations are integrated over all frequencies.  We also assume that the material properties are independent of photon frequency, or
equivalently we know the weighting spectrum of the frequency integrated cross sections.   Finally, we assume physical scattering
is isotropic in angle. With these assumptions, integration over the azimuthal angle and
all frequencies, with algebraic manipulation, ultimately yields the 1D grey equations~\cite{wollaber_thesis,mihalas}
\begin{align}\label{eq:rad_cont}
    \frac{1}{c}\pderiv{I(x,\mu,t)}{t} + \mu \pderiv{I(x,\mu,t)}{x} + \sigma_t
    I(x,\mu,t)
&= \frac{\sigma_s}{2} \phi(x,t) +\frac{1}{2} \sigma_a a c T^4(x,t)
    \\ \label{eq:mat_cont}
  \rho c_v \pderiv{T(x,t)}{t} &=  \sigma_a \phi(x,t) - \sigma_a a c T^4(x,t).
\end{align}
The equations have associated incident boundary conditions for the angular intensity:
\begin{align}
    I(0,\mu) &= I^{inc,+}(\mu),\quad\quad \mu>0 \\
    I(X,\mu) &= I^{inc,-}(\mu), \quad\quad \mu<0,
\end{align}
for a spatial domain spanning $0\leq x \leq X$.
In the above equations the fundamental unknowns are the material temperature $T(x,t)$ and
the grey angular intensity $I(x,\mu,t)=\int\limits_0^\infty I(x,\mu,\nu,t) \dd \nu$. The mean radiation intensity $\phi(x,t)=\int_{-1}^1
I(x,\mu,t) \dd \mu$ is related to the radiation energy density
$E$ (jk cm$^{-3}$ sh$^{-1}$) by the relation $E = \phi/c$.  The integral of
$B_\nu(\mathbf{r},\nu,T)$ over all frequencies and angles produced the 
grey Planckian emission source $\sigma_a a c T^4$~\cite{mihalas} in
Eq.~\eqref{eq:mat_cont}, where $a=0.01372$ jk cm$^{-3}$
keV$^{4}$ is the radiation constant, which is proportional to the Stefan-Boltzmann
constant.  The term $\sigma_a \phi$ is the rate of energy absorption by the material,
whereas the emission term represents losses to the material internal energy.  We have developed
and tested our algorithm for efficient solution to Eq.~\eqref{eq:rad_cont}
and~\eqref{eq:mat_cont}.

\subsection{The Equilibrium Diffusion Limit}

A critical aspect for any numerical solution to the thermal radiative transfer equations
is preservation of the asymptotic, equilibrium diffusion limit (EDL)~\cite{morel_ldtrt,larsen_edl}.
In the EDL, the material becomes optically thick and increasingly diffusive, as $\sigma_a$ becomes large and
$\rho c_v$ becomes small.  The solution approaches equilibrium with $I(x,\mu)=\frac{1}{2}acT^4$,
where the distribution of the solution is well described by the material
temperature~\cite{larsen_edl}.  For problems in this limit, spatial mesh cells that resolve the spatial variation of the solution are
still $O(100)$ mfps or greater; the solution typically varies relatively smoothly in space.
Discretization schemes of the transport equation must correctly limit to the appropriate
discretized diffusion equation in the EDL.  Spatial discretizations that do not preserve
the EDL can produce inaccurate solution, even though the mesh size should accurately capture the variation of the solution, with
inaccuracies that are much greater than expected from truncation error.  Such methods
require spatial mesh resolution on the order of a mfp~\cite{morel_ldtrt}.  The EDL regime is typical in
applications of TRT, so discretizations must preserve this limit to prevent unnecessarily
fine mesh resolutions.

\section{Previous and Related Work}

This sections describes related work on Monte Carlo solution to the TRT equations, as well
as some important properties that numerical solution to TRT equations must preserve.  
The Monte Carlo (MC) method~\cite{shultis_mc} is a standard computational method in
the field of radiation transport.  It has been used to great success, providing high-accuracy
solutions to particle
transport problems described by the linear Boltzmann transport equation for many decades.  The application of MC to the linear Boltzmann
equation is well documented in
literature~\cite{mcnp,shultis_mc,lewis}.  The Monte Carlo method samples the underlying physics distributions to estimate the
average behavior of a field of particles.  This can provide highly-accurate results, in
particular for treatment of the angular variable associated with particle transport
problems.  Detailed descriptions of MC simulation of particle tracking, sampling of
interactions, etc. can be found in literature~\cite{mcnp,wollaber_review,shultis_mc}.

With respect to TRT problems, the temperature equation is almost always solved
deterministically to produce a linear particle transport equation. Monte carlo solution to
this transport equation can introduce large statistical
noise into the material temperature distribution, which is undesirable when coupling to
other physics, e.g., in radiation hydrodynamics.  To improve the efficiency of MC solutions, hybrid MC methods utilize a
deterministic solution to accelerate the MC solution.  

In the remainder of this section, we detail the standard method for MC solution to TRT
equations, the implicit Monte Carlo (IMC) method, and then discuss related moment-based acceleration and other
alternative solution methods.  We also discuss the residual Monte Carlo (RMC) method, which is
similar to the HO solver in our method.

\subsection{The Implicit Monte Carlo Method}
\label{sec:imc}

The IMC method~\cite{fnc,wollaber_review} is the standard approach for applying the MC
method to TRT problems.  The IMC method partially linearizes Eq.~\eqref{eq:rad_cont} \&
Eq.~\eqref{eq:mat_cont} over a discrete time step, with material properties evaluated at
the previous-time-step temperature.  Linearization of the system produces a linear
transport equation that can be solved with MC simulation.  The transport equation contains
an approximate emission source and an effective scattering cross section representing
absorption and reemission of photons over a time step.  The transport equation is solved
with MC simulation to advance the distribution of radiation to the end of the time step
and determine the energy absorbed by the material over the time step.  The energy
absorption by the material is tallied over a discrete spatial mesh, computed with
cell-averaged quantities.  Integration of the time-variable is treated continuously for
radiation variables over the time step via MC sampling, but the linearized Planckian
source in the transport equation is based on a time-discrete approximation. 

The IMC method has some notable limitations.  In optically thick regions, or for large time steps,
the effective scattering dominates interactions.  In these diffusive regions IMC becomes
computationally expensive. Acceleration methods typically attempt to improve efficiency by
allowing particles to take discrete steps through optically-thick regions based on a
spatially-discretized diffusion approximation~\cite{imd,ddmc}. 
In IMC, temperature-dependent material
properties, in particular cross sections, are evaluated at the previous-time step
temperature. These lagged cross sections can produce inaccurate solutions but do not cause
stability issues.  

An important aspect for numerical simulation of TRT equations is preservation of the
discrete maximum principle (MP). The maximum principle states that the material temperature and mean intensity in the
interior of the domain should be bounded by the solution at the boundaries of the domain, in the
absence of interior energy sources~\cite{wollaber2013discrete,larsen_mpv}.  The analytic solution to the TRT equations satisfies a maximum
principle~\cite{larsen_mpv}, so we desire numerical approximations that preserve the MP in
a discrete sense, for each time step. For some problems, the IMC method can yield non-physical results that violate the MP if the time
step size is too large or the cell size is too small~\cite{wollaber2013discrete}. The violation of the maximum principle results in the material temperature being artificially
higher than the effective radiation temperature. 
The violation by IMC is caused by the approximate linearization of the end-of-time-step emission source; the emission source
is not truly implicit in time. The linearized estimate of the emission source typically can not be iteratively improved due
to the high computational cost of the MC transport.   
 The work in~\cite{iimc_gentile} uses less-expensive MC
iterations to produce an implicit system which prevents this from happening, but the
method has very
slow iterative convergence in diffusive problems.  

In IMC the material and radiation energy fields are discretized spatially to solve for cell-averaged values.
Inaccurate spatial representation of the emission source over a cell can result in
energy propagating through the domain artificially fast, yielding non-physical
results that are often referred to as ``teleportation error"~\cite{teleportation}.  The IMC method uses a fixup known as source tilting
to mitigate this problem.  Source tilting reconstructs a more accurate
linear-discontinuous representation of the
emission source within a cell based on the cell-averaged material temperatures in adjacent
cells.  Recent work in IMC has incorporated a linear-discontinuous finite-element representation directly into 
the discretization of the material temperature equation~\cite{wollaeger_ld}.

%PICTURE OF STANDARD MC SOLUTION AND MAXIMUM PRINCIPLE VIOLATION?

%WHAT IS UP WITH THE WEIRD DIFFUSION LIMIT

For TRT simulations, inaccurate spatial representation of the emission source over a cell can result in
energy propagating through the domain artificially fast, yielding non-physical
results referred to as ``teleportation error"~\cite{teleportation}.  The IMC method uses a fixup known as source tilting
to mitigate this problem.  Source tilting reconstructs a more accurate
linear-discontinuous representation of the
emission source within a cell based on the cell-averaged material temperatures in adjacent
cells. This linear reconstruction is also necessary to preserve the asymptotic equilibrium diffusion
limit (EDL), at least for a more general time step size and class of problems than for a piece-wise constant representation~\cite{diff_limit_imc}.  

\subsection{Moment-Based Acceleration Methods}

An alternative application of MC to the TRT equations is moment-based hybrid MC methods.
Recent work has focused on so-called high-order low-order (HOLO)
methods~\cite{willert,park,rmc,ans_2014,bolding_nse}. These methods involve fixed-point
iterations between high-order (HO) MC solution of a transport equation and a deterministic LO
system.  The low-order (LO)
operator is based on angular moments of the transport equation, formulated over a fixed
spatial mesh.  Physics operators that are time consuming for MC
to resolve, e.g., absorption-reemission physics, are moved to the LO
system.  The reduced angular dimensionality of the system and Newton methods allow for non-linearities in the LO equations to be fully
resolved efficiently~\cite{willert,park}.  The high-order (HO) transport problem is defined by 
Eq.~\eqref{eq:rad_cont}, with sources estimated from the previous LO solution.  
The high-order (HO) transport equation is solved via MC to produce a high-fidelity solution for
the angular intensity.  The MC estimate of the angular intensity is used to estimate
consistency terms,
present in the LO equations, that require the LO system to preserve the angular accuracy of the
MC solution.   
These consistency terms are present in all spatial-regions of the problem, requiring
statistical variance to be reduced sufficiently throughout the entire domain of the
problem.   The LO equations are typically based on nonlinear Diffusion Acceleration
(NDA)~\cite{willert,park}.  The HOLO method approach developed in this work was applied to
simplified 1D neutronics problems previously~\cite{ans_2014}.

\subsection{Residual Monte Carlo Methods}

Another area of related research is the application of
residual Monte Carlo to TRT problems.  The goal of these methods is to solve an auxiliary transport
equation with MC simulation for the error in some estimate of the intensity.  The error is then added to the
estimate of the solution, which can produce an overall solution for the intensity that has
less statistical noise than solution of the original transport equation would produce.
The work in~\cite{rmc} used residual MC as a HO solver for 1D grey TRT problems.
In~\cite{rmc}, the residual is formed with a fixed estimate of the solution, based
on the previous time intensity, such that only
sources on the faces of cells must be sampled. This reduces the dimension of the
phase-space to be sampled~\cite{rmc}. The RMC algorithm demonstrated impressive reduction
in statistical variance for slowly varying solutions.  However, a
piecewise constant representation is used for the space-angle representation of the
intensity, which does not preserve the EDL and can be inaccurate in angularly complex
regions of the problem.  In this work, we apply the exponentially convergence MC (ECMC)
algorithm that was applied to
simplified steady state neutronics problems previously~\cite{jake}.

Similar to RMC, a difference formulation has been applied to another algorithm known as the symbolic IMC method
(SIMC), for the case of 1D frequency-dependent problems~\cite{simc_const}.  SIMC forms a
standard FE solution to the material energy balance equation, and uses symbolic
weights in the MC transport to solve for expansion coefficents.  The difference
formulation modifies the transport equation to solve for unknowns representing the
deviation of the intensity from
equilibrium with the material energy.  The difference
formulation was also applied to a linear-discontinuous FE spatial
representation of the emission source, demonstrating accuracy in the EDL~\cite{simc}. 
Both~\cite{simc_const} and~\cite{rmc} produced minimal
statistical noise in slowly varying problems where the behavior of the system is near
equilibrium. 



