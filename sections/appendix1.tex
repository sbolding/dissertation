%%%%%%%%%%%%%%%%%%%%%%%%%%%%%%%%%%%%%%%%%%%%%%%%%%%
%
%  New template code for TAMU Theses and Dissertations starting Fall 2012.  
%  For more info about this template or the 
%  TAMU LaTeX User's Group, see http://www.howdy.me/.
%
%  Author: Wendy Lynn Turner 
%	 Version 1.0 
%  Last updated 8/5/2012
%
%%%%%%%%%%%%%%%%%%%%%%%%%%%%%%%%%%%%%%%%%%%%%%%%%%%

%%%%%%%%%%%%%%%%%%%%%%%%%%%%%%%%%%%%%%%%%%%%%%%%%%%%%%%%%%%%%%%%%%%%%%
%%                           APPENDIX A 
%%%%%%%%%%%%%%%%%%%%%%%%%%%%%%%%%%%%%%%%%%%%%%%%%%%%%%%%%%%%%%%%%%%%%

\phantomsection

\chapter{\uppercase{First Appendix}}

\section{Useful Moment Relations for LO Equations}
\label{app:lo_mom_relations}

There are several relations between various moment definitions that are useful in
derivation and manipulation of the LO equations. The following are derived for $\phi(x)$,
but can be applied to general moments of functions.  The volumetric average terms
can be eliminated in terms of the $L$ and $R$ moments from the relation
$b_{L,i}(x)+b_{R,i}(x)=1$.
\begin{align}\label{app:atoL}
    \phi_{i} &= \frac{1}{h_i} \int\limits_\xl^\xr 1 \;\phi(x) \dd x \\
             &= \frac{1}{h_i} \left(\int\limits_\xl^\xr b_{L,i}(x) \phi(x) \dd x +
             \int\limits_\xl^\xr b_{R,i}(x) \phi(x) \dd x \right)\\
             &= \frac{1}{2} \left( \mom{\phi}_{L,i} + \mom{\phi}_{R,i} \right)
\end{align}
A similar relation can be derived for the first moment in space as
\begin{equation}
    {\phi}_{x,i} = \frac{3}{2} \left(\mom{\phi}_{L,i} + \mom{\phi}_{R,i}\right)
\end{equation}
The above relations can be inverted to derived a relation for the $L$ and $R$ moments in terms of the slope
and average moments.  These moment expressions are defined purely in terms of integrals,
and are independent of the chosen spatial representation

Once a linear relation on the interior has been assumed, there are other useful closures that can be
derived.  The standard linear interpolatory expansion, for the positive half-range, is restated here:
\begin{equation}
\phi^+(x) = \phi^+_{L,i} b_{L,i}(x) + \phi^+_{R,i} b_{R,i}(x)
\end{equation}
Using this expansion, one can derive a relation between the outflow from a cell and the
hat function moments that is equivalent to the standard LDFE Galerkin method:
\begin{equation}
    \phi_{i,R}^+ = 2 \mom{\phi}_{R,i}^+ - \mom{\phi}_{L,i}^+
\end{equation}
this linear relation also defines the value for $\phi_{i,L}^+$.

To eliminate the LO unknowns in a manner that is equivalent to lumping the discrete system, the
following expression can be used for the outflow from a cell
\begin{equation}
    \phi_{i+1/2}^+ = \phi_i^+ + \frac{\phi_x^+}{3},
\end{equation}
which in terms of the hat function moments is equivalent to $\phi_{i+1/2}^+ =
\mom{\phi}_{R,i}^+$.  Inserting this expression into Eq.~\eqref{}, and using the same
definition for the linear representation over the interior of $\phi_{i+1/2}^+(x) =
\phi_{L,i} b_{L,i}(x) + \phi_{R,i} b_{R,i}(x)$, will produce an equivalent set of unknowns
as a linear discontinuous method with a lumped representation for the radiation.  The
temperature equation must be independently lumped. This
relation preserves the average within a cell but modifies the first moment.  

A similar expression produces a lumped-equivalent representation on the interior of the cell:
\begin{equation}
    \phi_{i,R}^+ = \phi_i^+ + \frac{\phi_x^+}{3},
\end{equation}
The moment equations are not modified by using this expression, however the interpretation
of the moments as a linear representation over the cell has been altered.  This allows for
us to ensure a lumped representation on the interior while still using the HO solution to
eliminate the outflow from the equations.


\section{Newtons Method for the LO Equations}
\label{app:lo_newton}

Because we have only considered problems with constant densities and heat capacities, the
linearization described below is in terms of temperature $T$ rather than material internal
energy, for simplicity. However, the linearization can be formed in terms of internal energy
to apply this method to a general equation of state.

To formulate the Newton iterations, the Planckian source is linearized in the material and radiation equations (Eq.~\eqref{??}
\& Eq.~\eqref{???}).
Application of the first order Taylor expansion in time to the
implicit emission source $B(T^{n+1})$, about some temperature $T^*$ at some
time $t^*\in[t^{n},t^{n+1}]$, yields
\begin{equation}\label{new_planck}
    \sigma_a^{n+1} a c T^{4,n+1} \simeq \sigma_a^* a c \left[T^{*4} + (T^{n+1} - T^*) 4T^{*3} \right]
\end{equation}
where $\sigma_a^*\equiv\sigma_a(T^*)$.  Substitution of this expression into Eq.~\eqref{eq:mat_cont} yields
\begin{equation}
    \rho c_v \left( \frac{T^{n+1} - T^{n}}{\Delta t} \right) = \sigma_a^* \phi^{n+1} -
    \sigma_a^* a c \left[ T^{*4} +  (T^{n+1} - T^*) 4T^{*3} \right].
\end{equation}
Algebraic manipulation of this equation yields an expression for $T^{n+1} - T^{*}$:
\begin{align*}
\left( T^{n+1} - T^* \right) &= \frac{ {\displaystyle \frac{\sigma_a^* \Delta t}{\rho
c_v}}  \left[ \phi^{n+1} -  a c T^{*4} \right] + (T^n - T^*) }{1 +
        \sigma_a^* a c \Delta t\frac{\displaystyle 4
T^{*3}}{\displaystyle \rho c_v } }.
\end{align*}
%This provides an expression for $T^{n+1}$ as a
%function of $T^*$ and the mean intensity $\phi^{n+1}$, i.e.,
%\begin{equation}
%\label{lo_t_new}
%T^{n+1}  = \frac{1}{\rho c_v } f\sigma_a^* \Delta t \left( \phi^{n+1} - c a T^{*4} \right)
%+ f T^n + (1-f) T^*.
%\end{equation}
This expression is substituted back into Eq.~\eqref{new_planck} to form
an explicit approximation for the emission source at $t^{n+1}$ as
\begin{equation}\label{t_next1}
    \sigma_a a c T^{4,n+1} \simeq \sigma_a^* (1 -f^*) \phi^{n+1}
    + f^* \sigma_a^* a c T^{4,n} + \rho c_v\frac{1-f^*}{\Delta t} (T^n - T^*)
\end{equation}
where $f^* = [1 + \sigma_a^* c \Delta t 4 a T^{*3}/(\rho c_v)]^{-1}$ is often referred to
as the Fleck factor~\cite{fnc}. 

Next, the above equation must be spatially discretized.  Application of the $L$ spatial
moment yields
\begin{multline}\label{eq:temp_const}
    \mom{\sigma_a^* a c T^{4,n+1}}_{L,i} = \sigma^*_{ai}(1-f_i^*)\mom{\phi^{n+1}}_{L,i} +
    f_i^*
    \sigma^*_{ai} a c \left(\frac{2}{3} T_{L,i}^{4,n} + \frac{1}{3} T_{R,i}^{4,n}\right)
    \\ \rho_i c_{vi} \frac{1 - f^*_i}{\Delta t} \left[\frac{2}{3}\left(T^n_{L,i} -
        T^*_{L,i}\right) + \frac{1}{3}\left(T^n_{R,i} -
    T^*_{R,i}\right)\right],
\end{multline}
where $T^{4,n}$ and $T^{n}$ have been assumed LD and $f^*$ is assumed constant over a cell, i.e., $f^*_i
\equiv \sigma_a(T_i^*)$.
The error introduced by a constant $f^*$ approaches zero as the
non-linearity is converged because $T^*$ approaches $T^{n+1}$. 
Based on an estimate for $T^*$, Eq.~\eqref{eq:temp_const} is an expression for
the Planckian emission source in the radiation moment equations with an additional effective scattering source.
A similar expression can be derived for $\mom{\sigma_{a,i} a c T^4}_R$ and the right
moment equations.
The expressions for the emissions source is substituted into the radiation moment equations
(Eq.~\eqref{eq:lo_tran}--~\eqref{???}) to produce a
linear system of equations for the new radiation intensity moments. 

Once the linear equations have been solved for new radiation moments, new temperature
unknowns can be estimated.  To conserve energy, the same linearization and discretizations used to
solve the radiation equation must be used in the material energy equation.
Substitution of Eq.~\eqref{eq:temp_const} into the material energy $L$ moment equation
ultimately yields
\begin{multline}\label{eq:new_temp}
    \frac{2}{3}T_{L,i}^{n+1} + \frac{1}{3}T_{R,i}^{n+1}= \frac{f_i^* \sigma_{ai}^* \Delta
t}{\rho c_{v}}  \left[ \mom{\phi^{n+1}}_{L,i}  - a c\left(\frac{2}{3} T_{L,i}^{4,n} + \frac{1}{3} T_{R,i}^{4,n}\right)
\right] + \\ (1- f^*_{i})\left(\frac{2}{3}T^*_{L,i} + \frac{1}{3}T^*_{R,i}\right) + f
\left(\frac{2}{3}T^n_{L,i} + \frac{1}{3}T^n_{R,i}\right)
\end{multline}
A similar expression is produced for the $R$ moment equation.  This produces a local
matrix equation to solve for new $T$ unknowns.  If both the radiation and temperature unknowns
are lumped, this matrix becomes diagonalized.

Based on these equations, the algorithm for solving the LO equations, with iteration index
$l$, is defined as
\begin{enumerate}
    \item Initialize $T$ unknowns using $T^n$ or the last estimate of $T^{n+1}$ from
        previous LO solve
    \item  Build the LO system based on the effective scattering $(1-f^l)$ and emission terms
          evaluated using $T^l$.
    \item Solve the linearized LO system to produce an estimate for $\phi^{n+1,l}$.
    \item Evaluate a new estimate of $T^{n+1}$ unknowns using Eq.~\eqref{eq:new_temp}.
    \item $T^*\leftarrow\tilde{T}^{n+1}$.
    \item Repeat 2-5 until $(T^{n+1,k})^4$ and $\phi^{n+1,k}$ are converged.
\end{enumerate}

\begin{figure}[H]
\centering
\includegraphics[scale=.50]{figures/Penguins.jpg}
\caption{TAMU figure}
\label{fig:tamu-fig5}
\end{figure}
