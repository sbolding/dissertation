%%%%%%%%%%%%%%%%%%%%%%%%%%%%%%%%%%%%%%%%%%%%%%%%%%%
%
%  New template code for TAMU Theses and Dissertations starting Fall 2012.  
%  For more info about this template or the 
%  TAMU LaTeX User's Group, see http://www.howdy.me/.
%
%  Author: Wendy Lynn Turner 
%	 Version 1.0 
%  Last updated 8/5/2012
%
%%%%%%%%%%%%%%%%%%%%%%%%%%%%%%%%%%%%%%%%%%%%%%%%%%%

%%%%%%%%%%%%%%%%%%%%%%%%%%%%%%%%%%%%%%%%%%%%%%%%%%%%%%%%%%%%%%%%%%%%%%
%%                           APPENDIX A 
%%%%%%%%%%%%%%%%%%%%%%%%%%%%%%%%%%%%%%%%%%%%%%%%%%%%%%%%%%%%%%%%%%%%%

\phantomsection

\chapter{\uppercase{Derivations and Equations for the LO System}}

\section{Useful Moment Relations for LO Equations}
\label{app:lo_mom_relations}

There are several relations between various moment definitions that are useful in
derivation and manipulation of the LO equations. The following are derived for $\phi(x)$,
but can be applied to general moments of functions.  The volumetric average terms
can be eliminated in terms of the $L$ and $R$ moments from the relation
$b_{L,i}(x)+b_{R,i}(x)=1$.
\begin{align}\label{app:atoL}
    \phi_{i} &= \frac{1}{h_i} \int\limits_\xl^\xr 1 \;\phi(x) \dd x \\
             &= \frac{1}{h_i} \left(\int\limits_\xl^\xr b_{L,i}(x) \phi(x) \dd x +
             \int\limits_\xl^\xr b_{R,i}(x) \phi(x) \dd x \right)\\
             &= \frac{1}{2} \left( \mom{\phi}_{L,i} + \mom{\phi}_{R,i} \right)
\end{align}
A similar relation can be derived for the first moment in space as
\begin{equation}
    {\phi}_{x,i} = \frac{3}{2} \left(\mom{\phi}_{R,i} - \mom{\phi}_{L,i}\right)
\end{equation}
The above relations can be inverted to derived a relation for the $L$ and $R$ moments in terms of the slope
and average moments.  These moment expressions are defined purely in terms of integrals,
and are independent of the chosen spatial representation

Once a linear relation on the interior has been assumed, there are other useful closures that can be
derived.  The standard linear interpolatory expansion, for the positive half-range, is restated here:
\begin{equation}
\phi^+(x) = \phi^+_{L,i} b_{L,i}(x) + \phi^+_{R,i} b_{R,i}(x)
\end{equation}
Using this expansion, one can derive a relation between the outflow from a cell and the
hat function moments that is equivalent to the standard LDFE Galerkin method:
\begin{equation}
    \phi_{i,R}^+ = 2 \mom{\phi}_{R,i}^+ - \mom{\phi}_{L,i}^+,
\end{equation}
where for standard LD $\phi_{i+1/2}^+\equiv\phi_{i,R}$.  The assumption of a linear relation on the interior
of the cell defines the value for $\phi_{i,L}^+$:
\begin{equation}
    \phi_{i,L}^+ = 2 \mom{\phi}_{L,i}^+ - \mom{\phi}_{R,i}^+,
\end{equation}

To eliminate the LO unknowns in a manner that produces the same moments as the LDFE
Galerkin method, the
following expression can be used for the outflow from a cell
\begin{equation}
    \phi_{i+1/2}^+ = \phi_i^+ + \frac{\phi_x^+}{3},
\end{equation}
which in terms of the hat function moments is equivalent to $\phi_{i+1/2}^+ =
\mom{\phi}_{R,i}^+$.  Inserting this expression into Eq.~\eqref{}, and using the same
definition for the linear representation over the interior of $\phi_{i+1/2}^+(x) =
\phi_{L,i} b_{L,i}(x) + \phi_{R,i} b_{R,i}(x)$, will produce an equivalent set of unknowns
as a linear discontinuous method with a lumped representation for the radiation.  The
temperature equation must be independently lumped. This
relation preserves the average within a cell but modifies the first moment.  

A similar expression produces a lumped-equivalent representation on the interior of the cell:
\begin{equation}
    \phi_{i,R}^+ = \phi_i^+ + \frac{\phi_x^+}{3},
\end{equation}
The moment equations are not modified by using this expression, however the interpretation
of the moments as a linear representation over the cell has been altered.  This allows for
us to ensure a lumped representation on the interior while still using the HO solution to
eliminate the outflow from the equations.


\section{Newtons Method for the LO Equations}
\label{app:lo_newton}

Because we have only considered problems with constant densities and heat capacities, the
linearization described below is in terms of temperature $T$ rather than material internal
energy, for simplicity. However, the linearization can be formed in terms of internal energy
to apply this method to a general equation of state.

To formulate the Newton iterations, the Planckian source is linearized in the material and radiation equations (Eq.~\eqref{??}
\& Eq.~\eqref{???}).
Application of the first order Taylor expansion in time to the
implicit emission source $B(T^{n+1})$, about some temperature $T^*$ at some
time $t^*\in[t^{n},t^{n+1}]$, yields
\begin{equation}\label{new_planck}
    \sigma_a^{n+1} a c T^{4,n+1} \simeq \sigma_a^* a c \left[T^{*4} + (T^{n+1} - T^*) 4T^{*3} \right]
\end{equation}
where $\sigma_a^*\equiv\sigma_a(T^*)$.  Substitution of this expression into Eq.~\eqref{eq:mat_cont} yields
\begin{equation}
    \rho c_v \left( \frac{T^{n+1} - T^{n}}{\Delta t} \right) = \sigma_a^* \phi^{n+1} -
    \sigma_a^* a c \left[ T^{*4} +  (T^{n+1} - T^*) 4T^{*3} \right].
\end{equation}
Algebraic manipulation of this equation yields an expression for $T^{n+1} - T^{*}$:
\begin{align*}
\left( T^{n+1} - T^* \right) &= \frac{ {\displaystyle \frac{\sigma_a^* \Delta t}{\rho
c_v}}  \left[ \phi^{n+1} -  a c T^{*4} \right] + (T^n - T^*) }{1 +
        \sigma_a^* a c \Delta t\frac{\displaystyle 4
T^{*3}}{\displaystyle \rho c_v } }.
\end{align*}
%This provides an expression for $T^{n+1}$ as a
%function of $T^*$ and the mean intensity $\phi^{n+1}$, i.e.,
%\begin{equation}
%\label{lo_t_new}
%T^{n+1}  = \frac{1}{\rho c_v } f\sigma_a^* \Delta t \left( \phi^{n+1} - c a T^{*4} \right)
%+ f T^n + (1-f) T^*.
%\end{equation}
This expression is substituted back into Eq.~\eqref{new_planck} to form
an explicit approximation for the emission source at $t^{n+1}$ as
\begin{equation}\label{t_next1}
    \sigma_a a c T^{4,n+1} \simeq \sigma_a^* (1 -f^*) \phi^{n+1}
    + f^* \sigma_a^* a c T^{4,n} + \rho c_v\frac{1-f^*}{\Delta t} (T^n - T^*)
\end{equation}
where $f^* = [1 + \sigma_a^* c \Delta t 4 a T^{*3}/(\rho c_v)]^{-1}$ is often referred to
as the Fleck factor~\cite{fnc}. 

Next, the above equation must be spatially discretized.  Application of the $L$ spatial
moment yields
\begin{multline}\label{eq:temp_const}
    \mom{\sigma_a^* a c T^{4,n+1}}_{L,i} = \sigma^*_{ai}(1-f_i^*)\mom{\phi^{n+1}}_{L,i} +
    f_i^*
    \sigma^*_{ai} a c \left(\frac{2}{3} T_{L,i}^{4,n} + \frac{1}{3} T_{R,i}^{4,n}\right)
    \\ \rho_i c_{vi} \frac{1 - f^*_i}{\Delta t} \left[\frac{2}{3}\left(T^n_{L,i} -
        T^*_{L,i}\right) + \frac{1}{3}\left(T^n_{R,i} -
    T^*_{R,i}\right)\right],
\end{multline}
where $T^{4,n}$ and $T^{n}$ have been assumed LD and $f^*$ is assumed constant over a cell, i.e., $f^*_i
\equiv \sigma_a(T_i^*)$.
The error introduced by a constant $f^*$ approaches zero as the
non-linearity is converged because $T^*$ approaches $T^{n+1}$. 
Based on an estimate for $T^*$, Eq.~\eqref{eq:temp_const} is an expression for
the Planckian emission source in the radiation moment equations with an additional effective scattering source.
A similar expression can be derived for $\mom{\sigma_{a,i} a c T^4}_R$ and the right
moment equations.
The expressions for the emissions source is substituted into the radiation moment equations
(Eq.~\eqref{eq:lo_tran}--~\eqref{???}) to produce a
linear system of equations for the new radiation intensity moments. 

Once the linear equations have been solved for new radiation moments, new temperature
unknowns can be estimated.  To conserve energy, the same linearization and discretizations used to
solve the radiation equation must be used in the material energy equation.
Substitution of Eq.~\eqref{eq:temp_const} into the material energy $L$ moment equation
ultimately yields
\begin{multline}\label{eq:new_temp}
    \frac{2}{3}T_{L,i}^{n+1} + \frac{1}{3}T_{R,i}^{n+1}= \frac{f_i^* \sigma_{ai}^* \Delta
t}{\rho c_{v}}  \left[ \mom{\phi^{n+1}}_{L,i}  - a c\left(\frac{2}{3} T_{L,i}^{4,n} + \frac{1}{3} T_{R,i}^{4,n}\right)
\right] + \\ (1- f^*_{i})\left(\frac{2}{3}T^*_{L,i} + \frac{1}{3}T^*_{R,i}\right) + f
\left(\frac{2}{3}T^n_{L,i} + \frac{1}{3}T^n_{R,i}\right)
\end{multline}
A similar expression is produced for the $R$ moment equation.  This produces a local
matrix equation to solve for new $T$ unknowns.  If both the radiation and temperature unknowns
are lumped, this matrix becomes diagonalized.

Based on these equations, the algorithm for solving the LO equations, with iteration index
$l$, is defined as
\begin{enumerate}
    \item Initialize $T$ unknowns using $T^n$ or the last estimate of $T^{n+1}$ from
        previous LO solve
    \item  Build the LO system based on the effective scattering $(1-f^l)$ and emission terms
          evaluated using $T^l$.
    \item Solve the linearized LO system to produce an estimate for $\phi^{n+1,l}$.
    \item Evaluate a new estimate of $T^{n+1}$ unknowns using Eq.~\eqref{eq:new_temp}.
    \item $T^*\leftarrow\tilde{T}^{n+1}$.
    \item Repeat 2-5 until $(T^{n+1,k})^4$ and $\phi^{n+1,k}$ are converged.
\end{enumerate}

\begin{figure}[H]
\centering
\includegraphics[scale=.50]{figures/Penguins.jpg}
\caption{TAMU figure}
\label{fig:tamu-fig5}
\end{figure}

\section{Derivation of the WLA-DSA Equations}
\label{sec:wla_derivation}

In this section, we derive the discretized diffusion equation and LD mapping equations
that are used in the WLA-DSA equations.  To simplify notation, we
derive the equations from a generic transport equation (rather than the error equations) with isotropic scattering
and source $q_0$, i.e.,
\begin{equation}\label{eq:ss_trans}
    \mu \pderiv{I}{x} + \sigma_t I = \frac{\sigma_s}{2}\left( \phi(x) + q_0\right).
\end{equation}

\subsection{Forming a Continuous Diffusion Equation}

First, a continuous spatial discretization of a diffusion equation is derived.  
The mean intensity $\phi$ will ultimately be assumed continuous at faces to produce a
standard three-point finite-difference diffusion discretization. 
The zeroth and first $\mu$ moment of Eq.~\eqref{eq:ss_trans} produce the $P_1$
equations~\cite{lewis,wla_thesis}, i.e., 
\begin{align}\label{eq:dsa_bal}
    \pderiv{J}{x} + \sigma_a \phi &= q_0 \\ \label{eq:p1}
    \sigma_t J + \frac{1}{3} \pderiv{\phi}{x} &= 0.
\end{align}
The spatial finite element moments (defined by Eq.~\eqref{eq:x_moml} and~\eqref{x_momr})
are taken of the above equations. 
The mean intensity is assumed linear on the interior of the cell, i.e.,
$\phi(x)=\phi_Lb_L(x) + \phi_Rb_R(x)$, for $x\in(\xl,\xr)$.   Taking the left moment,
evaluating integrals, and rearranging yields
\begin{equation}
    J_{i} - J_{\il}  + \frac{\sigma_{a,i}h_i}{2} \left(\frac{2}{3} \phi_{L,i} + \frac{1}{3}
    \phi_{R,i} \right) = \frac{h_i}{2} \mom{q}_{L,i}\,\,,
\end{equation}
where $J_i$ is the average of the flux $J$ over the cell. The moments of $q$ are
not simplified to be compatible with the error equations which are in terms of moments. For the $R$ moment
\begin{equation}
    J_{i+1/2} - J_{i}  + \frac{\sigma_{a,i}h_i}{2} \left(\frac{2}{3} \phi_{L,i} + \frac{1}{3}
    \phi_{R,i} \right) = \frac{h_i}{2} \mom{q}_{R,i}\,\,.
\end{equation}
The equation for the $L$ moment is evaluated for cell $i+1$ and added to the $R$ moment
equation evaluated at $i$.  The flux $J$ is assumed continuous at $\ir$ to eliminate
the face fluxes from the equations.  The sum of the two equations becomes
\begin{multline}\label{eq:diff_noclose}
    J_{i+1} - J_{i} + \frac{\sigma_{a,i+1} h_{i+1}}{2}\left(\frac{2}{3} \phi_{L,i+1} +
    \frac{1}{3}\phi_{R,i+1}\right) + \frac{\sigma_{a,i} h_i}{2} \left( \frac{1}{3} \phi_{L,i} +
    \frac{2}{3}\phi_{R,i}\right) =\\ \frac{h}{2} \left(\mom{q}_{L,i+1} + \mom{q}_{R,i}
    \right).
\end{multline}
The mean intensity is approximated as continuous at each face, i.e., $\phi_{L,i+1} = \phi_{R,i}
\equiv \phi_{i+1/2}$.  Adding the $L$ and $R$ moments of Eq.~\eqref{eq:p1} together, with
the continuous approximation for $\phi_{i+1/2}$, produces a discrete Fick's law equation~\cite{stacy}
\begin{equation}\label{eq:ficks}
    J_{i} = -D_i \frac{\phi_{i+1/2} - \phi_{i-1/2}}{h_i},
\end{equation}
where $D_i = 1/(3\sigma_{t,i})$.
Substitution of Eq.~\eqref{eq:ficks} into Eq.~\eqref{eq:diff_noclose} and rearranging yields the following discrete diffusion
equation:
\begin{multline}
        \left(\frac{\sigma_{a,i+1} h_{i+1}}{6} -
        \frac{D_{i+1}}{h_{i+1}}\right)\phi_{i+3/2} + \left(\frac{D_{i+1}}{h_{i+1}} +
        \frac{D_{i}}{h_i} + \frac{\sigma_{a,i+1} h_{i+1}}{3} + \frac{\sigma_{a,i}
        h_{i}}{3}\right)\phi_{i+1/2} \\ + \left(\frac{\sigma_{a,i} h_{i}}{6} -
        \frac{D_{i}}{h_{i}}\right)\phi_{i-1/2} = \frac{h_{i+1}}{2} \mom{q}_{L,i+1} +
        \frac{h_{i}}{2}\mom{q}_{R,i}\;\,. 
\end{multline}
To allow for the use of lumped
or standard LD in these equations, we introduce the factor $\theta$, with
$\theta=1/3$ for standard
LD, and $\theta=1$ for lumped LD.  The diffusion equation becomes
\begin{multline}\label{eq:dsa_lumped}
    \left(\frac{\sigma_{a,i+1} h_{i+1}}{4}\left(1 - \theta\right)  -
        \frac{D_{i+1}}{h_{i+1}}\right)\phi_{i+3/2} + \left(\frac{D_{i+1}}{h_{i+1}} +
        \frac{D_{i}}{h_i} + \left(\frac{1+\theta}{2} \right)\left[\frac{\sigma_{a,i+1} h_{i+1}}{2} + \frac{\sigma_{a,i}
        h_{i}}{2}\right]\right)\phi_{i+1/2} \\ + \left(\frac{\sigma_{a,i}
        h_{i}}{4}\left(1 - \theta\right) -
        \frac{D_{i}}{h_{i}}\right)\phi_{i-1/2} = \frac{h_{i+1}}{2} \mom{q}_{L,i+1} +
        \frac{h_{i}}{2}\mom{q}_{R,i}
        \;\,. 
\end{multline}
Summation over all cells forms a system of equations for $\phi$ at each face.  

\subsubsection{Diffusion Boundary Conditions}

The upwinding in the LO system exactly satisfies the inflow boundary conditions, therefore
a vacuum boundary condition is applied to the diffusion error equations.  The equation for the left moment
at the first cell is given by
\begin{equation}\label{eq:dsa_bc_app}
    J_{1} - J_{1/2}  + \frac{\sigma_{a,i}h_i}{2} \left(\frac{1+\theta}{2} \phi_{L,i}
    + \frac{1-\theta}{2}
    \phi_{R,i} \right) = \frac{h_i}{2} \mom{q}_{L,i}\,\,,
\end{equation}
The Marshak boundary condition for the vacuum inflow at face $x_{1/2}$ is given as
\begin{equation}
    J^+_{1/2} = 0 = \frac{\phi_{1/2}}{4} + \frac{J_{1/2}}{2},
\end{equation}
which can be solved for $J_{1/2}$.  Substitution of the above equation and
Eq.~\eqref{eq:ficks} into Eq.~\eqref{eq:dsa_bc_app} gives 
\begin{equation}\label{eq:bc_dsa}
    \left(\frac{1}{2}+ \sigma_{a,1}h_1\frac{1+\theta}{4} - \frac{D_1}{h_1}\right)\phi_{1/2} +
    \left( {\sigma_{a,1}{h_1}}\frac{1-\theta}{4} - \frac{D_1}{h_1}  \right)\phi_{3/2} =
    \frac{h_i}{2} \mom{q}_{L,1}
\end{equation}
A similar expression can be derived for the right-most cell.

\subsection{Mapping Solution onto LD Unknowns}

Solution of the continuous diffusion equation will provide an approximation to $\phi$ on
faces, denoted as $\phi_{i+1/2}^C$. We now need to map the face solution onto 
the LD representation of $\phi$. To do this, first we take the $L$ and $R$ finite element moments of the P$_1$
equations.  A LDFE dependence is assumed on the interior of the cell for $J$ and
$\phi$.  Taking moments of Eq.~\eqref{eq:dsa_bal} and simplifying yields
\begin{align}
    J_{\ir} - \frac{J_{L,i} + J_{R,i}}{2} + \frac{\sigma_{a,i} h_i}{2} \left(\frac{1}{3} \phi_{L,i} +
    \frac{2}{3}\phi_{R,i}\right) &= \frac{h_i}{2} \mom{q}_{R,i} \\
    \frac{J_{L,i} + J_{R,i}}{2} - J_{i-1/2} + \frac{\sigma_{a,i} h_i}{2}
    \left(\frac{2}{3} \phi_{L,i} +
    \frac{1}{3}\phi_{R,i}\right) &= \frac{h_i}{2} \mom{q}_{L,i}
\end{align}
The moment equations for Eq.~\eqref{eq:p1} are
\begin{align}
    \frac{1}{3}\left(\phi_{\ir} - \frac{\phi_{i,L} + \phi_{i,R}}{2}\right) +
    \frac{\sigma_{t,i} h_i}{2} \left(\frac{1}{3} J_{L,i} + \frac{2}{3}J_{R,i}\right)
    &= 0 \\
    \frac{1}{3}\left(\frac{\phi_{i,L} + \phi_{i,R}}{2} - \phi_{i-1/2} \right) +
    \frac{\sigma_{t,i} h_i}{2} \left(\frac{2}{3} J_{L,i} + \frac{1}{3}J_{R,i}\right)
    &= 0 
\end{align}

The face terms $J_{i\pm 1/2}$ and $\phi_{i\pm 1/2}$ need to be eliminated from the
system. First, the scalar intensity is assumed to be the value provided by the continuous
diffusion solution at each face, i.e., $\phi_{i\pm1/2} = \phi_{i\pm1/2}^C$.
Then, the fluxes are decomposed into half-range values to decouple the equations
between cells.  At $x_{\ir}$, the flux is composed as $J_{i+1/2} = J_{\ir}^+ + J_{\ir}^-$,
noting that in this notation the half-range fluxes are $J_{\ir}^{\pm}=\pm \int_{0}^\pm
\mu I(x_{i+1/2},\mu) \dd \mu$\footnote{Typically, the half-range fluxes are defined with
    integrals weighted with $| \mu |$, but this notation would not be consistent with our
definition of the half-range consistency terms}.  We approximate the incoming fluxes, e.g.,
$J_{i+1/2}^-$, based on $\phi_{i+1/2}^C$ and a P$_1$ approximation.   
The P$_1$ approximation provides the following relation~\cite{wla_thesis}
\begin{equation}
    \phi = 2(J^+ - J^-).
\end{equation}
At $\xr$, the above expression is solved for the incoming current $J_{i+1/2}^-$.  The
total current becomes
\begin{equation}\label{eq:jelim}
    J_{\ir} = J_{\ir}^+ - J_{\ir}^- = 2J_{\ir}^+ - \frac{\phi_{i+1/2}^C}{2},
\end{equation}
In the positive direction, at the right face, the
values of $\phi$ and $J$ are based on the LD representation within the cell at that
face, i.e., $\phi_{R,i}$ and $J_{R,i}$.  The standard P$_1$ approximation for the
half-range fluxes is used\cite{stacy}, i.e.,
\begin{align}
    J^{\pm} &= \frac{\gamma \phi}{2} \pm \frac{J}{2},
\end{align}
where $\gamma$ accounts for the difference between the LO parameters and the true
P$_1$ approximation. Thus, for the right face and positive half-range,
\begin{align}
    J_{\ir}^+ &= \frac{\gamma}{2}\phi_{i,R} + \frac{J_{i,R}}{2} 
\end{align}
A similar expression can be derived for $\xl$.  The total fluxes at each face are
thus
\begin{align}
    J_{i+1/2} &= \gamma\phi_{i,R} + J_{i,R} - \frac{\phi_{\ir}^C}{2} \\
    J_{i-1/2} &= \frac{\phi_{i-1/2}^C}{2} - \gamma \phi_{i,L} + J_{i,L}
\end{align}
Substitution of these results back into the LD balance equations and introduction of the
lumping notation yields the final equations 
\begin{align}\label{eq:update1}
    \left(\gamma\phi_{i,R} + J_{i,R} - \frac{\phi_{\ir}^C}{2} \right) - \frac{J_{L,i} + J_{R,i}}{2} + \frac{\sigma_{a,i} h_i}{2} \left(
    \frac{(1-\theta)}{2} \phi_{L,i} +
    \frac{(1+\theta)}{2}\phi_{R,i}\right) &= \frac{h_i}{2} \mom{q}_{R,i} \\
    \frac{J_{L,i} + J_{R,i}}{2} -\left(\frac{\phi_{i-1/2}^C}{2} - \gamma \phi_{i,L} +
    J_{i,L}\right) + \frac{\sigma_{a,i} h_i}{2} \left(
    \frac{(1+\theta)}{2} \phi_{L,i} +
    \frac{(1-\theta)}{2}\phi_{R,i}\right) &= \frac{h_i}{2} \mom{q}_{L,i} 
    \\
    \frac{1}{3}\left(\phi_{i+1/2}^C - \frac{\phi_{i,L} + \phi_{i,R}}{2}\right) +
    \frac{\sigma_{t,i} h_i}{2}\left( \frac{(1-\theta)}{2} J_{L,i} +
    \frac{(1+\theta)}{2}J_{R,i}\right)    &= 0 \\
    \frac{1}{3}\left(\frac{\phi_{i,L} + \phi_{i,R}}{2} - \phi^C_{i-1/2} \right) +
    \frac{\sigma_{t,i} h_i}{2} \left( \frac{(1+\theta)}{2} J_{L,i} +
    \frac{(1-\theta)}{2}J_{R,i}\right) &= 0 . \label{eq:update2}
\end{align}
The above equations are completely local to each cell and fully defined, including for
boundary cells. For simplicity, we just take $\gamma=1/2$.  The system can be solved for the desired unknowns
$\phi_{i,L}$, $\phi_{i,R}$, $J_{i,L}$, and $J_{i,R}$, which represent the mapping of
$\phi_{i+1/2}^C$ onto the LD representation for $\phi^{\pm}(x)$.
