
\chapter{\uppercase {Overview of the HOLO Algorithm}}
\label{chp:holo}

For simplicity, our HOLO method will use a backward Euler (BE) discretization in time, as
well as constant specific heats and cell-wise constant cross sections. The time-discretized
equations are
\begin{align}
    \mu \pderiv{I^{n+1}}{x} + \left(\sigma_t^{n+1} + \frac{1}{c \Delta t }\right) I^{n+1}
&= \frac{\sigma_s}{2} \phi^{n+1} +\frac{1}{2} \left(\sigma_a a c T^4 \right)^{n+1} + \frac{I^n}{c \Delta t} \label{eq:trans_td} \\
\rho c_v \frac{T^{n+1} - T^n}{\Delta t} &= \sigma_a^{n+1} \phi^{n+1}
- \sigma_a a c (T^4)^{n+1} \label{eq:mat_td},
\end{align}
where $\Delta t$ is the uniform time step size, the superscript $n$ is used to indicate
the $n$-th time step. Cross sections are evaluated at the end of time step
temperature, i.e., $\sigma_a^{n+1}\equiv\sigma_a(T^{n+1})$. It is noted that in IMC the time derivative in
Eq.~\eqref{ho_cont} is typically treated continuously using time-dependent MC over each
time step.  Our HO transport equation is
discrete in time for simpler application of ECMC and to avoid difficulties in coupling to the
fully-discrete LO solver.  However, this does introduce some artificial propagation of
energy due to the implicit time differencing in optically thin regions. 

The directionality of the S$_2$ equations is convienent for closing the equations
spatially with a discontinuous trial space.

In the HOLO context, the LO solver models isotropic scattering and
resolves the material temperature spatial distribution $T(x)$ at each time step.  The LO equations are formed via half-range 
angular and spatial moments of
Eq.~\eqref{eq:trans_td} and Eq.~\eqref{eq:mat_td}. The spatial moments are formed over a
finite-element mesh and linear-discontinuous representation of the temperature is used to
close the system.  The angular treatment in the LO equations has the same form as those
used in the hybrid-S$_2$ method in~\cite{wolters},  with consistency parameters that
represent angularly-weighted averages of the intensity.  These consistency parameters
area analogous to a variable Eddington factor~\cite{chandrasekhar}.  If the angular consistency parameters were exact, then the LO
equations are exact, neglecting spatial discretization errors.  These consistency
parameters are lagged in each LO solve, estimated from the previous HO solution for
$I^{n+1}(x,\mu)$, as explained below. For the initial LO solve for each time step, the
parameters are calculated with $I^{n}(x,\mu)$.  The discrete LO equations always conserve
total energy, independent of the accuracy of the consistency terms.  The LO system uses a
LDFE spatial discretization for the temperature and half-range scalar intensities.  The
LDFE spatial discretization should correctly preserve the equilibrium diffusion limit, a
critical aspect for TRT equations~\cite{larsen_edl,morel_newton}.  Additionally, the
implicit time discretization with sufficient convergence of the nonlinear emission source
will ensure that the method will not exhibit maximum principle
violations~\cite{larsen_mpv}.

We expect our method, with a fully implicit time discretization,
to preserve the MP with sufficient convergence of the nonlinear emission
source~\cite{larsen_mpv}.


If the angular consistency parameters were estimated exactly, then
the LO equations preserve the exact angular-averaged solution,  neglecting spatial
discretization errors.  These consistency parameters are lagged in each LO solve,
estimated from the previous HO solution for the intensity, or from a previous time
step.  The discrete LO equations always conserve total energy, independent of the accuracy of the consistency terms.
It is noted that our LO operator is different from the nonlinear
diffusion acceleration (NDA) methods used by other HOLO methods~\cite{rmc,park,willert}.  In
NDA methods, an artificial term is added to the LO equations to enforce consistency and estimated using a
previous HO solution.  In our method we have simply algebraically 
manipulated space-angle moment equations to produce our consistency terms,
which will hopefully produce more
stability in optically-thick regions where NDA methods demonstrate stability issues.

It is noted that the LD projection of the HO solution does not produce the same moments, because
MC was used to obtain this projection, the outflow will not agree with the LO equations.
For example, the upwinding inflow from a previous cell does not match the actual energy
that flowed through that surface due to MC noise.

The solution to the LO system is used to construct a LDFE spatial representation of
the isotropic scattering and emission sources on the right hand side of
Eq.~\eqref{eq:trans_td}.   The LDFE representation of the emission source mitigates teleportation error.
 This defines a fixed-source, pure absorber
transport problem for the HO operator. This HO transport problem represents a characteristic method that uses MC to
invert the continuous streaming plus removal operator with an LDFE representation of
sources; the representation of sources is similar to the linear moments method
discussed in~\cite{larsen_error}.  We will solve this transport problem using ECMC.  The output from ECMC is
$\tilde{I}^{n+1}(x,\mu)$, a space-angle LDFE projection of the exact solution
$I^{n+1}(x,\mu)$ to the described transport problem.  Once computed, $\tilde{I}^{n+1}(x,\mu)$ is used
to directly evaluate the necessary consistency parameters for the next LO solve.  Since there is a global, functional representation of
the angular intensity,  LO parameters are estimated using quadrature and do not
require additional tallies.  The HO solution is not used to directly estimate a new
temperature at the end of the time step; it is
only used to estimate the angular consistency parameters for the LO equations, which eliminates
typical operator splitting stability issues that require linearization of the emission source.

The process of performing subsequential HO and LO solves, within a single time step, can be repeated to obtain an increasingly accurate solution for $\phi^{n+1}(x)$ and $T^{n+1}(x)$.  Thus, the HOLO algorithm, for the $n$-th time step, is
\begin{enumerate}
\item Perform a LO solve to produce an initial guess for $T^{n+1,0}(x)$
    and $\phi^{n+1,0}(x)$, based on consistency terms estimated with $\tilde{I}^{n}$.
\item Solve the HO system for $\tilde{I}^{n+1,k+1/2}(x,\mu)$ with ECMC, based on the current
    LO estimate of the emission and scattering sources.%$\sigma_s(T^k)\phi^{k}$ and $B(T)^{k}$.
\item Compute LO consistency parameters with $\tilde{I}^{n+1,k+1/2}$.  
\item Solve the LO system with HO consistency parameters to produce a new
    estimate of $\phi^{n+1,k+1}$ and $T^{n+1,k+1}$.
\item Optionally repeat 2 -- 4 until desired convergence is achieved.
\item Store $\tilde{I}^{n}\leftarrow\tilde{I}^{n+1}$, and move to the next time step.
\end{enumerate}
where the superscript $k$ denotes the outer HOLO iteration.
The consistency terms force the HO
and LO solutions for $\phi^{n+1}(x)$ to be consistent to the order of the current HOLO
iteration error, as long as the LDFE spatial representation can accurately represent
$\phi(x)$ and $T(x)$.
%One HOLO fixed-point iteration $k$ denotes the process of an ECMC solve of the HO problem to estimate LO parameters, based on
%the current LO estimate of sources, followed by a solution of the 
%LO system for $T^{n+1}(x)$ and $\phi^{n+1}(x)$.

