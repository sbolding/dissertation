
\chapter{\uppercase {Overview of the HOLO Algorithm}}
\label{chp:holo}

The HOLO algorithm is a nonlinear acceleration method.  Iterations are performed between a
high-order (HO) transport equation and a low-order (LO) system formulated with angular moments
and a fixed spatial mesh. 
With the exception of Chapter~\ref{sec:time}, we will derive and discuss our HOLO
method using a backward Euler (BE) time discretization for simplicity.  We have also assumed constant specific heats
and cell-wise constant cross sections, although our method could easily be extended
to a general material equation of state. The BE time-discretized
equations are
\begin{align}
    \mu \pderiv{I^{n+1}}{x} + \left(\sigma_t^{n+1} + \frac{1}{c \Delta t }\right) I^{n+1}
&= \frac{\sigma_s}{2} \phi^{n+1} +\frac{1}{2} \left(\sigma_a a c T^4 \right)^{n+1} + \frac{I^n}{c \Delta t} \label{eq:trans_td} \\
\rho c_v \frac{T^{n+1} - T^n}{\Delta t} &= \sigma_a^{n+1} \phi^{n+1}
- \sigma_a a c (T^4)^{n+1} \label{eq:mat_td},
\end{align}
where $\Delta t$ is the time step size, and the superscript $n$ is used to indicate
the $n$-th time step. Cross sections are evaluated implicitly, i.e., at the end of time step
temperature ($\sigma_a^{n+1}\equiv\sigma_a(T^{n+1})$). It is noted that in IMC the time derivative in
Eq.~\eqref{eq:rad_cont} is typically treated continuously using time-dependent MC over each
time step, but with the temperature implicitly discretized.  Our HO transport equation is
discrete in time for simpler application of ECMC and to avoid difficulties in coupling to the
fully-discrete LO solver.  In chapter~\ref{sec:time}, time-dependent transport is
included in the ECMC algorithm with consistent LO equations, similar to the IMC
treatment of the time variable, improving accuracy in optically thin
regions.

In the HOLO context, the LO solver models isotropic scattering and
resolves the material temperature spatial distribution $T(x)$ at each time step.  The LO equations are formed via half-range 
angular and spatial moments of
Eq.~\eqref{eq:trans_td} and Eq.~\eqref{eq:mat_td}. 
The spatial moments are formed over a
finite-element mesh using linear finite-element (FE) basis functions.   The angular treatment in the LO equations has the same form as those
used in the hybrid-S$_2$ method in~\cite{wolters},  with consistency parameters that
represent angularly-weighted averages of the intensity. 
These consistency parameters
area analogous to a variable Eddington factor~\cite{chandrasekhar}.  If the angular consistency parameters were exact, then the LO
equations are exact, neglecting spatial discretization errors.  This provides the
potential for the LO equations to correctly reproduce the associated space-angle
moments of the HO solution. These consistency
parameters are lagged in each LO solve, estimated from the previous HO solution for
$I^{n+1}(x,\mu)$, as explained below. For the initial LO solve for each time step, the
parameters are calculated with $I^{n}(x,\mu)$ from the previous HO solve.  The discrete LO equations always conserve
total energy, independent of the accuracy of the consistency terms. 
Additionally, the
implicit time discretization, with sufficient convergence of the nonlinear emission
source, will ensure that the method will not exhibit maximum principle
violations~\cite{larsen_mpv}.

Our LO operator is different from the nonlinear
diffusion acceleration (NDA) methods used by other HOLO methods~\cite{rmc,park,willert}.  In
NDA methods, an artificial term is added to the LO equations to enforce consistency and estimated using a
previous HO solution.  In our method we simply algebraically 
manipulated space-angle moment equations to produce our consistency terms, and then
introduce a spatial closure. This should produce more
stability in optically-thick regions where NDA methods demonstrate stability issues,
although a formal stability analysis has not been performed.

The directionality of the half-range integrals are convenient for closing the equations
spatially with a discontinuous trial space.  Spatially, the radiation moment
equations are closed using either a linear-discontinuous (LD) closure or with a
parametric relation derived from the HO solution.  A linear-discontinuous
finite-element (LDFE) representation of $T(x)$
and $T^4(x)$ are used to eliminate the remaining spatial unknowns.
The LDFE spatial discretization correctly preserves the equilibrium diffusion limit, a
critical aspect for TRT equations~\cite{larsen_edl,morel_newton}. Also, the LDFE
representation of the emission source mitigates artificial propagation of radiation
energy across a spatial cell.

The solution to the LO system is used to construct a LDFE spatial representation of
the isotropic scattering and emission sources on the right hand side of
Eq.~\eqref{eq:trans_td}.    This defines a fixed-source, pure absorber
transport problem for the HO operator. This HO transport problem represents a characteristic method that uses MC to
invert the continuous streaming plus removal operator with an LDFE representation of
sources; the representation of sources is similar to the linear moments method
discussed in~\cite{larsen_error}.  We will solve this transport problem, which has
the form of a steady-state transport equation, using the ECMC method.  The output from ECMC is
$\tilde{I}^{n+1}(x,\mu)$, a space-angle LDFE projection of the exact solution
$I^{n+1}(x,\mu)$ to the described transport problem.  Once computed, $\tilde{I}^{n+1}(x,\mu)$ is used
to directly evaluate the necessary consistency parameters for the next LO solve.  Since there is a global, functional representation of
the angular intensity,  LO parameters are estimated using quadrature and do not
require additional tallies.  The HO solution is not used to directly estimate a new
temperature at the end of the time step; it is
only used to estimate the angular consistency parameters for the LO equations, which eliminates
typical operator splitting stability issues that require linearization of the emission source.

One HOLO fixed-point iteration $k$ denotes the process of an ECMC solve of the HO problem to estimate LO parameters, based on
the current LO estimate of sources, followed by a solution of the 
LO system for $T^{n+1}(x)$ and $\phi^{n+1}(x)$.
The process of performing subsequential HO and LO solves, within a single time step,
can be repeated to obtain an increasingly accurate solutions if the HO solution has
sufficiently low statistical noise.  Thus, the HOLO algorithm, for the $n$-th time step, is
\begin{enumerate}
\item Perform a LO solve to produce an initial guess for $T^{n+1,0}(x)$
    and $\phi^{n+1,0}(x)$, based on consistency terms estimated with $\tilde{I}^{n}$.
\item Solve the HO system for $\tilde{I}^{n+1,k+1/2}(x,\mu)$ with ECMC, based on the current
    LO estimate of the emission and scattering sources.%$\sigma_s(T^k)\phi^{k}$ and $B(T)^{k}$.
\item Compute LO consistency parameters with $\tilde{I}^{n+1,k+1/2}$.  
\item Solve the LO system with HO consistency parameters to produce a new
    estimate of $\phi^{n+1,k+1}$ and $T^{n+1,k+1}$.
\item Optionally repeat 2 -- 4 until desired convergence is achieved.
\item Store $\tilde{I}^{n}\leftarrow\tilde{I}^{n+1}$, and move to the next time step.
\end{enumerate}
where the superscript $k$ denotes the outer HOLO iteration\footnote{Throughout this
    dissertation, the outer HOLO iteration index $k$ is often suppressed for visual clarity.
    Where necessary, subscript ``$HO$'' and ``$LO$'' are also used to indicate terms from
respective solvers explicitly.}
The consistency terms force the HO
and LO solutions for $\phi^{n+1}(x)$ to be consistent to the order of the current HOLO
iteration error, as long as the LDFE spatial representation can accurately represent
$\phi(x)$ and $T(x)$.

