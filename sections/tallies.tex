%Commands for this section
\newcommand{\dep}{\ensuremath{\delta\epsilon^{(m)}}}

\subsection{MC solution with LDD trial space}
\label{sec:ldd_mc}

The inclusion of the outflow discontinuity has a minimal effect on the treatment of the
residual source. The residual source and process of estimating moments of
the error on the interior of a space-angle cell is unchanged.  The process of estimating
the solution on the outgoing face requires tallying the solution when particles leave a
cell. The specific
tallies are described in Section~\ref{sec:face_tallies}.  The introduction of the 
discontinuity results in two $\delta$ function face sources.  Ultimately
face source of the
residual, there is no need to change the residual source on faces either, however care
must be taken in how the error is added to the solution on a frace.  

To demonstrate that the residual face source is unchanged, it is necessary to look at both 
$\delta$-function face sources, which results from the discontinuity in the trial space.  For
positive flow, at the node $x_{i+1/2}$, the face source is defined as
\begin{equation}
    \label{eq:res_face}
    \rface = -\mu \pderiv{I^{(m)}}{x}\big|_{x_{i+1/2}} =-\mu\left( I^{(m)}(x_{i+1/2}^+,\mu) -I^{(m)}(x_{i+1/2}^-,\mu) \right)
\end{equation}
which is equivalent to the initial residual source. 

The only difference from the original residual
is that particles born from the $\delta^-$ source will contribute to the face tally at $x_{i+1/2}$.  
Rather than sampling this source and contributing to these tallies, we can derive
the analytic solution at that point resulting from the face source and add it to the MC
solution without the contributions to the face from that tally.  Define the additional error
from the face source as $\delta \epsilon^{(m)}$.  The transport equation satisfied by $\delta
\epsilon^{(m)}$ with effective total cross 
section $\hat \sigma_t$ is
\begin{equation}
    \label{eq:ho_face}
    \mu \pderiv{\delta\epsilon^{(m)}}{x} + \hat\sigma_t \epsilon^{n+1} = 
\end{equation}
This equation is integrated from $x_{i+1/2}-\epsilon$ to $x_{i+1/2}$
\begin{equation}
    \dep(x_{i+1/2},\mu) - \dep(x_{i+1/2}-\epsilon,\mu)  + \int_{x_{i+1/2}-\epsilon}^0 
      \hat \sigma_t \dep = 
\end{equation}
taking the limit as $\epsilon\ra 0$



\subsection{Face Tallies and correction near $\mu=0$}
\label{sec:face_tallies}

Face-averaged estimators of the angular error are required to compute the outflow for
estimating the spatial closure. The standard face-based
tallies~\cite{shultis_mc,favorite_faces} are used.  Tallies are weighted by
the appropriate basis functions to compute a linear projection along the face.  The
tally score, for the angular-averaged error $\epsilon_{a,i}$ is defined as
\begin{equation}
    \hat \epsilon_{a,i\pm1/2,j} = \frac{1}{N} \sum_{m=1}^{N_{i+1/2,j}}
    \frac{w_m(x_{i\pm1/2},\mu)}{h_{\mu} |\mu|}
\end{equation}
where $N$ is the number of histories performed and $N_{i+1/2,j}$ is the number of histories
that crossed the surface $i+1/2$, in the $j$ angular element.   For the first
moment, the tally is
\begin{equation}
    \hat \epsilon_{\mu,i\pm1/2,j} = \frac{1}{N} \sum_{m=1}^{N_{i+1/2,j}} 
    6\left(\frac{\mu-\mu_j}{h_\mu}\right) \frac{w_m(x_{i\pm1/2},\mu)}{|\mu| h_{\mu}}
\end{equation}
For positive and negative direction outflows are tallied
on the $x_{i+1/2}$ and $x_{i-1/2}$ faces, respectively. Particles are only tallied after leaving
a cell, and, as discussed in Section~\ref{sec:ldd_mc}, particles born on a surface do not contribute
to the tally of that surface.

Near $\mu=0$, particles can contribute large scores which can lead to large and
unbounded variances~\cite{favorite_faces}.  To avoid large variances, we have applied the standard fixup~\cite{mcnp,favorite_faces}.  
For $|\mu|$ below some small value $\mu_{cut}$, then 
particles contribute the average score over the range $(0,\mu_{cut})$, based on an
approximate isotropic intensity.  Assuming an isotropic intensity, the average of
$1/|\mu|$ in s given by
\begin{equation}
    \mom{1/\mu} = \frac{\int_0^{\mu_{cut}}\frac{1}{\mu} I \d \mu}{\int_0^{\mu_{cut}} I \d \mu} =
    \frac{2}{\mu_{cut}}.
\end{equation}
All particles in the range $(0,\mu_{cut})$ contribute the expected score, thus the tally
equations are evaluated with $\mu = 2/\mu_cut$.  It is noted that the first moment of $\mu$
could be estimated without this correction, but it would be inconsistent with the zeroth
moment.  However, this helps to smooth the solution near $\mu=0$, which the LD trial space
generally cannot resolve anyways, resulting in a floored zeroth moment.


STUFF ABOUT LDFE NOW FROM NSE ARTICLE



\section{Implementation of ECMC finite-element space, tallies, and residual sampling}

\label{app:tallies}

The ECMC solver uses a finite element representation in space and angle. On the
interior of the cell with the $i$-th spatial index and $j$-th angular index, the linear representation is defined as
\begin{equation*}
    \tilde I(x,\mu) = I_{a,ij} + \frac{2}{h_x}I_{x,ij}\left(x-x_i\right) +
    \frac{2}{h_\mu}I_{\mu,ij}\left(\mu-\mu_j\right), \quad x_\il <  x < x_\ir,\quad
     \mu_\jl \leq \mu \leq \mu_\jr
\end{equation*}
The spatial cell width is $h_x$, the angular width is
$h_\mu$, the center of the cell is $(x_i,\mu_j)$, and
\begin{align}\label{app1}
    I_{a,ij} &= \frac{1}{h_x h_\mu} \iint\limits_{\mathcal{D}} I(x,\mu)\, \d x \d \mu \\
    I_{x,ij} &= \frac{6}{h_xh_\mu}\iint\limits_{\mathcal{D}} \left(\frac{x - x_i}{h_{x}}\right)
    I(x,\mu)\, \d x \d \mu \\ \label{app2}
    I_{\mu,ij} &= \frac{6}{h_xh_\mu}\iint\limits_{\mathcal{D}}
     \left(\frac{\mu - \mu_j}{h_{\mu}}\right)
    I(x,\mu)\, \d x \d \mu,
\end{align}
where $\mathcal{D}: x_\il \leq  x \leq  x_\ir \times \mu_\jl \leq \mu \leq \mu_\jr$.
%$I_a$ is the cell average intensity, and $I_\mu$ and $I_x$ define the
%the first moment in $\mu$ and $x$ of the intensity, respectively. 
Standard upwinding in space is used to
define $I(\mu)$ on incoming faces. 

This representation can directly be plugged into
Eq.~\eqref{eq:resid} and evaluated to produce the residual source in the ECMC HO transport
problem.  The MC source $r^{(m)}(x,\mu)$ in Eq.~\eqref{eq:mc_err}
consists of both face and volumetric sources and can produce positive and
negative weight particles.  The distribution for sampling particle coordinates, in space and angle, is based on the $L_1$
norm over space and angle of the residual~\cite{jake}.  A particular cell volume or face 
is sampled, and then rejection sampling~\cite{shultis_mc} is used to sample from
the appropriate distribution on the face or interior of the space-angle cell.  If the
residual is negative at the sampled coordinates, the weight of the particle history is negative.

During a MC batch, moments of the error are tallied.  The necessary moments of the error are
defined analogously to Eq.'s~\eqref{app1}--\eqref{app2}.
The tallies are evaluated by weighting the particle density with the appropriate
basis function and integrating along the history path through the cell.  For the cell average, the $n$-th
particle makes the contribution
\begin{equation}
    \epsilon^n_{a,ij} = \frac{1}{h_xh_\mu} \int\limits_{s^n_o}^{s^n_f}  w^n(x,\mu) \d s,
\end{equation}
where $s_o^n$ and $s_f^n$ are the beginning and end of the $n$-th particle track in the cell and $w(x,\mu)$ is
the weight of the error particle in the MC simulation.  Weight is attenuated exponentially, i.e., $w(x,\mu)\propto
\exp(-\sigma_t|x/\mu|)$.
Substitution of the exponential attenuation of the weight produces the result
\begin{equation}
    \epsilon^n_{a,ij} = \frac{w(x_0,\mu)}{\sigma_t h_x h_\mu} \left(1 -
    e^{-\sigma_ts^n}\right).
\end{equation}
Here, $w(x_0,\mu)$ is the particle weight at the start of the path and $s^n$ is the
length of the track. The contribution of a
particle track to $\epsilon_x$ is given by
\begin{equation}
    \epsilon^n_{x,ij} = \frac{w(x_0,\mu)}{h_x^2h_\mu \sigma_t} \left[x_0 - x_f e^{-\sigma_t s^n}
        + \left(\frac{\mu}{\sigma_t} - x_i \right)\left(1-e^{-\sigma_t s^n}\right),
    \right]
\end{equation}
where $x_0$ and $x_f$ are the beginning and ending $x$ coordinates of the $n$-th
path.  The contribution to the first moment in $\mu$ is 
\begin{equation}
    \epsilon^n_{\mu,ij} = \frac{w(x_0,\mu)}{h_{\mu}^2h_x\sigma_t}\left(\mu -
    \mu_j\right) \left(1 - e^{-\sigma_ts^n}\right),
\end{equation}
where the particle $x$-direction cosine $\mu$ does not change because it is a pure-absorber simulation.
Finally, the moments of the error are simply the average contribution of all particles.

\section{Adaptive Mesh Refinement}
\label{app:refinement}
This section describes the adaptive refinement strategy for the ECMC algorithm.
Detailed equations for performing projections between meshes and computing the residual source on
the refined meshes can be found in~\cite{jake}.  At the end of the ECMC batch,
refinement is performed in space-angle cells based on a jump indicator.  The jump
indicator is the magnitude of the different between $I(x,\mu)$ in adjacent cells,
averaged over each edge.  The value of the largest jump, out of the four edges within a
cell, is used as the
indicator for that cell.  Based on this indicator, the 20\% of cells with the largest jump are
refined.  Future work will explore simply using $\epsilon$ to indicate refinement,
rather than the jump error.  The refinement of a cell is chosen to be symmetric, with each space-angle cell divided into four
equal-sized cells.  The solution for $\tilde{I}^{n+1}(x,\mu)$ of the batch is projected onto
the finer mesh for the next batch. Because the dimensionality of the sample space has
increased, we increase the number of histories per batch s.t. the ratio of the number
of histories to total cells is approximately constant for all meshes.  At the end of the last HO solve in a time step,
$\tilde{I}^{n+1}$ is projected back onto the original, coarsest mesh and stored as
$\tilde{I}^{n}$ for the next time step.
