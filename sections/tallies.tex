
\section{Face Tallies and correction near $\mu=0$}
\label{sec:face_tallies}

Face-averaged estimators of the angular error are required to compute the outflow for
estimating the spatial closure. The standard face-based
tallies~\cite{shultis_mc,favorite_faces} are used.  Tallies are weighted by
the appropriate basis functions to compute a linear FE projection in $\mu$ at each face.  The
tally score, for the angular-averaged error $\epsilon_{a,i}$ is defined as
\begin{equation}
    \hat \epsilon_{a,i\pm1/2,j} = \frac{1}{N} \sum_{m=1}^{N_{i\pm1/2,j}}
    \frac{w_m(x_{i\pm1/2},\mu)}{h_{\mu} |\mu|},
\end{equation}
where $N$ is the number of histories performed and $N_{i\pm1/2,j}$ is the number of histories
that crossed the surface $i\pm1/2$, in the $j$ angular element.   For the first
moment, the tally is
\begin{equation}\label{eq:face_mutally}
    \hat \epsilon_{\mu,i\pm1/2,j} = \frac{1}{N} \sum_{m=1}^{N_{i+1/2,j}} 
    6\left(\frac{\mu-\mu_j}{h_\mu}\right) \frac{w_m(x_{i\pm1/2},\mu)}{|\mu| h_{\mu}}.
\end{equation}
For positive and negative direction outflows are tallied
on the $x_{i+1/2}$ and $x_{i-1/2}$ faces, respectively. Particles are only tallied after leaving
a cell, and, as discussed in Section~\ref{sec:ldd_mc}, particles born on a surface do not contribute
to the tally of that surface.

Near $\mu=0$, particles can contribute large scores to the zeroth angular moment that lead to large and
unbounded variances~\cite{favorite_faces}.  To avoid large variances, we have applied the standard fixup~\cite{mcnp,favorite_faces}.  
For $|\mu|$ below some small value $\mu_{cut}$, 
particles contribute the expected score over the range $(0,|\mu_{cut}|)$, based on an
approximate, isotropic particle density. Thus, scores in this range have no variance.  Assuming
an isotropic particle density $I_0$, the average of
$1/\mu$, for positive $\mu$, is
\begin{equation}
    \overline{1/\mu} = \frac{\displaystyle \int_0^{\mu_{cut}}\frac{1}{\mu} I_0 \,\dd
\mu}{\displaystyle \int_0^{\mu_{cut}} I_0\, \dd \mu} =
    \frac{2}{\mu_{cut}}.
\end{equation}
For negative $\mu$, $\overline{1/\mu}=-2/\mu_{cut}$.
All particles in the range $(0,|\mu_{cut}|)$ contribute the expected score by evaluating
the tallies at $\pm\mu = \pm2/\mu_{cut}$.  It is noted that the first angular moment tallies are
well defined because there is no $\mu$ term in the tally. THIS ISNT REALLY TRUE NOW
BECAUSE THE FINITE ELEMENT FIRST MOMENT IS FINE. Additionally, assuming an isotropic intensity over the range helps to limit
the first moment near $\mu=0$, which the LD trial space
generally cannot resolve anyways, as discussed in ???.
