
\subsection{MC solution with LDD trial space}

The inclusion of the outflow discontinuity has a minimal effect on the treatment of the
residual source. The residual source and process of estimating moments of
the error on the interior of a space-angle cell is unchanged.  The process of estimating
the solution on the outgoing face requires tallying the solution when particles leave a
cell. The specific
tallies are described in Section~\ref{sec:face_tallies}.  As far as face source of the
residual, there is no need to change the residual source on faces either, however care
must be taken in how the error is added to the solution on a frace.  

To demonstrate that the residual face source is unchanged, it is necessary to look at the
$\delta$-function face source, which results from the discontinuity in the trial space.  For
positive flow, at the node $x_{i+1/2}$, the face source is defined as
\begin{equation}
    \rface = -\mu \pderiv{I^{(m)}}{x}\big|_{x_{i+1/2}}  
\end{equation}

PROOF THAT FACE SOURCES CANCEL OUT IN EFFECT AND THAT SOURCES CANCEL OUT




\subsection{Face Tallies and correction near $\mu=0$}
\label{sec:face_tallies}

Face-averaged estimators of the angular error are required to compute the outflow for
estimating the spatial closure. The standard face-based
estimators~\cite{shultis_mc,favorite_faces} are used.  The tallies are weighted by
the appropriate basis functions to compute a linear projection along the face.  The
tally score, for the angular-average error on a face, is defined as
\begin{equation}
    \hat \epsilon_{a,i\pm1/2,j} = \frac{1}{N} \sum_{m=1}^{N_{i+1/2,j}}
\frac{w_m(x_{i\pm1/2},\mu)}{h |\mu|}
\end{equation}
where $N$ is the number of histories performed and $N_{i+1/2,j}$ is the number of histories
that crossed the surface of interest, in the appropriate angular bin. For the first
moment, the tally is
\begin{equation}
    \hat \epsilon_{\mu,i\pm1/2,j} = \frac{1}{N} \sum_{m=1}^{N_{i+1/2,j}} 
        6\frac{\mu-\mu_j}{h_\mu} \frac{w_m(x_{i\pm1/2},\mu)}{|\mu|}
\end{equation}

Near $\mu=0$, particles can contribute large scores which can lead to large and
unbounded variance~\cite{favorite_faces}.  To avoid this, the standard fixup 
is used~\cite{mcnp,favorite_faces}.  If $|\mu|$ is below some small cutoff $\mu_{cut}$, then 
particles contribute the average score over the range $(0,\mu_{cut})$, based on an
approximate isotropic intensity.  Assuming an isotropic intensity, the average of
1/$|mu|$ is given by
\begin{equation}
\end{equation}

