\section{Analytic Neutronics answer for Source fixup}

In this section we model a fixed-source, pure-absorber neutronics calculation where we know the
analytic answer to test our fixup.  If we make the mesh thick enough, we can set the
solution to be the equilibrium answer $\psi(x) = \frac{q(x)}{2\sigma_a}$. For a general
isotropic source $Q(x)$, the 1D transport equation to be solved is
\begin{equation}
    \mu \pderiv{\psi}{x} + \sigma_a \psi(x,\mu) = \frac{q(x)}{2}
\end{equation}
with boundary condition $\psi(0,\mu)=\psi_{inc}$, $\mu>0$ and
$\psi(x_R,\mu)=\frac{q(x_R)}{2\sigma_a}$ for
$\mu<0$, where $x_R$ is the right boundary.  
This first order differential equation is solved using an integration factor.
The solution to this equation for $\mu>0$ is given by
\begin{equation}
    \psi(x,\mu) = \psi_{inc}e^{\frac{-\sigma_a x}{\mu}} + \int_0^x \frac{q(x')}{2\mu}
    e^{\frac{-\sigma_a x'}{\mu}} \d x',\quad \mu>0
\end{equation}
Integration of this result over the positive half range of $\mu$ gives
\begin{equation}
    \phi^+(x) = \psi_{inc}\E2(\sigma_a x) + \frac{1}{2}\int_0^x q(x')\E1(\sigma_a x')
    \d x'.
\end{equation}
In the simplification of a constant source, the integral reduces to
\begin{equation}
    \phi^+(x) = \psi_{inc}\E2(\sigma_a x) + \frac{q}{2\sigma_a} \left(1 -
    \E2(\sigma_ax)\right).
\end{equation}
Also, for a constant source the solution for the negative half range becomes a constant, i.e.,
\begin{equation}
    \phi^{-}(x) = \frac{q}{\sigma_a}
\end{equation}
Combination of the above two equations gives the solution for the scalar flux:
\begin{equation}
    \phi(x) = \psi_{inc}\E2(\sigma_a x) + \frac{q}{2\sigma_a} \left(1 -
    \E2(\sigma_ax)\right) + \frac{q}{\sigma_a}.
\end{equation}

