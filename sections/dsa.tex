
\chapter{\uppercase {Accelerated Iterative Solution to the LO Equations}}
\label{chp:dsa}

The fully-discrete, S$_2$-like LO equations 
cannot be directly inverted easily in higher spatial dimensions.  
To demonstrate a possible path forward in
higher dimensions, we have investigated the use of a standard
source iteration scheme~\cite{lewis} to invert the scattering terms in the linearized LO
equations during each Newton iteration.  As
material properties become more diffusive (e.g., $c_v$ is small and $\sigma_a$ is
large), the effective scattering cross sections becomes large.  This results in a spectral radius of source iterations that approaches
unity~\cite{morel_ldtrt}.  These regimes are typical in TRT simulations, so an
acceleration method for iterative solutions is critical. 
We have accelerated the source iterations with a nearly-consistent diffusion synthetic acceleration
(DSA) method known as WLA~\cite{wla,wla_thesis}. We have also recast the DSA method as a preconditioner to an iterative
Krylov solution~\cite{larson_morel_sn} of the LO equations.  Generally, Krylov
methods mitigate acceleration losses due to inconsistencies in the acceleration
method.  In higher dimensions, the use of a Krylov method is necessary for effective
acceleration for nearly-consistent acceleration methods in problems with
mixed optical thicknesses~\cite{larson_morel_sn}, e.g., typical radiative transfer
problems.  Also, applying the preconditioned-Krylov approach allows for the use of
spatially lumped DSA equations as a preconditioner, with the LO equations using an LD or
HO spatial closure.

The remainder of this chapter is structured as follows:  The source
iteration solution to the LO equations is detailed.  Then, the equations for the WLA DSA
method are derived and the acceleration algorithm is given.  The DSA method is then recast
as a preconditoner to a GMRES solution of the scattering iteration equations.  Finally,
convergence results are compared for several test problems.

\section{Source Iteration Solution to the Linearized LO Equations}
\label{sec:si}

The time-discrete LO equations, after linearization, can be solved with a source iteration
method~\cite{lewis,morel_dsa,mcclarren_notes}.  In the source iteration
process the scattering source is lagged, which
uncouples unknowns between the two half ranges.  This produces a lower-triangular
system where the spatial unknowns can be solved for sequentially along the two directions of flow via a
standard sweeping procedure~\cite{lewis,morel_ldtrt}.  Beginning at the left boundary, the
positive unknowns can be determined for each cell from $i=1,\ldots,N_c$; because the
inflow to the $i$-th cell is defined from the previous cell or boundary condition, a local system
of equations can be solved for the $\mom{\phi}_{L,i}^+$ and $\mom{\phi}_{R,i}^+$ unknowns.
The negative direction unknowns are
determined similarly, starting from the
right boundary towards the left.  The newly computed half-range
intensities can then be used to estimate the scattering source for the next iteration.  This
process is repeated until convergence.  

The source iteration process can be written in operator notation as
\begin{equation}
    \B M \u\psi^{l+1} = \frac{1}{2}\B S\u\psi^{l} + \u{Q},
\end{equation}
where $\B M$ is the LO streaming and removal operator (i.e., the left-hand side of
Eqs.~\eqref{eq:exact_lmomp}--\eqref{eq:exact_rmomm} without the scattering terms
included), $\u\psi$ is a vector of the half-range FE moment unknowns, and the vector
$\u Q$ contains the fixed source terms resulting from the linearized emission source and previous
time step moments, for each equation.  For the $i$-th element and the $L$
equation, for both half-ranges, the source terms are
\begin{equation}
    (\u Q)_{i,L}^\pm = \frac{h_i\mom{\phi}_L^{\pm,n}}{c\Delta t} + \frac{1}{2}h_if_i \sigma_{a,i} a c \mom{
        (T^{n})^4}_{L,i},
\end{equation}
the scattering operator terms are
\begin{equation}
    (\B S \u\psi^l)^{\pm}_{i,L} = h_i\left(\sigma_{a,i} (1-f_i) + \sigma_{s,i}\right)
    \left(\mom{ \phi^l}_{i,L}^{+,n+1} + \mom{ \phi^l}_{i,L}^{-,n+1}\right),
\end{equation}
and the streaming and removal terms, for the LD spatial closure, are
\begin{multline}
    (\B M \u\psi^{l+1})^{\pm}_{i,L} = 
    -2{\mu}_{i-1/2}^{n+1,+} \left(2\mom{\phi^{l+1}}_{R,i-1}^{n+1,+} -
    \mom{\phi^{l+1}}_{L,i-1}^{n+1,+} \right) +  \cur {\mu}_{L,i}^{n+1,+}
    \mom{\phi^{l+1}}_{L,i}^{n+1,+} \\
  +  \cur\mu_{R,i}^{n+1,+}
  \mom{\phi^{l+1}}_{R,i}^{n+1,+} +  \left(\sigma_{t,i}^{n+1}+\frac{1}{c \Delta t} \right) h_i 
  \mom{\phi^{l+1}}_{L,i}^{n+1,+}
\end{multline}
where all consistency terms are known from the HO solver.
Equivalent expressions are defined for the $R$ moment equations and boundary conditions,
forming a fully defined set of equations.  The process of sweeping is denoted as $\B
M^{-1}$.

The scattering inversion must be
performed within each Newton iteration.  Thus, for the $m$-th \emph{Newton step}, the source
iteration process is defined as
\begin{enumerate}
    \item Evaluate effective scattering and absorption cross sections with
        ${\{T^m_i:\;\, i=1,2,\ldots,N_c\}}$.
    \item\label{en:si_beg} Compute new scattering source $\frac{1}{2}\B S \u\psi^l$.
    \item Perform sweeps to calculate $\u \psi^{l+1} = \B M^{-1} \B S\u\psi^{l} + \B M^{-1} \u Q$
    \item\label{en:si_end} If $\|\u\psi^{l+1} - \u\psi^{l} \|_2 < $ tolerance
        $\|\u\psi^{l+1}\|_2$, move to next Newton step. Else, repeat steps~\ref{en:si_beg}--\ref{en:si_end}.
\end{enumerate}

\section{Linear Diffusion Synthetic Acceleration}

A form of DSA referred to as the WLA method is used to accelerate the source
iterations~\cite{wla,wla_thesis}. 
Between each sweep, an error equation for the scattering iterations is solved with an approximate angular
discretization of the transport equation.  The estimated error is used to correct the
zeroth moment of the intensity unknowns.  In operator notation, the DSA equations for a single
iteration are
\begin{align}
    \B M \u \psi^{l+1/2} &= \frac{1}{2}\B S \u \psi^{l} + \u Q \\
    \B D \delta \u\phi^{l+1/2} &= \B S (\u\psi^{l+1/2} - \u\psi^{l}) \label{eq:dsa_oper} \\
     \u \psi^{l+1} &= \u \psi^{l+1/2} + \delta \u \phi^{l+1/2},
\end{align}
where $\delta \phi$ represents the error in the mean intensity unknowns.
The operator $\B D$ represents a diffusion-like approximation to the transport equation. The DSA equations contain a standard
finite-difference diffusion discretization that can be more efficiently
inverted than the S$_2$-like equations that are being accelerated (particularly in higher
spatial dimensions), but will accurately resolve the
slowly-converging, diffusive error modes. 

It is important for the spatial discretization of Eq.~\eqref{eq:dsa_oper} to be closely related to the discretization of the LO equations for the
acceleration to be effective and stable~\cite{adams_dsa}.  The WLA method first solves a spatially-continuous
discretization of the diffusion equation
for the error at faces $\{x_{i+1/2}:\; i=0,1,\ldots,N_c\}$.  The error on the faces is then mapped onto the
volumetric moment unknowns via a LD discretization of the P$_1$ equations~\cite{wla}.
The LD mapping resolves issues that would occur in optically-thick cells, and the
continuous diffusion equation is accurate in the EDL where acceleration is important~\cite{adams_dsa}.

The continuous diffusion equation and mapping equations for the WLA method are derived in Appendix~\ref{sec:wla_derivation}.
To allow for the use of lumped
or standard LD in the DSA equations, we introduce the factor $\theta$, with
$\theta=1/3$ for standard
LD, and $\theta=1$ for lumped LD.  With suppression of the time indices, the diffusion equation for the face at $x_{i+1/2}$ is 
\begin{multline}\label{eq:dsa_lumped}
    \left(\frac{\sigma_{a,i+1} h_{i+1}}{4}\left(1 - \theta\right)  -
        \frac{D_{i+1}}{h_{i+1}}\right)\delta\phi_{i+3/2} + \left(\frac{D_{i+1}}{h_{i+1}} +
        \frac{D_{i}}{h_i} + \left(\frac{1+\theta}{2} \right)\left[\frac{\sigma_{a,i+1} h_{i+1}}{2} + \frac{\sigma_{a,i}
        h_{i}}{2}\right]\right)\delta\phi_{i+1/2} \\ + \left(\frac{\sigma_{a,i}
        h_{i}}{4}\left(1 - \theta\right) -
        \frac{D_{i}}{h_{i}}\right)\delta\phi_{i-1/2} = \frac{h_{i+1}}{2} \mom{q}_{L,i+1} +
        \frac{h_{i}}{2}\mom{q}_{R,i}
        \;\,. 
\end{multline}
The source in Eq.~\eqref{eq:dsa_lumped} is the residual for a given scattering iteration~\cite{morel_dsa,lewis}
\begin{equation}\label{eq:dsa_src}
    \mom{q}_{L/R,i} = \sigma_{s,i}\left(\mom{\phi^{l+1/2}}_{L/R,i} -
    \mom{\phi^{l}}_{L/R,i}\right).
\end{equation}
It is noted that there is no need to define the source differently for the lumped or
standard LD DSA equations, because the source is in terms of moments, which are provided
by the LO unknowns.
%or LD, resulting in equations that are
%consistent.  For instance, for lumped LD, we desire the right equation to have $\mom{q}_{R,i} =
%\sigma_s(\phi^{l+1/2}_{R,i} - \phi^{l}_{R,i})$.   The lumped LO system uses a spatial closure and
%linear relation where the edge values are defined as the moments, e.g.,
%$\mom{\phi}^+_{R,i} \equiv \phi^+_{R,i}$.
%Substituting the closure into the right side of Eq.~\eqref{eq:dsa_src} gives the desired
%source $\mom{q}_{R,i} = \sigma_{s,i}\left({\phi^{l+1/2}}_{R,i} -
%    \phi^{l}_{R,i}\right) $. 

The upwinding in the LO system exactly satisfies the inflow boundary conditions, therefore
a vacuum boundary condition is applied to the diffusion error equations.
Application of Eq.~\eqref{eq:dsa_bc_app} gives the left boundary condition:
\begin{equation}\label{eq:bc_dsa}
    \left(\frac{1}{2}+ \sigma_{a,1}h_1\frac{1+\theta}{4} - \frac{D_1}{h_1}\right)\delta\phi_{1/2} +
    \left( {\sigma_{a,1}{h_1}}\frac{1-\theta}{4} - \frac{D_1}{h_1}  \right)\delta \phi_{3/2} =
    \frac{h_1}{2} \mom{q}_{L,1}
\end{equation}
The boundary condition for the right-most face is
\begin{equation}\label{eq:bc_dsar}
    \left(\frac{1}{2}+ \sigma_{a,I}h_I\frac{1+\theta}{4} - \frac{D_I}{h_I}\right)\delta\phi_{I+1/2} +
    \left( {\sigma_{a,I}{h_I}}\frac{1-\theta}{4} - \frac{D_I}{h_I}  \right)\delta\phi_{I-1/2} =
    \frac{h_I}{2} \mom{q}_{R,I}
\end{equation}
where $I$ is the index of the right-most cell. 

The system of equations formed from Eqs.~\eqref{eq:bc_dsa},~\eqref{eq:bc_dsar},
and~\eqref{eq:dsa_lumped} is symmetric and has a matrix bandwidth of 3, compared to the
bandwidth of 7 of the LO equations.  The system is solved directly with a banded matrix solver. Then,
Eq.~\eqref{eq:update1}--\eqref{eq:update2} are solved in each cell to map the face errors onto
an LD representation over the interior.   It is noted that unlike fully consistent DSA equations, the WLA-DSA algorithm does not
preserve particle balance to round off.  This is because the mapping procedure uses an
approximate inflow to each cell, which is inconsistent with the partial outflows from
adjacent cells.  Thus, overall, our algorithm will only conserve energy to the order of
scattering iteration convergence.

Because we are interested in the time-dependent solution, we need to accelerate the solution for the
half-range intensities, rather than just the zeroth moment. We do not accelerate the first
moment of the angular intensity, as the solution for $\Delta J$ is inaccurate due to the
approximations introduced.  The error in the half-range moments, using
the lumping notation\footnote{The $\theta$ is different for lumped LD or standard LD
    discretization of the DSA equations, independent of the discretization in the LO
equations.} 
\begin{align}\label{eq:up1}
    \mom{\delta \phi}^{\pm}_L =&  \frac{1+\theta}{4}\delta \phi^{\pm}_{L} +
    \frac{1-\theta}{4}\delta \phi^{\pm}_{R}  \\ \label{eq:up2}
    \mom{\delta \phi}^{\pm}_R =&  \frac{1-\theta}{4}\delta \phi^{\pm}_{L} +
    \frac{1+\theta}{4}\delta \phi^{\pm}_{R}      ,
\end{align}

\subsection{The WLA-DSA Accelerated Source Iteration Algorithm}

We define the process of solving the diffusion like equations and mapping the error
unknowns back onto the moment equations as the operator $\B D^{-1}$.
The source iteration with linear DSA procedure, for the $m$-th Newton iteration, is
\begin{enumerate}
    \item Evaluate effective scattering and absorption cross sections with
        ${\{T^m_i:\;\, i=1,2,\ldots,N_c\}}$.
    \item\label{en:dsa_beg} Compute new scattering source $\B S \u \psi^l$.
    \item Perform sweeps to calculate $\u \psi^{l+1/2} = \B M^{-1} \B S \u \psi^{l} + \B
        M^{-1} \u Q$
    \item Perform DSA iteration to solve $ \delta \u \phi^{l+1/2} = \B D^{-1}
        \B S(\u \psi^{l+1/2}-\u \psi^l)$
        \begin{itemize}
    \item Solve continuous DSA equations, i.e., Eqs.~\eqref{eq:dsa_lumped}, \eqref{eq:bc_dsar},
and~\eqref{eq:dsa_lumped}, for \\${\{\delta
            \phi^{l+1/2}_{i+1/2}:\;i=0,1,\ldots,N_c\}}$.
        \item Map the continuous error onto the moment unknowns, via
            Eq.~\eqref{eq:update1}--\eqref{eq:update2}.
    \end{itemize}
\item Add correction to the moment unknowns with Eq.~\eqref{eq:up1} and~\eqref{eq:up2}.
    \item\label{en:dsa_end} If $\|\u \psi^{l+1} - \u \psi^{l} \|_2 < $ tolerance $\|\u
        \psi^{l+1} \|_2$, then
        exit. Else, repeat steps~\ref{en:dsa_beg}--\ref{en:dsa_end}.
\end{enumerate}

\section{GMRES Solution to the LO Equations}

The source iteration procedure can be recast as an iterative solution to a matrix
equation. Using operator notation, we manipulate
the moment equations to form a matrix equation:
\begin{equation}\label{eq:gmres}
    \left(\B I  - \B M^{-1}\B S\right) \u \psi = \B M^{-1} \u Q,
\end{equation}
where $\B I$ is an identity matrix.  The GMRES method is used to approximate
the solution to the above
linear system. The GMRES method is an iterative Krylov solution method for asymmetric, sparse matrix
equations.  Approximate solution to the above equation is formed by producing the $l$-th
Krylov vector $\u \psi^l$, where $\u \psi^l$ minimizes the norm of the residual for
Eq.~\eqref{eq:gmres} and is a member of the $l$-th orthonormalized Krylov
subspace~\cite{saad}.  To form the Krylov subspace in each iteration, the
matrix operator, i.e., the left-hand side of Eq.~\eqref{eq:gmres}, is applied to the
previous Krylov vector.  Rather than building the full matrix system, we apply the
operation of $\B S$ and $\B M^{-1}$ as detailed in Sec.~\eqref{sec:si} to apply $\left(\B
I  - \B M^{-1}\B S\right)$ to the generated Krylov vectors. 

The GMRES method will generally converge faster than the source iteration
procedure~\cite{morel_dsa}.  However, as the system becomes scattering dominated,
convergence will degrade.  To improve the convergence rate, we precondition the GMRES
system with a solution of the WLA-DSA equations.  The goal of preconditioning is to efficiently apply an
operator to the equation that will approximate the inverse of the matrix operator. Left
preconditioning~\cite{saad} was applied to the above system.  In matrix form, we write the preconditioned GMRES equations as
\begin{equation}
    \left(\B I + \B D^{-1} \B S\right)\left(\B I - \B M^{-1} \B S \right) \u \psi = \left( \B I + \B
    D^{-1}\B S\right) \B M^{-1}\u  Q.
\end{equation}
The operation of $\left(\B I + \B D^{-1} \B S\right)^{-1}$ is equivalent to the DSA
procedure where the scattering residual is simply $\B S \psi^{l}$ and the correction is
directly added to the passed in Krylov vector.

The open-source library \verb{mgmres{ was modified to implement the matrix-free version of
    the GMRES procedure. The
    infrastructure from the source iteration with DSA procedure is reused to provide
    the operation of $\left(\B I + \B D^{-1} \B S \right)\left(\B I  - \B M^{-1}\B
    S\right)$ applied to the Krylov vectors returned from the GMRES solver.
 The preconditioned-GMRES algorithm is
\begin{enumerate}
    \item Evaluate effective scattering and absorption cross sections with
        ${\{T^m_i:\;\, i=1,2,\ldots,N_c\}}$.
    \item Initialize Krylov vector as $\u \psi^0 = 0$.
    \item \label{en:dsa1}Form source vector $\u b$ with sweep:  $\u b = \B
        M^{-1}\u Q$.
    \item Apply left-preconditioner operator to $\u b$, so $\u b \leftarrow \left( \B I + \B
        D^{-1}\B S\right) \B M^{-1}\u Q$
    \item \label{en:gm1}Apply sweep and subtraction to Krylov vector: $\u \psi^{l+1/2} = \left (\B I -
       \B M^{-1}\right) \u \psi^l$.
    \item\label{en:dsa2} Perform DSA iteration to determine $\u
        \psi^{l+1/2} \leftarrow \left(\B I + \B D^{-1} \B S\right) \u \psi^{l+1/2}$
    \item \label{en:gm2} Apply GMRES step to $\u b$ and $\u \psi^{l+1/2}$ to generate next Krylov vector
        $\u \psi^{l+1}$.
    \item If the norm of the residual for $\u \psi^{l+1}$  is below tolerance, then exit.
        Else, repeat steps~\ref{en:gm1}--\ref{en:gm2}.
\end{enumerate}
The convergence tolerance is relative to the initial residual of the first iteration with
non-zero $\u \psi$.
To perform GMRES without preconditioning, steps~\ref{en:dsa1}
and~\ref{en:dsa2} are removed.

\section{Computational Results}

We have tested the iterative solution methods for three test problems and compare the average number
of scattering iterations to converge.   For each simulation, three batches of 10,000 particles are ran for each HO
the single  HO solve per time step, and 200 spatial cells were used.   The average number of source iterations per
Newton step is recorded, as well as the total number of Newton
iterations per time step (there are two LO solves per time step).  The initial guess for
the effective scattering source is set to zero at the beginning of
each LO solve.   All
scattering iterations have a relative convergence of 10$^{-10}$.
For all DSA simulations, we have used the lumped spatial representation for the DSA
equations.
%It is noted that we are not interested in measuring the reduction of
%computational time in this section, because in 1D the LO equations can be directly solved
%efficiently anyways.
%We are just interested in ensuring that the acceleration methods can reduce the
%number of scattering iterations sufficiently.

The first test problem is the two material
problem in Sec~\ref{sec:two}.     The time step is increased linearly by 15\% each time step from $\Delta
t=0.001$ sh to reach a maximum time step size of $0.01$ sh.  The large time step sizes
increases the magnitude of the effective scattering cross section.
Table~\ref{tab:twomat_dsa_iters} compares iteration counts each method: unaccelerated
source iteration (SI), source iteration with DSA (SI-DSA), unaccelerated GMRES (GMRES),
and GMRES with DSA preconditioning (GMRES-DSA).  As demonstrated, DSA improves the
convergence of the source iteration method.  The preconditioned GMRES was more efficient
than standard GMRES.

The second test problem is a modification of the two material problem. The problem
specifications are the same as before except for modifications to the cross sections for
$x>0.5$ cm; in the right half of the domain, the
parameters are $\sigma_a = 20,000$ cm$^{-1}$, $\sigma_s=500 $ cm$^{-1}$.  This problem is
highly diffusive and nonlinear. The Newton method required damping with a damping factor of $0.6$ to
stably converge.  Table~\ref{tab:twomat_hard_dsa_iters} compares the iterations
Overall, the damping increases the number of Newton steps, as expected. For this problem,
acceleration is much more critical, reducing the number of scattering iterations by a
factor of 100.  

For the final test problem, we test the equilibrium diffusion limit problem from
Sec.~\ref{sec:edl_results}.  The problem was tested with standard LD and
lumping-equivalent LD spatial closures, with the DSA using the lumped representation in
both cases.  Table.~\ref{tab:edl_iters} compares scattering iterations for the EDL
problem.  There was minimal
degradation observed for the diffusion limit problem due to the difference in spatial
discretizations.  This is likely because both lumped LD and
LD representations produce accurate solutions in the EDL. 
\begin{table}[p]
    \centering
    \caption{\label{tab:twomat_dsa_iters} Scattering source iterations for the two
material problem.  Simulation end time is 1 sh.}
    \begin{tabular}{|ccc|} \hline
        Method & Sweeps/Newton Iter. & Newton Iters./Time Step \\ \hline
        SI     & 247.0 & 19.4                \\
        SI-DSA & 10.1   & 19.3      \\
        GMRES  & 13.1    &  19.4     \\
        GMRES-DSA & 7.7  &  19.3  \\ \hline
    \end{tabular}
\end{table}
\begin{table}[p]
    \centering
    \caption{\label{tab:twomat_hard_dsa_iters} Scattering source iterations for the
    modified, diffusive two material problem. Simulation end time is 2 sh.}
    \begin{tabular}{|ccc|} \hline
        Method & Sweeps/Newton Step & Newton Iters./Time Step \\ \hline
        SI        & 1037   &  25.2    \\
        SI-DSA    & 10.9   &  25.1   \\
        GMRES     & 11.6   &  25.1   \\
        GMRES-DSA & 6.0    &  25.2   \\ \hline
    \end{tabular}
\end{table}
\begin{table}[p]
    \centering
    \caption{\label{tab:edl_iters} Scattering source iterations for the equilibrium
    diffusion limit problem.  Simulation end time is 3 sh.}
    \begin{tabular}{|ccc|} \hline
        \multicolumn{3}{|c|}{LD LO Equations} \\ \hline
        Method    &  Sweeps/Newton Step & Newton Steps/LO Solve \\ \hline
        SI        & 357.4  &  8.4     \\
        SI-DSA    & 21.9   &  8.4     \\
        GMRES     & 36.5   &  8.4    \\
        GMRES-DSA & 13.3   &  8.4   \\ \hline 
         \multicolumn{3}{|c|}{Lumped LO Equations} \\ \hline
        SI        & 359.8         & 8.2      \\ 
        SI-DSA    & 14.6          & 8.2     \\ 
        GMRES     & 37.3          &  8.2   \\
        GMRES-DSA & 9.8           &  8.2   \\ \hline 
    \end{tabular}
\end{table}




