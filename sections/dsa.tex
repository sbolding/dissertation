\section{Diffusion Synthetic Acceleration}

To accelerate source iteration in the LO system, a version of WLA DSA is used.  The
following derivations are to solve a diffusion equation which can be used to compute
the source iteration error in the LO sweeps.

\subsection{Forming a Continuous Diffusion Equation}

Beginning with the P$_1$ equations for a steady-state problem
\begin{align}\label{eq:dsa_bal}
    \pderiv{J}{x} + \sigma_a \phi &= Q \\ \label{eq:p1}
    \sigma_t J + \frac{1}{3} \pderiv{\phi}{x} &= 0 \\
\end{align}
spatial finite element moments are taken. The spatial moments are defined as
\begin{align}
    \mom{\cdot}_L &= \frac{2}{h_i} \int_{\xl}^{\xr} \! \d x \,b_{L,i}(x)    \left( \cdot \right) \\
    \mom{\cdot}_R &= \frac{2}{h_i} \int_{\xl}^{\xr} \! \d x \,b_{R,i}(x) \left( \cdot \right).
\end{align}
where $b_{L,i}(x) = (\xr - x)/h_i$ and $b_{R,i}(x) = (x - \xl)/h_i$.
The scalar flux $\phi$ will ultimately be assumed continuous at faces. The scalar
flux is assumed linear on the interior of the cell, i.e.,
$\phi(x)=\phi_Lb_L(x) + \phi_Rb_R(x)$, for $x\in(\xl,\xr)$.   Taking the left moment,
evaluating integrals, and rearranging yields
\begin{equation}
    J_{i} - J_{\il}  + \frac{\sigma_{a,i}h_i}{2} \left(\frac{2}{3} \phi_{L,i} + \frac{1}{3}
    \phi_{R,i} \right) = \frac{h_i}{2} \mom{q}_{L,i}\,\,,
\end{equation}
where $J_i$ is the average of the current over the cell. The moments of $q$ are
not simplified to be compatible with the LO moment equations. For the $R$ moment
\begin{equation}
    J_{i+1/2} - J_{i}  + \frac{\sigma_{a,i}h_i}{2} \left(\frac{2}{3} \phi_{L,i} + \frac{1}{3}
    \phi_{R,i} \right) = \frac{h_i}{2} \mom{q}_{R,i}\,\,.
\end{equation}
The equation for the $L$ moment is evaluated for cell $i+1$ and added to the $R$ moment
equation evaluated at $i$.  The current is assumed continuous at $\ir$ to eliminate
the face current from the system.  The sum of the two equations becomes
\begin{multline}
    J_{i+1} - J_{i} + \frac{\sigma_{a,i+1} h_{i+1}}{2}\left(\frac{2}{3} \phi_{L,i+1} +
    \frac{1}{3}\phi_{R,i+1}\right) + \frac{\sigma_{a,i} h_i}{2} \left( \frac{1}{3} \phi_{L,i} +
    \frac{2}{3}\phi_{R,i}\right) =\\ \frac{h}{2} \left(\mom{q}_{L,i+1} + \mom{q}_{R,i}
    \right).
\end{multline}
The scalar flux is assumed continuous at each face, i.e., $\phi_{L,i+1} = \phi_{R,i}
\equiv \phi_{i+1/2}$.  We then approximate the cell-averaged currents with Fick's law
as \begin{equation}\label{eq:ficks}
    J_{i} = -D_i \frac{\phi_{i+1/2} - \phi_{i-1/2}}{h_i}.
\end{equation}
Combining the definition and rearranging yields the following discrete diffusion
equation:
\begin{multline}
        \left(\frac{\sigma_{a,i+1} h_{i+1}}{6} -
        \frac{D_{i+1}}{h_{i+1}}\right)\phi_{i+3/2} + \left(\frac{D_{i+1}}{h_{i+1}} +
        \frac{D_{i}}{h_i} + \frac{\sigma_{a,i+1} h_{i+1}}{3} + \frac{\sigma_{a,i}
        h_{i}}{3}\right)\phi_{i+1/2} \\ + \left(\frac{\sigma_{a,i} h_{i}}{6} -
        \frac{D_{i}}{h_{i}}\right)\phi_{i-1/2} = \frac{h_{i+1}}{2} \mom{q}_{L,i+1} +
        \frac{h_{i}}{2}\mom{q}_{R,i}\;\,. 
\end{multline}
This system can be solved to get $\phi$ at each face.  To allow for the use of lumped
or standard LD in these equations, we introduce the factor $\theta$, with
$\theta=1/3$ for standard
LD, and $\theta=1$ for lumped LD.  The diffusion equation becomes
\begin{multline}
    \left(\frac{\sigma_{a,i+1} h_{i+1}}{4}\left(1 - \theta\right)  -
        \frac{D_{i+1}}{h_{i+1}}\right)\phi_{i+3/2} + \left(\frac{D_{i+1}}{h_{i+1}} +
        \frac{D_{i}}{h_i} + \left(\frac{1+\theta}{2} \right)\left[\frac{\sigma_{a,i+1} h_{i+1}}{2} + \frac{\sigma_{a,i}
        h_{i}}{2}\right]\right)\phi_{i+1/2} \\ + \left(\frac{\sigma_{a,i}
        h_{i}}{4}\left(1 - \theta\right) -
        \frac{D_{i}}{h_{i}}\right)\phi_{i-1/2} = \frac{h_{i+1}}{2} \mom{q}_{L,i+1} +
        \frac{h_{i}}{2}\mom{q}_{R,i}
        \;\,. 
\end{multline}

\subsubsection{Boundary Conditions}

The LO system exactly satisfies the inflow boundary conditions, therefore we choose a
vacuum boundary condition for the left-most cell.  The equation for the left moment
at the first cell is given by
\begin{equation}\label{eq:dsa_bc}
    J_{1} - J_{1/2}  + \frac{\sigma_{a,i}h_i}{2} \left(\frac{1+\theta}{2} \phi_{L,i}
    + \frac{1-\theta}{2}
    \phi_{R,i} \right) = \frac{h_i}{2} \mom{q}_{L,i}\,\,,
\end{equation}
The Marshak boundary condition for the vacuum inflow at face $x_{1/2}$ is given as
\begin{equation}
    J^+_{1/2} = 0 = \frac{\phi_{1/2}}{4} + \frac{J_{1/2}}{2},
\end{equation}
which can be solved for $J_{1/2}$.  Substitution of the above equation and
Eq.~\eqref{eq:ficks} into Eq.~\eqref{eq:dsa_bc} gives 
\begin{equation}
    \left(\frac{1}{2}+ \sigma_{a,1}h_1\frac{1+\theta}{4} - \frac{D_1}{h_1}\right)\phi_{1/2} +
    \left( {\sigma_{a,1}{h_1}}\frac{1-\theta}{4} - \frac{D_1}{h_1}  \right)\phi_{3/2} =
    \frac{h_i}{2} \mom{q}_{L,1}
\end{equation}
a similar expression can be derived for the last cell.


\subsection{Mapping Solution onto LD Unknowns}

Solution of the continuous diffusion equation in the previous section provides
correction values for $\phi$ on the faces, denoted as $\phi_{i+1/2}^C$. We now need
to determine the correction these results provide for the LD representation of
$\phi$. To do this, first we take the $L$ and $R$ finite element moments of the P$_1$
equations.  A LDFE dependence is assumed on the interior of the cell for $J$ and
$\phi$.  Taking moments of Eq.~\eqref{eq:dsa_bal} and simplifying yields
\begin{align}
    J_{\ir} - \frac{J_{L,i} + J_{R,i}}{2} + \frac{\sigma_{a,i} h_i}{2} \left(\frac{1}{3} \phi_{L,i} +
    \frac{2}{3}\phi_{R,i}\right) &= \frac{h_i}{2} \mom{q}_{R,i} \\
    \frac{J_{L,i} + J_{R,i}}{2} - J_{i-1/2} + \frac{\sigma_{a,i} h_i}{2}
    \left(\frac{2}{3} \phi_{L,i} +
    \frac{1}{3}\phi_{R,i}\right) &= \frac{h_i}{2} \mom{q}_{L,i}
\end{align}
The moment equations for Eq.~\eqref{eq:p1} are
\begin{align}
    \frac{1}{3}\left(\phi_{\ir} - \frac{\phi_{i,L} + \phi_{i,R}}{2}\right) +
    \frac{\sigma_{t,i} h_i}{2} \left(\frac{1}{3} J_{L,i} + \frac{2}{3}J_{R,i}\right)
    &= 0 \\
    \frac{1}{3}\left(\frac{\phi_{i,L} + \phi_{i,R}}{2} - \phi_{i-1/2} \right) +
    \frac{\sigma_{t,i} h_i}{2} \left(\frac{2}{3} J_{L,i} + \frac{1}{3}J_{R,i}\right)
    &= 0 
\end{align}

Using similar equation for all the inflow currents, the balance equations for $\phi$
become
The face terms $J_{i\pm 1/2}$ and $\phi_{i\pm 1/2}$ need to be eliminated from the
system. The scalar flux is assumed to be the value provided by the continuous
diffusion solution at each face, i.e., $\phi_{i\pm1/2} = \phi_{i\pm1/2}^C$.
The currents are decomposed into half-range values to decouple the equations
between cells.  At $x_{\ir}$, the current is composed as $J_{i+1/2} = J_{\ir}^+ - J_{\ir}^-$,
where $+$ and $-$ denote the positive and negative
half ranges of $\mu$, respectively.  Typically, the incoming current $J_{\ir}^-$ is
upwinded from cell $i+1$. However, we approximate the incoming current based on $\phi_{i+1/2}^C$.  
The P$_1$ approximation provides the following relation
\begin{equation}
    \phi = 2(J^+ + J^-).
\end{equation}
At $\xr$, the above expression is solved for the incoming current $J_{i+1/2}^-$.  The
total current becomes, with $\phi_{i+1/2}=\phi_{i+1/2}^C$,
\begin{equation}\label{eq:jelim}
    J_{\ir} = J_{\ir}^+ - J_{\ir}^- = 2J_{\ir}^+ - \frac{\phi_{i+1/2}^C}{2},
\end{equation}
In the positive direction, at the right face, the
values of $\phi$ and $J$ are based on the LD representation within the cell at that
face, i.e., $\phi_{R,i}$ and $J_{R,i}$.  The standard P$_1$ approximation for the
half-range currents and fluxes are used\cite{stacy}, i.e.,
\begin{align}
    J^{\pm} &= \frac{\gamma \phi}{2} \pm \frac{J}{2},
\end{align}
where $\gamma$ accounts for the difference between the LO parameters and the true
P$_1$ approximation. Thus, for the right face and positive half-range,
\begin{align}
    J_{\ir}^+ &= \frac{\gamma}{2}\phi_{i,R} + \frac{J_{i,R}}{2} 
\end{align}
A similar expression can be derived for $\xl$.  The total currents at each face are
thus
\begin{align}
    J_{i+1/2} &= \gamma\phi_{i,R} + J_{i,R} - \frac{\phi_{\ir}^C}{2} \\
    J_{i-1/2} &= \frac{\phi_{i-1/2}^C}{2} - \gamma \phi_{i,L} + J_{i,L}
\end{align}
Substitution of these results back into the LD balance equations and introduction of the
lumping notation yields the final equations 
\begin{align}
    \left(\gamma\phi_{i,R} + J_{i,R} - \frac{\phi_{\ir}^C}{2} \right) - \frac{J_{L,i} + J_{R,i}}{2} + \frac{\sigma_{a,i} h_i}{2} \left(
    \frac{(1-\theta)}{2} \phi_{L,i} +
    \frac{(1+\theta)}{2}\phi_{R,i}\right) &= \frac{h_i}{2} \mom{q}_{R,i} \\
    \frac{J_{L,i} + J_{R,i}}{2} -\left(\frac{\phi_{i-1/2}^C}{2} - \gamma \phi_{i,L} +
    J_{i,L}\right) + \frac{\sigma_{a,i} h_i}{2} \left(
    \frac{(1+\theta)}{2} \phi_{L,i} +
    \frac{(1-\theta)}{2}\phi_{R,i}\right) &= \frac{h_i}{2} \mom{q}_{L,i} 
    \\
    \frac{1}{3}\left(\phi_{i+1/2}^C - \frac{\phi_{i,L} + \phi_{i,R}}{2}\right) +
    \frac{\sigma_{t,i} h_i}{2}\left( \frac{(1-\theta)}{2} J_{L,i} +
    \frac{(1+\theta)}{2}J_{R,i}\right)    &= 0 \\
    \frac{1}{3}\left(\frac{\phi_{i,L} + \phi_{i,R}}{2} - \phi^C_{i-1/2} \right) +
    \frac{\sigma_{t,i} h_i}{2} \left( \frac{(1+\theta)}{2} J_{L,i} +
    \frac{(1-\theta)}{2}J_{R,i}\right) &= 0 .
\end{align}
The above equations are completely local to each cell and fully defined.  The system can be
solved for the the desired unknowns
$\phi_{i,L}$, $\phi_{i,R}$, $J_{i,L}$, and $J_{i,R}$.

%\subsection{Wareing version of mapping solution onto LD Unknowns using P$_1$ incident currents}
%
%Solution of the continuous diffusion equation in the previous section provides
%correction values for $\phi$ on the faces, denoted as $\phi_{i+1/2}^C$. We now need
%to determine the correction these results provide for the LD representation of
%$\phi$. To do this, first we take the $L$ and $R$ finite element moments of the P$_1$
%equations.  A LDFE dependence is assumed on the interior of the cell for $J$ and
%$\phi$.  Taking moments of Eq.~\eqref{eq:dsa_bal} and simplifying yields
%\begin{align}
%    J_{\ir} - \frac{J_{L,i} + J_{R,i}}{2} + \frac{\sigma_{a,i} h_i}{2} \left(\frac{1}{3} \phi_{L,i} +
%    \frac{2}{3}\phi_{R,i}\right) &= \frac{h_i}{2} \mom{q}_{R,i} \\
%    \frac{J_{L,i} + J_{R,i}}{2} - J_{i-1/2} + \frac{\sigma_{a,i} h_i}{2}
%    \left(\frac{2}{3} \phi_{L,i} +
%    \frac{1}{3}\phi_{R,i}\right) &= \frac{h_i}{2} \mom{q}_{L,i}
%\end{align}
%The moment equations for Eq.~\eqref{eq:p1} are
%\begin{align}
%    \frac{1}{3}\left(\phi_{\ir} - \frac{\phi_{i,L} + \phi_{i,R}}{2}\right) +
%    \frac{\sigma_{t,i} h_i}{2} \left(\frac{1}{3} J_{L,i} + \frac{2}{3}J_{R,i}\right)
%    &= 0 \\
%    \frac{1}{3}\left(\frac{\phi_{i,L} + \phi_{i,R}}{2} - \phi_{i-1/2} \right) +
%    \frac{\sigma_{t,i} h_i}{2} \left(\frac{2}{3} J_{L,i} + \frac{1}{3}J_{R,i}\right)
%    &= 0 
%\end{align}
%The currents and fluxes on faces are decomposed into half-range values. This allows
%the cells to be decoupled by using values of $\phi_{i+1/2}^C$. 
%
%First, the definitions at face $\xr$ are considered.  The current is composed as $J_{i+1/2} = J_{\ir}^+ - J_{\ir}^-$ and the scalar flux as $\phi_{i+1/2} =
%\phi_{i+1/2}^+ + \phi_{i+1/2}^-$, where $+$ and $-$ denote the positive and negative
%half ranges of $\mu$, respectively.  The negative direction values $J_{\ir}^-$ and
%$\phi_{\ir}^-$ are upwinded from cell $i+1$. However, we approximate these values
%based on a P$_1$ expansion of the incident flux, e.g.,
%\begin{equation}
%    \psi_{\ir}^C = \frac{\phi_{\ir}^C + 3\mu J_{\ir}^C}{2}, \quad \mu<0
%\end{equation}
%We do not know $J_{\ir}^C$, but taking angular moments of the LD closure equations
%will provide enough equations to eliminate $J_{i\pm 1/2}^C$ from the system with an
%approximation.
%In the positive direction, at the right face, the
%values of $\phi$ and $J$ are based on the LD representation within the cell at that
%face, i.e., $\phi_{R,i}$ and $J_{R,i}$.  The standard P$_1$ approximation for the
%half-range currents and fluxes are used\cite{stacy}, i.e.,
%\begin{align}
%    J^{\pm} &= \frac{\gamma \phi}{2} \pm \frac{J}{2} \\
%    \phi{\pm} &= \frac{\phi}{2} \pm \frac{3J\gamma}{2}.
%\end{align}
%Thus, for the right face and positive half-range,
%\begin{align}
%    J_{\ir}^+ &= \frac{\gamma}{2}\phi_{i,R} + \frac{J_{i,R}}{2} \\
%    \phi_{\ir}^+  &= \frac{\phi_{i,R}}{2} + \frac{3\gamma}{2} J_{i,R}
%\end{align}
%The currents and partial fluxes are defined similarly for $\xl$.  Combining these
%results, equations can be formed for the face terms as
%\begin{align}
%    J_{i+1/2} &= \frac{\gamma}{2}\phi_{i,R} + \frac{J_{i,R}}{2} -
%    \left(\frac{\gamma \phi_{\ir}^C}{2} - \frac{J_{\ir}^C}{2}\right) \label{eq:jpclose} \\
%    J_{i-1/2} &= \left(\frac{\gamma \phi_{\il}^C}{2} + \frac{J_{\il}^C}{2}\right) - \left(\frac{\gamma}{2}\phi_{i,L} -
%    \frac{J_{i,L}}{2}\right) \label{eq:jlclose} \\
%    \phi_{i+1/2} &= \frac{1}{2}\phi_{i,R} + \frac{3\gamma J_{i,R}}{2} +
%    \left( \frac{\phi_{\ir}^C}{2} - \frac{3\gamma J_{\ir}^C}{2}  \right)
%    \label{eq:ppclose}  \\
%    \phi_{i-1/2} &= \frac{1}{2}\phi_{i,L} - \frac{3\gamma J_{i,L}}{2} +
%    \left(\frac{\phi_{i-1/2}^C}{2} + \frac{3 \gamma J_{\il}^C}{2}  \right)
%    \label{eq:plclose}
%\end{align}
%The lumping notation is introduced, which yields the balance equations
%\begin{align}
%    J_{\ir}^C - \frac{J_{L,i} + J_{R,i}}{2} + \frac{\sigma_{a,i} h_i}{2} \left(
%    \frac{(1-\theta)}{2} \phi_{L,i} +
%    \frac{(1+\theta)}{2}\phi_{R,i}\right) &= \frac{h_i}{2} \left(
%    \frac{(1-\theta)}{2} q_{L,i} +
%    \frac{(1+\theta)}{2}q_{R,i}\right) \\
%    \frac{J_{L,i} + J_{R,i}}{2} - J_{i-1/2}^C + \frac{\sigma_{a,i} h_i}{2}\left(
%    \frac{(1+\theta)}{2} \phi_{L,i} +
%    \frac{(1-\theta)}{2}\phi_{R,i}\right) &= \frac{h_i}{2} \left(
%    \frac{(1+\theta)}{2} q_{L,i} +
%    \frac{(1-\theta)}{2}q_{R,i}\right)      \\
%    \frac{1}{3}\left(\phi_{i+1/2}^C - \frac{\phi_{i,L} + \phi_{i,R}}{2}\right) +
%    \frac{\sigma_{t,i} h_i}{2}\left( \frac{(1-\theta)}{2} J_{L,i} +
%    \frac{(1+\theta)}{2}J_{R,i}\right)    &= 0 \\
%    \frac{1}{3}\left(\frac{\phi_{i,L} + \phi_{i,R}}{2} - \phi^C_{i-1/2} \right) +
%    \frac{\sigma_{t,i} h_i}{2} \left( \frac{(1+\theta)}{2} J_{L,i} +
%    \frac{(1-\theta)}{2}J_{R,i}\right) &= 0 .
%\end{align}
%The $q$ sources on the edges are obtained from the source moments as $q_{L,i} =
%2\mom{q}_{L,i} - \mom{q}_{R,i}$ and $q_{R,i} =
%2\mom{q}_{R,i} - \mom{q}_{L,i}$.
%Angular moments of the closure equation are manipulated to form the necessary 2 remaining equations.
%The sum of Eq.~\eqref{eq:plclose} and~\eqref{eq:ppclose} and the sum of
%Eq.~\eqref{eq:jlclose} and~\eqref{eq:jpclose} provide the two auxiliary equations,
%where we make the assumption of $J_{i+1/2}=J_{i+1/2}^C$ (this is not an
%approximation, as much as a side effect of my notation. The currents can be
%eliminated before we ever make the assumption that $\phi_{i+1/2}=\phi_{i+1/2}^C$):
%\begin{align}
%    \frac{1}{2}\left(\phi_{L,i} + \phi_{R,i}\right) + 3\gamma\left(J_{R,i} -
%    J_{L,i}\right) - 3\gamma\left(J_{i+1/2}^C - J_{i-1/2}^C\right) &=
%    \frac{1}{2}\left(\phi_{i+1/2}^C + \phi_{i-1/2}^C\right) \\
%    \frac{\gamma}{2}\left(\phi_{R,i} - \phi_{L,i}\right) + \frac{1}{2}\left(J_{R,i} +
%    J_{L,i}\right) - \frac{1}{2}\left(J_{i+1/2}^C + J_{i-1/2}^C\right) &=
%    \frac{\gamma}{2}\left( \phi_{i+1/2} - \phi_{i-1/2} \right)
%\end{align}
%These equations are completely local to each cell and fully defined.  The system can be
%solved for the desired unknowns
%$\phi_{i,L}$, $\phi_{i,R}$, $J_{i,L}$, and $J_{i,R}$, as well as the auxiliary
%unknowns $J_{i\pm 1/2}^C$.


\subsection{DSA Source Definition}

The above discretization procedure is used to determine the error in the scalar flux.
The sources $\mom{q}_{L/R}$ thus need to be defined.  They are simply the residual in
the scattering iterations, given by
\begin{equation}
    q = \sigma_s\left(\phi^{l+1/2} - \phi^{l}\right).
\end{equation}
The spatial moments are straight forward:
\begin{equation}
    \mom{q}_{L,i} = \sigma_{s,i}\left(\mom{\phi^{l+1/2}}_{L,i} -
    \mom{\phi^{l}}_{L,i}\right)
\end{equation}
The above equation is valid for lumping or standard LD.  This is because the LO
moments are defined differently for LLD or LD, resulting in equations that are
consistent.  For instance, for lumped LD, the LO system uses the spatial closure that
the edge value is defined as the moment, i.e., $\mom{\phi}_{R,i} \equiv \phi_{R,i}$.
For a standard lumped source, we desire the right equation to have $\mom{q}_{R,i} =
\sigma_s(\phi^{l+1/2}_{R,i} - \phi^{l}_{R,i})$.  Substituting the lumped closure into
the right hand side of this equation gives back the original equation, i.e., $\mom{q}_{R,i} = \sigma_{s,i}\left(\mom{\phi^{l+1/2}}_{R,i} -
    \mom{\phi^{l}}_{R,i}\right) $.  The same is true for standard LD.

\subsection{Updating the LO Unkowns}

We now have a correction to $J$ and $\phi$ for the volumetric finite element
unknowns.
Because we are interested in the time-dependent solution, we need to accelerate the solution for the
half-range fluxes, rather than just the scalar flux. We only accelerate the zeroth
moment of the angular intensity.  The error in the scalar intensities are defined as
\begin{equation}\label{eq:p1psi}
    \delta \phi^{\pm} = \frac{\delta \phi}{2} \pm \frac{3 \delta J}{4}
\end{equation}
Spatial moments are taken of $\delta \phi^{\pm}$, using the lumping notation of LD on the interior
\begin{align}
    \mom{\delta \phi^{\pm}}_L =&  \frac{1+\theta}{2}\delta \phi^{\pm}_{L} +
    \frac{1-\theta}{2}\delta \phi^{\pm}_{R}  \\
    \mom{\delta \phi^{\pm}}_R =&  \frac{1-\theta}{2}\delta \phi^{\pm}_{L} +
    \frac{1+\theta}{2}\delta \phi^{\pm}_{R}      ,
\end{align}
where Eq.~\eqref{eq:p1psi} can be used to get in terms of $\delta \phi_{L/R}$ and $\delta J_{L/R}$.
It is noted that for consistency, the updates to the moments depend on the lumping
notation, even though the sources are defined the same in both cases.








