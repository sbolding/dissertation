\section{Diffusion Synthetic Acceleration}

To accelerate source iteration in the LO system, a version of WLA DSA is used.  The
following derivations are to solve a diffusion equation which can be used to compute
the source iteration error in the LO sweeps.

\subsection{Forming a Continuous Diffusion Equation}

Beginning with the P$_1$ equations for a steady-state problem
\begin{align}\label{eq:dsa_bal}
    \pderiv{J}{x} + \sigma_a \phi &= Q \\ \label{eq:p1}
    \sigma_t J + \frac{1}{3} \pderiv{\phi}{x} &= 0 \\
\end{align}
spatial finite element moments are taken. The spatial moments are defined as
\begin{align}
    \mom{\cdot}_L &= \frac{2}{h_i} \int_{\xl}^{\xr} \! \d x \,b_{L,i}(x)    \left( \cdot \right) \\
    \mom{\cdot}_R &= \frac{2}{h_i} \int_{\xl}^{\xr} \! \d x \,b_{R,i}(x) \left( \cdot \right).
\end{align}
where $b_{L,i}(x) = (\xr - x)/h_i$ and $b_{R,i}(x) = (x - \xl)/h_i$.
The scalar flux $\phi$ will ultimately be assumed continuous.  For now it assumed LD, i.e.,
$\phi(x)=\phi_Lb_L(x) + \phi_Rb_R(x)$, for $x\in(\xl,\xr)$.   Taking the left moment,
evaluating integrals, and rearranging yields
\begin{equation}
    J_{i} - J_{\il}  + \frac{\sigma_{a,i}h_i}{2} \left(\frac{2}{3} \phi_{L,i} + \frac{1}{3}
    \phi_{R,i} \right) = \frac{h_i}{2} \mom{q}_{L,i}\,\,,
\end{equation}
where $J_i$ is the average of the current over the cell. The moments of $q$ are
not simplified to be compatible with the LO moment equations. For the $R$ moment
\begin{equation}
    J_{i+1/2} - J_{i}  + \frac{\sigma_{a,i}h_i}{2} \left(\frac{2}{3} \phi_{L,i} + \frac{1}{3}
    \phi_{R,i} \right) = \frac{h_i}{2} \mom{q}_{R,i}\,\,.
\end{equation}
The equation for the $L$ moment is evaluated for cell $i+1$ and added to the $R$ moment
equation evaluated at $i$.  The current is assumed continuous at $\ir$ to eliminate
the face current from the system.  The sum of the two equations becomes
\begin{multline}
    J_{i+1} - J_{i} + \frac{\sigma_{a,i+1} h_{i+1}}{2}\left(\frac{2}{3} \phi_{L,i+1} +
    \frac{1}{3}\phi_{R,i+1}\right) + \frac{\sigma_{a,i} h_i}{2} \left( \frac{1}{3} \phi_{L,i} +
    \frac{2}{3}\phi_{R,i}\right) =\\ \frac{h}{2} \left(\mom{q}_{L,i+1} + \mom{q}_{R,i}
    \right).
\end{multline}
The scalar flux is assumed continuous at each face, i.e., $\phi_{L,i+1} = \phi_{R,i}
\equiv \phi_{i+1/2}$.  We then approximate the cell-averaged currents with Fick's law
as \begin{equation}\label{eq:ficks}
    J_{i} = -D_i \frac{\phi_{i+1/2} - \phi_{i-1/2}}{h_i}.
\end{equation}
Combining the definition and rearranging yields the following discrete diffusion
equation:
\begin{multline}
        \left(\frac{\sigma_{a,i+1} h_{i+1}}{6} -
        \frac{D_{i+1}}{h_{i+1}}\right)\phi_{i+3/2} + \left(\frac{D_{i+1}}{h_{i+1}} +
        \frac{D_{i}}{h_i} + \frac{\sigma_{a,i+1} h_{i+1}}{3} + \frac{\sigma_{a,i}
        h_{i}}{3}\right)\phi_{i+1/2} \\ + \left(\frac{\sigma_{a,i} h_{i}}{6} -
        \frac{D_{i}}{h_{i}}\right)\phi_{i-1/2} = \frac{h_{i+1}}{2} \mom{q}_{L,i+1} +
        \frac{h_{i}}{2}\mom{q}_{R,i}\;\,. 
\end{multline}
This system can be solved to get $\phi$ at each face.

\subsubsection{Boundary Conditions}

The LO system exactly satisfies the inflow boundary conditions, therefore we choose a
vacuum boundary condition for the left-most cell.  The equation for the left moment
at the first cell is given by
\begin{equation}\label{eq:dsa_bc}
    J_{1} - J_{1/2}  + \frac{\sigma_{a,i}h_i}{2} \left(\frac{2}{3} \phi_{L,i} + \frac{1}{3}
    \phi_{R,i} \right) = \frac{h_i}{2} \mom{q}_{L,i}\,\,,
\end{equation}
The Marshak boundary condition for the vacuum inflow at face $x_{1/2}$ is given as
\begin{equation}
    J^+_{1/2} = 0 = \frac{\phi_{1/2}}{4} + \frac{J_{1/2}}{2},
\end{equation}
which can be solved for $J_{1/2}$.  Substitution of the above equation and
Eq.~\eqref{eq:ficks} into Eq.~\eqref{eq:dsa_bc} gives 
\begin{equation}
    \left(\frac{1}{2}+ \frac{\sigma_{a,1}h_1}{3} - \frac{D_1}{h_1}\right)\phi_{1/2} +
    \left( \frac{\sigma_{a,1}{h_1}}{6} - \frac{D_1}{h_1}  \right)\phi_{3/2} =
    \frac{h_i}{2} \mom{q}_{L,1}
\end{equation}
a similar expression can be derived for the last cell.


\subsection{Mapping Solution onto LD Unknowns}

Solution of the continuous diffusion equation in the previous section provides
correction values for $\phi$ on the faces, denoted as $\phi_{i+1/2}^C$. We now need
to determine the correction these results provide for the LD representation of
$\phi$. To do this, first we take the $L$ and $R$ finite element moments of the P$_1$
equations.  A LDFE dependence is assumed on the interior of the cell for $J$ and
$\phi$.  Taking moments of Eq.~\eqref{eq:dsa_bal} and simplifying yields
\begin{align}
    J_{\ir} - \frac{J_{L,i} + J_{R,i}}{2} + \frac{\sigma_{a,i} h_i}{2} \left(\frac{1}{3} \phi_{L,i} +
    \frac{2}{3}\phi_{R,i}\right) &= \frac{h_i}{2} \mom{q}_{R,i} \\
    \frac{J_{L,i} + J_{R,i}}{2} - J_{i-1/2} + \frac{\sigma_{a,i} h_i}{2}
    \left(\frac{2}{3} \phi_{L,i} +
    \frac{1}{3}\phi_{R,i}\right) &= \frac{h_i}{2} \mom{q}_{L,i}
\end{align}
The moment equations for Eq.~\eqref{eq:p1} are
\begin{align}
    \frac{1}{3}\left(\phi_{\ir} - \frac{\phi_{i,L} + \phi_{i,R}}{2}\right) +
    \frac{\sigma_{t,i} h_i}{2} \left(\frac{1}{3} J_{L,i} + \frac{2}{3}J_{R,i}\right)
    &= 0 \\
    \frac{1}{3}\left(\frac{\phi_{i,L} + \phi_{i,R}}{2} - \phi_{i-1/2} \right) +
    \frac{\sigma_{t,i} h_i}{2} \left(\frac{2}{3} J_{L,i} + \frac{1}{3}J_{R,i}\right)
    &= 0 
\end{align}
The currents and fluxes on faces are decomposed into half-range values. This allows
the cells to be decoupled by using values of $\phi_{i+1/2}^C$. 

First, the definitions at face $\xr$ are considered.  The current is composed as $J_{i+1/2} = J_{\ir}^+ - J_{\ir}^-$ and the scalar flux as $\phi_{i+1/2} =
\phi_{i+1/2}^+ + \phi_{i+1/2}^-$, where $+$ and $-$ denote the positive and negative
half ranges of $\mu$, respectively.  The negative direction values $J_{\ir}^-$ and
$\phi_{\ir}^-$ are upwinded from cell $i+1$. However, we approximate these values
based on $\phi_{i+1/2}^C$.  The incoming flux is assumed isotropic, which yields an
incoming current of $J^-_{\ir} = \gamma\frac{\phi_{\ir}^C}{2}$, where $\gamma$
accounts for the different between the LO equations estimate of the current compared
to the P$_1$ assumption of the flux that is used in the approximate equations.
Similarly, the half-range flux on the face is $\phi^-_{\ir} =
\frac{\phi_{\ir}^-}{2}$.   In the positive direction, at the right face, the
values of $\phi$ and $J$ are based on the LD representation within the cell at that
face, i.e., $\phi_{R,i}$ and $J_{R,i}$.  The standard P$_1$ approximation for the
half-range currents and fluxes are used\cite{stacy}, i.e.,
\begin{align}
    J^{\pm} &= \frac{\gamma \phi}{2} \pm \frac{J}{2} \\
    \phi{\pm} &= \frac{\phi}{2} \pm \frac{3J\gamma}{2}.
\end{align}
Thus, for the right face and positive half-range,
\begin{align}
    J_{\ir}^+ &= \frac{\gamma}{2}\phi_{i,R} + \frac{J_{i,R}}{2} \\
    \phi_{\ir}^+  &= \frac{\phi_{i,R}}{2} + \frac{3\gamma}{2} J_{i,R}
\end{align}
The currents and partial fluxes are defined similarly for $\xl$.  Combining these
results, the remaining equations necessary for solving the system are
\begin{align}
    J_{i+1/2} &= \frac{\gamma}{2}\phi_{i,R} + \frac{J_{i,R}}{2} -
    \frac{\gamma}{2}\phi_{\ir}^C \\
    J_{i-1/2} &= \frac{\gamma}{2}\phi_{\il}^C - \left(\frac{\gamma}{2}\phi_{i,L} -
    \frac{J_{i,L}}{2}\right) \\
    \phi_{i+1/2} &= \frac{1}{2}\phi_{i,R} + \frac{3\gamma J_{i,R}}{2} +
    \frac{1}{2}\phi_{\ir}^C  \\
    \phi_{i-1/2} &= \frac{1}{2}\phi_{i,L} - \frac{3\gamma J_{i,L}}{2} +
    \frac{1}{2}\phi_{\il}^C  
\end{align}
These equations are substituted back into the original system to produce
\begin{align}
\frac{\gamma}{2}\phi_{i,R} + \frac{J_{i,R}}{2} -
    \frac{\gamma}{2}\phi_{\ir}^C    - \frac{J_{L,i} + J_{R,i}}{2} + \frac{\sigma_{a,i} h_i}{2} \left(\frac{1}{3} \phi_{L,i} +
    \frac{2}{3}\phi_{R,i}\right) &= \frac{h_i}{2} \mom{q}_{R,i} \\
    \frac{J_{L,i} + J_{R,i}}{2} -\left(\frac{\gamma}{2}\phi_{\il}^C -
    \frac{\gamma}{2}\phi_{i,L} + \frac{J_{i,L}}{2}\right) + \frac{\sigma_{a,i} h_i}{2}
    \left(\frac{2}{3} \phi_{L,i} +
    \frac{1}{3}\phi_{R,i}\right) &= \frac{h_i}{2} \mom{q}_{L,i} \\
    \frac{1}{3}\left(\frac{1}{2}\phi_{i,R} + \frac{3\gamma J_{i,R}}{2} +
    \frac{1}{2}\phi_{\ir}^C - \frac{\phi_{i,L} + \phi_{i,R}}{2}\right) +
    \frac{\sigma_{t,i} h_i}{2} \left(\frac{1}{3} J_{L,i} + \frac{2}{3}J_{R,i}\right)
    &= 0 \\
    \frac{1}{3}\left(\frac{\phi_{i,L} + \phi_{i,R}}{2} - \left[\frac{1}{2}\phi_{i,L} - \frac{3\gamma J_{i,L}}{2} +
    \frac{1}{2}\phi_{\il}^C\right] \right) +
    \frac{\sigma_{t,i} h_i}{2} \left(\frac{2}{3} J_{L,i} + \frac{1}{3}J_{R,i}\right)
    &= 0 
\end{align}
These equations are completely local to each cell and fully defined.  The system can be
solved for the the desired unknowns
$\phi_{i,L}$, $\phi_{i,R}$, $J_{i,L}$, and $J_{i,R}$.

\subsection{Alternative update equations}

As an alternative approach, we can use the following equation (which is true for
P$_1$ expansion of the flux) to eliminate the incoming
currents:
\begin{equation}
    \phi = 2(J^+ + J^-)
\end{equation}
At a face, the continuous solution provides $\phi$, and 
the current are eliminated in terms of the outflow current on that face.  For the
left face, the total current then becomes
\begin{equation}\label{eq:jelim}
    J_{\il} = J_{\il}^+ - J_{\il}^- = \frac{\phi}{2} - 2J_{\il}^-
\end{equation}
Substituting the continuous solution, the current becomes
\begin{equation}
    J_{\il} = \frac{\phi_{\il}^C}{2} - 2J^-_{\il} = \frac{\phi_{\il}^C}{2} -
    2\left(\frac{\gamma}{2}\phi_{i,L} - \frac{J_{i,L}}{2}\right)
\end{equation}
Using similar equation for all the inflow currents, the balance equations for $\phi$
become
\begin{align}
    \left(\gamma\phi_{i,R} + J_{i,R} - \frac{\phi_{\ir}^C}{2} \right) - \frac{J_{L,i} + J_{R,i}}{2} + \frac{\sigma_{a,i} h_i}{2} \left(\frac{1}{3} \phi_{L,i} +
    \frac{2}{3}\phi_{R,i}\right) &= \frac{h_i}{2} \mom{q}_{R,i} \\
    \frac{J_{L,i} + J_{R,i}}{2} -\left(\frac{\phi_{i-1/2}^C}{2} - \gamma \phi_{i,L} +
    J_{i,L}\right) + \frac{\sigma_{a,i} h_i}{2}
    \left(\frac{2}{3} \phi_{L,i} +
    \frac{1}{3}\phi_{R,i}\right) &= \frac{h_i}{2} \mom{q}_{L,i} 
\end{align}
We can also break up $\phi$ in Eq.~\eqref{eq:jelim} into its half-range components,
producing the equations
\begin{align}
    \left(\gamma\phi_{i,R} + J_{i,R} - \left[ \gamma\frac{\phi_{\ir}^C}{2} +
    \frac{\phi_{i,R}}{4} + \frac{3}{4} \gamma J_{i,R} \right] \right) - \frac{J_{L,i} + J_{R,i}}{2} + \frac{\sigma_{a,i} h_i}{2} \left(\frac{1}{3} \phi_{L,i} +
    \frac{2}{3}\phi_{R,i}\right) &= \frac{h_i}{2} \mom{q}_{R,i} \\
    \frac{J_{L,i} + J_{R,i}}{2} -\left(\left[ \frac{\gamma}{2} \phi_{i-1/2}^C +
    \frac{\phi_{i,L}}{4} - \frac{3}{4}\gamma J_{i,L} \right] - \gamma \phi_{i,L} +
    J_{i,L}\right) + \frac{\sigma_{a,i} h_i}{2}
    \left(\frac{2}{3} \phi_{L,i} +
    \frac{1}{3}\phi_{R,i}\right) &= \frac{h_i}{2} \mom{q}_{L,i} 
\end{align}


\subsection{DSA Source Definition}

The above discretization procedure is used to determine the error in the scalar flux.
The sources $\mom{q}_{L/R}$ thus need to be defined.  They are simply the residual in
the scattering iterations, given by
\begin{equation}
    q = \sigma_s\left(\phi^{l+1/2} - \phi^{l}\right).
\end{equation}
The spatial moments are straight forward:
\begin{equation}
    \mom{q}_{L,i} = \sigma_{s,i}\left(\mom{\phi^{l+1/2}}_{L,i} -
    \mom{\phi^{l}}_{L,i}\right)
\end{equation}

\subsection{Updating the LO Unkowns}

We now have a correction to $J$ and $\phi$ for the volumetric finite element
unknowns.
Because we are interested in the time-dependent solution, we need to accelerate the solution for the
half-range fluxes, rather than just the scalar flux. Beginning with the P$_1$
approximation for the angular intensity
\begin{equation}
 \delta I(\mu) = \frac{\delta\phi}{4\pi} + \frac{3\delta J}{4\pi} \mu.
\end{equation}
Taking the half range integrals gives
\begin{equation}\label{eq:p1psi}
    \delta \phi^{\pm} = \frac{\delta \phi}{2} \pm \frac{3 \delta J}{4}
\end{equation}
Spatial moments are taken of $\delta \phi^{\pm}$, using the LD definition on the interior
\begin{align}
    \mom{\delta \phi^{\pm}}_L =&  \frac{2}{3}\delta \phi^{\pm}_{L} + \frac{1}{3}\delta \phi^{\pm}_{R}  \\
    \mom{\delta \phi^{\pm}}_R =&  \frac{1}{3}\delta \phi^{\pm}_{L} + \frac{2}{3}\delta \phi^{\pm}_{R}      ,
\end{align}
where Eq.~\eqref{eq:p1psi} can be used to get in terms of $\delta \phi_{L/R}$ and $\delta J_{L/R}$.
Thus, each of the volumetric moments can be updated accordingly.








