
\chapter{\uppercase {Accelerated Iterative Solution to the LO Equations}}

As described in Sec.~\ref{sec:lo_sol}, the fully-discrete, S$_2$-like LO equations
are not easy to directly inverted efficiently in higher spatial dimensions.  
To demonstrate a possible path forward in
higher dimensions, we have investigated the use of a standard
source iteration scheme to invert the scattering terms in the LO equations.  As
material properties become more diffusive (e.g., $c_v$ is small and $\sigma_a$ is
large), the effective scattering cross sections becomes large.  This results in a spectral radius of the source iterations that approaches
unity~\cite{morel_ldtrt}.  These regimes are typical in TRT simulations, so an
acceleration method for an iterative solution is critical. 
We have accelerated the source iterations with a nearly-consistent diffusion synthetic acceleration
(DSA) method known as WLA~\cite{wla,wla_thesis}.

REWRITE: STUFF ABOUT LUMPING ETC.
We have also recast the DSA method as a preconditioner to an iterative
Krylov solution~\cite{larson_morel_sn} of the LO equations.  Generally, Krylov
methods mitigate acceleration losses due to inconsistencies in the acceleration
method.  In higher dimensions, the use of a Krylov method is necessary for effective
acceleration for nearly-consistent acceleration methods in problems with
mixed optical thicknesses~\cite{larson_morel_sn}, e.g., typical radiative transfer
problems.  Also, applying the preconditioned-Krylov approach allows for the use of
spatially lumped DSA  equations as a preconditioner, with the LO equations using
alternative fix-up methods.  We expect better acceleration performance (DID WE GET IT?) when the
LDD discretization is introduced into the LO equations. 

In the remainder of this chapter is structured as follows:  The source
iteration solution to the LO equations is detailed.  Then, the equations for the WLA DSA
method are derived and the acceleration algorithm is given.  The DSA method is then recast
as a preconditoner to a GMRES solution to the scattering iteration equations.  Finally,
results are given for a modified test problem which requires the use of acceleration.

\section{Source Iteration Solution to the Linearized LO Equations}

The linearized LO equations can be solved with a source iteration
method~\cite{lewis,morel_dsa,mcclarren_notes}.  In the source iteration
process the scattering source is lagged, which
uncouples unknowns between the two half-ranges.  This produces a lower-triangular
system where the spatial unknowns can be solve for sequentially along the two directions of flow via a
standard sweeping procedure~\cite{lewis,morel_ldtrt}.  Beginning at the left boundary, the
positive unknowns can be determined for each cell from $i=1,\ldots,N_c$.  Because the
inflow to the $i$-th cell is defined from the previous cell or boundary condition, a local system
of equations can be solved for the $\mom{\phi}_{L,i}^+$ and $\mom{\phi}_{R,i}^+$ unknowns.
The negative direction unknowns are
determined similarly, starting from the
right boundary towards the left.  The newly computed half-range
intensities can then be used to estimate the scattering source for the next iteration.  This
process is repeated until convergence.  

The source iteration process can be written in operator notation as
\begin{equation}
    \B M \psi^{l+1} = S\psi^{l} + Q,
\end{equation}
where $M$ is the discretized LO streaming plus removal operator (i.e., scattering is not
included) and $\psi$ is a vector of the half-range, FE moment unknowns.  The vector $Q$
contains the fixed source terms resulting from the linearized emission source and previous
time step moments, for each equation; the terms for the $i$-th element and the $L$ equation
are
\begin{equation}
    (\B Q)_{i,L}^\pm = \frac{\mom{\phi}_L^\pm}{c\Delta t} + \frac{1}{2}f_i \sigma_{a,i} a c \mom{
        (T^{n})^4}_{L,i}
\end{equation}
The scattering operator terms for the $i$-th element and the $L$ equations is
\begin{equation}
    (\B S \psi^l)^{\pm}_{i,L} = \frac{1}{2}\left(\sigma_{a,i} (1-f_i) + \sigma_{s,i}\right)
    \left(\mom{ \phi^l}_{i,L}^+ + \mom{ \phi^l}_{i,L}^-\right).
\end{equation}
Equivalent expressions are defined for the $R$ moment equations and boundary conditions,
forming a fully defined set of equations.

The scattering inversion must be
performed within each Newton iteration.  Thus, for the $m$-th Newton step, the source
iteration process is defined as
\begin{enumerate}
    \item Evaluate effective scattering and absorption cross sections with
        ${\{T^m_i:\;\, i=1,2,\ldots,N_c\}}$.
    \item\label{en:si_beg} Compute new scattering source $\B S \psi^l$.
    \item Perform sweeps to calculate $\psi^{l+1} = \B M^{-1} \B S\psi^{l} + \B M^{-1} Q$
    \item\label{en:si_end} If $\|\psi^{l+1} - \psi^{l} \| < $ tolerance, exit
    \item Else: repeat steps~\ref{en:si_beg}--\ref{en:si_end}.
\end{enumerate}

\section{Linear Diffusion Synthetic Acceleration}

A form of DSA referred to as the WLA method is used to accelerate the source
iterations~\cite{wla,wla_thesis}. 
Between each source iteration, a residual equation is formed that provides 
the error in the current scattering iteration. The DSA method uses an approximate,
lower-order operator to solve the error equation for a correction to the zeroth angular moment of the
intensity, i.e., the mean intensity $\phi(x)$.  The DSA equations contain a standard
finite-difference diffusion discretization that can be more efficiently
solved than the S$_2$-like equations that are being accelerated, but will accurately resolve the
slowly converging diffusive error modes.  It is important for the spatial discretization of the DSA
equations to be closely related to the discretization of the LO equations for the
acceleration to be effective and stable~\cite{adams_dsa}.  The WLA method first solves a spatially-continuous
discretization of the diffusion equation
for the iterative error on faces.  The error on the faces is then mapped onto the
volumetric moment unknowns via a LD discretization of diffusion equation~\cite{wla}.
The LD mapping resolves issues that would occur in optically-thick cells, while the
continuous diffusion equation is accurate in the EDL where acceleration is important~\cite{adams_dsa}.

In the remained of this section we derive the discretized diffusion and update equations.  To simplify notation, we
derive the diffusion equations from a generic transport equation with isotropic scattering
and source $q_0$, i.e.,
\begin{equation}\label{eq:ss_trans}
    \mu \pderiv{I}{x} + \sigma_t I = \frac{\sigma_s}{2}\left( \phi(x) + q_0\right)
\end{equation}
The WLA-DSA algorithm is then detailed.

\subsection{Forming a Continuous Diffusion Equation}

First, a continuous spatial discretization of a diffusion equation is derived.  
The mean intensity $\phi$ will ultimately be assumed continuous at faces to produce a
standard three-point finite-difference diffusion discretization. 
The zeroth and first $\mu$ moment of Eq.~\eqref{eq:ss_trans} produce the $P_1$
equations~\cite{lewis,wla_thesis}, i.e., 
\begin{align}\label{eq:dsa_bal}
    \pderiv{J}{x} + \sigma_a \phi &= q_0 \\ \label{eq:p1}
    \sigma_t J + \frac{1}{3} \pderiv{\phi}{x} &= 0, 
\end{align}
where $q_0=\int_{-1}^1 q(x,\mu) \dd \mu $.
The spatial finite element moments (defined by Eq.~\eqref{eq:x_moml} and~\eqref{x_momr})
are taken of the above equations. 
The mean intensity is assumed linear on the interior of the cell, i.e.,
$\phi(x)=\phi_Lb_L(x) + \phi_Rb_R(x)$, for $x\in(\xl,\xr)$.   Taking the left moment,
evaluating integrals, and rearranging yields
\begin{equation}
    J_{i} - J_{\il}  + \frac{\sigma_{a,i}h_i}{2} \left(\frac{2}{3} \phi_{L,i} + \frac{1}{3}
    \phi_{R,i} \right) = \frac{h_i}{2} \mom{q}_{L,i}\,\,,
\end{equation}
where $J_i$ is the average of the flux $J$ over the cell. The moments of $q$ are
not simplified to be compatible with the error equations which are in terms of moments. For the $R$ moment
\begin{equation}
    J_{i+1/2} - J_{i}  + \frac{\sigma_{a,i}h_i}{2} \left(\frac{2}{3} \phi_{L,i} + \frac{1}{3}
    \phi_{R,i} \right) = \frac{h_i}{2} \mom{q}_{R,i}\,\,.
\end{equation}
The equation for the $L$ moment is evaluated for cell $i+1$ and added to the $R$ moment
equation evaluated at $i$.  The flux $J$ is assumed continuous at $\ir$ to eliminate
the face fluxes from the equations.  The sum of the two equations becomes
\begin{multline}\label{eq:diff_noclose}
    J_{i+1} - J_{i} + \frac{\sigma_{a,i+1} h_{i+1}}{2}\left(\frac{2}{3} \phi_{L,i+1} +
    \frac{1}{3}\phi_{R,i+1}\right) + \frac{\sigma_{a,i} h_i}{2} \left( \frac{1}{3} \phi_{L,i} +
    \frac{2}{3}\phi_{R,i}\right) =\\ \frac{h}{2} \left(\mom{q}_{L,i+1} + \mom{q}_{R,i}
    \right).
\end{multline}
The mean intensity is approximated as continuous at each face, i.e., $\phi_{L,i+1} = \phi_{R,i}
\equiv \phi_{i+1/2}$.  Adding the $L$ and $R$ moments of Eq.~\eqref{eq:p1} together, with
the continuous approximation for $\phi_{i+1/2}$, produces a discrete Fick's law equation~\cite{stacy}
\begin{equation}\label{eq:ficks}
    J_{i} = -D_i \frac{\phi_{i+1/2} - \phi_{i-1/2}}{h_i},
\end{equation}
where $D_i = 1/(3\sigma_{t,i})$.
Substitution of Eq.~\eqref{eq:ficks} into Eq.~\eqref{eq:diff_noclose} and rearranging yields the following discrete diffusion
equation:
\begin{multline}
        \left(\frac{\sigma_{a,i+1} h_{i+1}}{6} -
        \frac{D_{i+1}}{h_{i+1}}\right)\phi_{i+3/2} + \left(\frac{D_{i+1}}{h_{i+1}} +
        \frac{D_{i}}{h_i} + \frac{\sigma_{a,i+1} h_{i+1}}{3} + \frac{\sigma_{a,i}
        h_{i}}{3}\right)\phi_{i+1/2} \\ + \left(\frac{\sigma_{a,i} h_{i}}{6} -
        \frac{D_{i}}{h_{i}}\right)\phi_{i-1/2} = \frac{h_{i+1}}{2} \mom{q}_{L,i+1} +
        \frac{h_{i}}{2}\mom{q}_{R,i}\;\,. 
\end{multline}
To allow for the use of lumped
or standard LD in these equations, we introduce the factor $\theta$, with
$\theta=1/3$ for standard
LD, and $\theta=1$ for lumped LD.  The diffusion equation becomes
\begin{multline}\label{eq:dsa_lumped}
    \left(\frac{\sigma_{a,i+1} h_{i+1}}{4}\left(1 - \theta\right)  -
        \frac{D_{i+1}}{h_{i+1}}\right)\phi_{i+3/2} + \left(\frac{D_{i+1}}{h_{i+1}} +
        \frac{D_{i}}{h_i} + \left(\frac{1+\theta}{2} \right)\left[\frac{\sigma_{a,i+1} h_{i+1}}{2} + \frac{\sigma_{a,i}
        h_{i}}{2}\right]\right)\phi_{i+1/2} \\ + \left(\frac{\sigma_{a,i}
        h_{i}}{4}\left(1 - \theta\right) -
        \frac{D_{i}}{h_{i}}\right)\phi_{i-1/2} = \frac{h_{i+1}}{2} \mom{q}_{L,i+1} +
        \frac{h_{i}}{2}\mom{q}_{R,i}
        \;\,. 
\end{multline}
Summation over all cells forms a system of equations for $\phi$ at each face.  

\subsubsection{Diffusion Boundary Conditions}

The upwinding in the LO system exactly satisfies the inflow boundary conditions, therefore
a vacuum boundary condition is applied to the diffusion error equations.  The equation for the left moment
at the first cell is given by
\begin{equation}\label{eq:dsa_bc}
    J_{1} - J_{1/2}  + \frac{\sigma_{a,i}h_i}{2} \left(\frac{1+\theta}{2} \phi_{L,i}
    + \frac{1-\theta}{2}
    \phi_{R,i} \right) = \frac{h_i}{2} \mom{q}_{L,i}\,\,,
\end{equation}
The Marshak boundary condition for the vacuum inflow at face $x_{1/2}$ is given as
\begin{equation}
    J^+_{1/2} = 0 = \frac{\phi_{1/2}}{4} + \frac{J_{1/2}}{2},
\end{equation}
which can be solved for $J_{1/2}$.  Substitution of the above equation and
Eq.~\eqref{eq:ficks} into Eq.~\eqref{eq:dsa_bc} gives 
\begin{equation}
    \left(\frac{1}{2}+ \sigma_{a,1}h_1\frac{1+\theta}{4} - \frac{D_1}{h_1}\right)\phi_{1/2} +
    \left( {\sigma_{a,1}{h_1}}\frac{1-\theta}{4} - \frac{D_1}{h_1}  \right)\phi_{3/2} =
    \frac{h_i}{2} \mom{q}_{L,1}
\end{equation}
A similar expression can be derived for the right-most cell.

\subsection{Mapping Solution onto LD Unknowns}

Solution of the continuous diffusion equation will provide an approximation to the error
of $\phi$ on faces, denoted as $\phi_{i+1/2}^C$. We now need
to determine the correction these results provide for the LD representation of
$\phi$. To do this, first we take the $L$ and $R$ finite element moments of the P$_1$
equations.  A LDFE dependence is assumed on the interior of the cell for $J$ and
$\phi$.  Taking moments of Eq.~\eqref{eq:dsa_bal} and simplifying yields
\begin{align}
    J_{\ir} - \frac{J_{L,i} + J_{R,i}}{2} + \frac{\sigma_{a,i} h_i}{2} \left(\frac{1}{3} \phi_{L,i} +
    \frac{2}{3}\phi_{R,i}\right) &= \frac{h_i}{2} \mom{q}_{R,i} \\
    \frac{J_{L,i} + J_{R,i}}{2} - J_{i-1/2} + \frac{\sigma_{a,i} h_i}{2}
    \left(\frac{2}{3} \phi_{L,i} +
    \frac{1}{3}\phi_{R,i}\right) &= \frac{h_i}{2} \mom{q}_{L,i}
\end{align}
The moment equations for Eq.~\eqref{eq:p1} are
\begin{align}
    \frac{1}{3}\left(\phi_{\ir} - \frac{\phi_{i,L} + \phi_{i,R}}{2}\right) +
    \frac{\sigma_{t,i} h_i}{2} \left(\frac{1}{3} J_{L,i} + \frac{2}{3}J_{R,i}\right)
    &= 0 \\
    \frac{1}{3}\left(\frac{\phi_{i,L} + \phi_{i,R}}{2} - \phi_{i-1/2} \right) +
    \frac{\sigma_{t,i} h_i}{2} \left(\frac{2}{3} J_{L,i} + \frac{1}{3}J_{R,i}\right)
    &= 0 
\end{align}
The face terms $J_{i\pm 1/2}$ and $\phi_{i\pm 1/2}$ need to be eliminated from the
system. The scalar flux is assumed to be the value provided by the continuous
diffusion solution at each face, i.e., $\phi_{i\pm1/2} = \phi_{i\pm1/2}^C$.
The fluxes are decomposed into half-range values to decouple the equations
between cells.  At $x_{\ir}$, the current is composed as $J_{i+1/2} = J_{\ir}^+ - J_{\ir}^-$,
where $+$ and $-$ denote the positive and negative
half ranges of $\mu$, respectively.  We approximate the incoming fluxes, e.g.,
$J_{i+1/2}^-$, based on $\phi_{i+1/2}^C$.  
The P$_1$ approximation provides the following relation~\cite{wla_thesis}
\begin{equation}
    \phi = 2(J^+ + J^-).
\end{equation}
At $\xr$, the above expression is solved for the incoming current $J_{i+1/2}^-$.  The
total current becomes, with $\phi_{i+1/2}=\phi_{i+1/2}^C$,
\begin{equation}\label{eq:jelim}
    J_{\ir} = J_{\ir}^+ - J_{\ir}^- = 2J_{\ir}^+ - \frac{\phi_{i+1/2}^C}{2},
\end{equation}
In the positive direction, at the right face, the
values of $\phi$ and $J$ are based on the LD representation within the cell at that
face, i.e., $\phi_{R,i}$ and $J_{R,i}$.  The standard P$_1$ approximation for the
half-range fluxes is used\cite{stacy}, i.e.,
\begin{align}
    J^{\pm} &= \frac{\gamma \phi}{2} \pm \frac{J}{2},
\end{align}
where $\gamma$ accounts for the difference between the LO parameters and the true
P$_1$ approximation. Thus, for the right face and positive half-range,
\begin{align}
    J_{\ir}^+ &= \frac{\gamma}{2}\phi_{i,R} + \frac{J_{i,R}}{2} 
\end{align}
A similar expression can be derived for $\xl$.  The total fluxes at each face are
thus
\begin{align}
    J_{i+1/2} &= \gamma\phi_{i,R} + J_{i,R} - \frac{\phi_{\ir}^C}{2} \\
    J_{i-1/2} &= \frac{\phi_{i-1/2}^C}{2} - \gamma \phi_{i,L} + J_{i,L}
\end{align}
Substitution of these results back into the LD balance equations and introduction of the
lumping notation yields the final equations 
\begin{align}
    \left(\gamma\phi_{i,R} + J_{i,R} - \frac{\phi_{\ir}^C}{2} \right) - \frac{J_{L,i} + J_{R,i}}{2} + \frac{\sigma_{a,i} h_i}{2} \left(
    \frac{(1-\theta)}{2} \phi_{L,i} +
    \frac{(1+\theta)}{2}\phi_{R,i}\right) &= \frac{h_i}{2} \mom{q}_{R,i} \\
    \frac{J_{L,i} + J_{R,i}}{2} -\left(\frac{\phi_{i-1/2}^C}{2} - \gamma \phi_{i,L} +
    J_{i,L}\right) + \frac{\sigma_{a,i} h_i}{2} \left(
    \frac{(1+\theta)}{2} \phi_{L,i} +
    \frac{(1-\theta)}{2}\phi_{R,i}\right) &= \frac{h_i}{2} \mom{q}_{L,i} 
    \\
    \frac{1}{3}\left(\phi_{i+1/2}^C - \frac{\phi_{i,L} + \phi_{i,R}}{2}\right) +
    \frac{\sigma_{t,i} h_i}{2}\left( \frac{(1-\theta)}{2} J_{L,i} +
    \frac{(1+\theta)}{2}J_{R,i}\right)    &= 0 \\
    \frac{1}{3}\left(\frac{\phi_{i,L} + \phi_{i,R}}{2} - \phi^C_{i-1/2} \right) +
    \frac{\sigma_{t,i} h_i}{2} \left( \frac{(1+\theta)}{2} J_{L,i} +
    \frac{(1-\theta)}{2}J_{R,i}\right) &= 0 .
\end{align}
The above equations are completely local to each cell and fully defined, including for
boundary cells.  The system can be solved for the desired unknowns
$\phi_{i,L}$, $\phi_{i,R}$, $J_{i,L}$, and $J_{i,R}$, which represent the mapping of
$\phi_{i+1/2}^C$ onto the LD representation for $\phi(x)$.

%\subsection{Wareing version of mapping solution onto LD Unknowns using P$_1$ incident currents}
%
%Solution of the continuous diffusion equation in the previous section provides
%correction values for $\phi$ on the faces, denoted as $\phi_{i+1/2}^C$. We now need
%to determine the correction these results provide for the LD representation of
%$\phi$. To do this, first we take the $L$ and $R$ finite element moments of the P$_1$
%equations.  A LDFE dependence is assumed on the interior of the cell for $J$ and
%$\phi$.  Taking moments of Eq.~\eqref{eq:dsa_bal} and simplifying yields
%\begin{align}
%    J_{\ir} - \frac{J_{L,i} + J_{R,i}}{2} + \frac{\sigma_{a,i} h_i}{2} \left(\frac{1}{3} \phi_{L,i} +
%    \frac{2}{3}\phi_{R,i}\right) &= \frac{h_i}{2} \mom{q}_{R,i} \\
%    \frac{J_{L,i} + J_{R,i}}{2} - J_{i-1/2} + \frac{\sigma_{a,i} h_i}{2}
%    \left(\frac{2}{3} \phi_{L,i} +
%    \frac{1}{3}\phi_{R,i}\right) &= \frac{h_i}{2} \mom{q}_{L,i}
%\end{align}
%The moment equations for Eq.~\eqref{eq:p1} are
%\begin{align}
%    \frac{1}{3}\left(\phi_{\ir} - \frac{\phi_{i,L} + \phi_{i,R}}{2}\right) +
%    \frac{\sigma_{t,i} h_i}{2} \left(\frac{1}{3} J_{L,i} + \frac{2}{3}J_{R,i}\right)
%    &= 0 \\
%    \frac{1}{3}\left(\frac{\phi_{i,L} + \phi_{i,R}}{2} - \phi_{i-1/2} \right) +
%    \frac{\sigma_{t,i} h_i}{2} \left(\frac{2}{3} J_{L,i} + \frac{1}{3}J_{R,i}\right)
%    &= 0 
%\end{align}
%The currents and fluxes on faces are decomposed into half-range values. This allows
%the cells to be decoupled by using values of $\phi_{i+1/2}^C$. 
%
%First, the definitions at face $\xr$ are considered.  The current is composed as $J_{i+1/2} = J_{\ir}^+ - J_{\ir}^-$ and the scalar flux as $\phi_{i+1/2} =
%\phi_{i+1/2}^+ + \phi_{i+1/2}^-$, where $+$ and $-$ denote the positive and negative
%half ranges of $\mu$, respectively.  The negative direction values $J_{\ir}^-$ and
%$\phi_{\ir}^-$ are upwinded from cell $i+1$. However, we approximate these values
%based on a P$_1$ expansion of the incident flux, e.g.,
%\begin{equation}
%    \psi_{\ir}^C = \frac{\phi_{\ir}^C + 3\mu J_{\ir}^C}{2}, \quad \mu<0
%\end{equation}
%We do not know $J_{\ir}^C$, but taking angular moments of the LD closure equations
%will provide enough equations to eliminate $J_{i\pm 1/2}^C$ from the system with an
%approximation.
%In the positive direction, at the right face, the
%values of $\phi$ and $J$ are based on the LD representation within the cell at that
%face, i.e., $\phi_{R,i}$ and $J_{R,i}$.  The standard P$_1$ approximation for the
%half-range currents and fluxes are used\cite{stacy}, i.e.,
%\begin{align}
%    J^{\pm} &= \frac{\gamma \phi}{2} \pm \frac{J}{2} \\
%    \phi{\pm} &= \frac{\phi}{2} \pm \frac{3J\gamma}{2}.
%\end{align}
%Thus, for the right face and positive half-range,
%\begin{align}
%    J_{\ir}^+ &= \frac{\gamma}{2}\phi_{i,R} + \frac{J_{i,R}}{2} \\
%    \phi_{\ir}^+  &= \frac{\phi_{i,R}}{2} + \frac{3\gamma}{2} J_{i,R}
%\end{align}
%The currents and partial fluxes are defined similarly for $\xl$.  Combining these
%results, equations can be formed for the face terms as
%\begin{align}
%    J_{i+1/2} &= \frac{\gamma}{2}\phi_{i,R} + \frac{J_{i,R}}{2} -
%    \left(\frac{\gamma \phi_{\ir}^C}{2} - \frac{J_{\ir}^C}{2}\right) \label{eq:jpclose} \\
%    J_{i-1/2} &= \left(\frac{\gamma \phi_{\il}^C}{2} + \frac{J_{\il}^C}{2}\right) - \left(\frac{\gamma}{2}\phi_{i,L} -
%    \frac{J_{i,L}}{2}\right) \label{eq:jlclose} \\
%    \phi_{i+1/2} &= \frac{1}{2}\phi_{i,R} + \frac{3\gamma J_{i,R}}{2} +
%    \left( \frac{\phi_{\ir}^C}{2} - \frac{3\gamma J_{\ir}^C}{2}  \right)
%    \label{eq:ppclose}  \\
%    \phi_{i-1/2} &= \frac{1}{2}\phi_{i,L} - \frac{3\gamma J_{i,L}}{2} +
%    \left(\frac{\phi_{i-1/2}^C}{2} + \frac{3 \gamma J_{\il}^C}{2}  \right)
%    \label{eq:plclose}
%\end{align}
%The lumping notation is introduced, which yields the balance equations
%\begin{align}
%    J_{\ir}^C - \frac{J_{L,i} + J_{R,i}}{2} + \frac{\sigma_{a,i} h_i}{2} \left(
%    \frac{(1-\theta)}{2} \phi_{L,i} +
%    \frac{(1+\theta)}{2}\phi_{R,i}\right) &= \frac{h_i}{2} \left(
%    \frac{(1-\theta)}{2} q_{L,i} +
%    \frac{(1+\theta)}{2}q_{R,i}\right) \\
%    \frac{J_{L,i} + J_{R,i}}{2} - J_{i-1/2}^C + \frac{\sigma_{a,i} h_i}{2}\left(
%    \frac{(1+\theta)}{2} \phi_{L,i} +
%    \frac{(1-\theta)}{2}\phi_{R,i}\right) &= \frac{h_i}{2} \left(
%    \frac{(1+\theta)}{2} q_{L,i} +
%    \frac{(1-\theta)}{2}q_{R,i}\right)      \\
%    \frac{1}{3}\left(\phi_{i+1/2}^C - \frac{\phi_{i,L} + \phi_{i,R}}{2}\right) +
%    \frac{\sigma_{t,i} h_i}{2}\left( \frac{(1-\theta)}{2} J_{L,i} +
%    \frac{(1+\theta)}{2}J_{R,i}\right)    &= 0 \\
%    \frac{1}{3}\left(\frac{\phi_{i,L} + \phi_{i,R}}{2} - \phi^C_{i-1/2} \right) +
%    \frac{\sigma_{t,i} h_i}{2} \left( \frac{(1+\theta)}{2} J_{L,i} +
%    \frac{(1-\theta)}{2}J_{R,i}\right) &= 0 .
%\end{align}
%The $q$ sources on the edges are obtained from the source moments as $q_{L,i} =
%2\mom{q}_{L,i} - \mom{q}_{R,i}$ and $q_{R,i} =
%2\mom{q}_{R,i} - \mom{q}_{L,i}$.
%Angular moments of the closure equation are manipulated to form the necessary 2 remaining equations.
%The sum of Eq.~\eqref{eq:plclose} and~\eqref{eq:ppclose} and the sum of
%Eq.~\eqref{eq:jlclose} and~\eqref{eq:jpclose} provide the two auxiliary equations,
%where we make the assumption of $J_{i+1/2}=J_{i+1/2}^C$ (this is not an
%approximation, as much as a side effect of my notation. The currents can be
%eliminated before we ever make the assumption that $\phi_{i+1/2}=\phi_{i+1/2}^C$):
%\begin{align}
%    \frac{1}{2}\left(\phi_{L,i} + \phi_{R,i}\right) + 3\gamma\left(J_{R,i} -
%    J_{L,i}\right) - 3\gamma\left(J_{i+1/2}^C - J_{i-1/2}^C\right) &=
%    \frac{1}{2}\left(\phi_{i+1/2}^C + \phi_{i-1/2}^C\right) \\
%    \frac{\gamma}{2}\left(\phi_{R,i} - \phi_{L,i}\right) + \frac{1}{2}\left(J_{R,i} +
%    J_{L,i}\right) - \frac{1}{2}\left(J_{i+1/2}^C + J_{i-1/2}^C\right) &=
%    \frac{\gamma}{2}\left( \phi_{i+1/2} - \phi_{i-1/2} \right)
%\end{align}
%These equations are completely local to each cell and fully defined.  The system can be
%solved for the desired unknowns
%$\phi_{i,L}$, $\phi_{i,R}$, $J_{i,L}$, and $J_{i,R}$, as well as the auxiliary
%unknowns $J_{i\pm 1/2}^C$.


\subsection{DSA Source Definition}

The above discretization procedure is used to determine the error in the scalar flux.
The sources $\mom{q}_{L/R}$ for the error equations remain to be defined.  They are simply the residual in
the scattering iterations, given by~\cite{lewis,morel_dsa}
\begin{equation}
    q = \sigma_s\left(\phi^{l+1/2} - \phi^{l}\right).
\end{equation}
The spatial moments are straight forward:
\begin{equation}
    \mom{q}_{L,i} = \sigma_{s,i}\left(\mom{\phi^{l+1/2}}_{L,i} -
    \mom{\phi^{l}}_{L,i}\right)
\end{equation}
The above equation is valid for lumping or standard LD.  This is because the LO
moments are defined differently for LLD or LD, resulting in equations that are
consistent.  For instance, for lumped LD, the LO system uses the spatial closure that
the edge value is defined as the moment, i.e., $\mom{\phi}_{R,i} \equiv \phi_{R,i}$.
For a standard lumped source, we desire the right equation to have $\mom{q}_{R,i} =
\sigma_s(\phi^{l+1/2}_{R,i} - \phi^{l}_{R,i})$.  Substituting the lumped closure into
the right hand side of this equation gives back the original equation, i.e., $\mom{q}_{R,i} = \sigma_{s,i}\left(\mom{\phi^{l+1/2}}_{R,i} -
    \mom{\phi^{l}}_{R,i}\right) $.  The same is true for standard LD.

\subsection{Updating the LO Unkowns}

We now have a correction to $J$ and $\phi$ for the volumetric finite element
unknowns.
Because we are interested in the time-dependent solution, we need to accelerate the solution for the
half-range fluxes, rather than just the scalar flux. We only accelerate the zeroth
moment of the angular intensity.  The error in the scalar intensities are defined as
\begin{equation}\label{eq:p1psi}
    \delta \phi^{\pm} = \frac{\delta \phi}{2} \pm \frac{3 \delta J}{4}
\end{equation}
Spatial moments are taken of $\delta \phi^{\pm}$, using the lumping notation of LD on the interior
\begin{align}
    \mom{\delta \phi^{\pm}}_L =&  \frac{1+\theta}{2}\delta \phi^{\pm}_{L} +
    \frac{1-\theta}{2}\delta \phi^{\pm}_{R}  \\
    \mom{\delta \phi^{\pm}}_R =&  \frac{1-\theta}{2}\delta \phi^{\pm}_{L} +
    \frac{1+\theta}{2}\delta \phi^{\pm}_{R}      ,
\end{align}
where Eq.~\eqref{eq:p1psi} can be used to get in terms of $\delta \phi_{L/R}$ and $\delta J_{L/R}$.

\subsection{The WLA-DSA Accelerated Source Iteration Algorithm}

We define the process of solving the diffusion like equations and mapping the error
uknowns back onto the moment equations as the operator $D^{-1}$.

The source iteration with linear DSA procedure, for the $m$-th Newton iteration, is defined as
\begin{enumerate}
    \item Evaluate effective scattering and absorption cross sections with
        ${\{T^m_i:\;\, i=1,2,\ldots,N_c\}}$.
    \item\label{en:dsa_beg} Compute new scattering source $\B S \psi^l$.
    \item Perform sweeps to calculate $\psi^{l+1/2} = \B M^{-1} \B S\psi^{l} + \B M^{-1} Q$
    \item Compute DSA residual source $\sigma_s(\phi^{l+1/2}-\phi^l)$
    \item Solve continuous DSA equations (i.e., Eq.~\eqref{eq:dsa_lumped}) for ${\{\delta \phi^{l+1/2}_{i+1/2}:\;i=0,1,\ldots,N_c\}}$
    \item Map the continuous error onto the moment unknowns.
    \item\label{en:dsa_end} If $\|\psi^{l+1} - \psi^{l} \| < $ tolerance, exit
    \item Else: repeat steps~\ref{en:dsa_beg}--\ref{en:dsa_end}.
\end{enumerate}

\section{GMRES Solution to the LO Equations}

The source iteration procedure can be recast as an iterative solution to a linear maxtrix
system.  GMRES is an effective iterative solution method for asymmetric, sparse matrix
equations~\cite{saad}.  Using the operators that have already been defined, we manipulate
the source iteration equations to form a procedure for the Krylov solution approach:
\begin{equation}
    \left(\B I  - \B M^{-1} S\right) \psi = \B M^{-1} Q,
\end{equation}
where $I$ is an identity matrix. We define combination of operators in parenthesis on the left side
of the above equation as $A$, producing a matrix equation.  Rather than form the
full matrix system, we can apply the $\B S$ and $\B M^{-1}$ operators as defined in the
previous sections.  The GMRES solution to these equations thus applies an iterative
solution technique to the above equations by repeatedly apply the operator $A$ to krylov
vectors.

To formulate the preconditioned GMRES method, we note that we want the preconditioner to approximate
the inverse of the operator on the left side of the above equation. This can be thought of
as a form of physics-based preconditioning.  Left preconditioning
was applied to the above system~\cite{saad}.  Thus in matrix form, we write the preconditioned GMRES equations as
\begin{equation}
    \left(\B I + \B D^{-1} S\right)\left(\B I - \B L^{-1} \B S \right) \psi = \left( \B I + \B
    D^{-1}\right) \B L^{-1} Q
\end{equation}
The opensource library \verb{mgmres{ is used to produce the Krylov vectors.  The
    infrastructure from the source iteration with DSA procedure can be reused to provide
    the operation of $A$ applied to the Krylov vectors returned from the GMRES solve.

Some modifications need to be made to solve these equations explicilty.  Thus, the total
algorithm for the GMRES solution is
\begin{enumerate}
    \item Evaluate effective scattering and absorption cross sections with
        ${\{T^m_i:\;\, i=1,2,\ldots,N_c\}}$.
    \item\label{en:dsa_beg} Compute new scattering source $\B S \psi^l$.
    \item Perform sweeps to calculate $\psi^{l+1/2} = \B M^{-1} \B S\psi^{l} + \B M^{-1} Q$
    \item Compute DSA residual source $\sigma_s(\phi^{l+1/2}-\phi^l)$
    \item Solve continuous DSA equations (i.e., Eq.~\eqref{eq:dsa_lumped}) for ${\{\delta \phi^{l+1/2}_{i+1/2}:\;i=0,1,\ldots,N_c\}}$
    \item Map the continuous error onto the moment unknowns.
    \item\label{en:dsa_end} If $\|\psi^{l+1} - \psi^{l} \| < $ tolerance, exit
    \item Else: repeat steps~\ref{en:dsa_beg}--\ref{en:dsa_end}.
\end{enumerate}
Without preconditioning, the diffusion solve is simply removed.


\section{Computational Results}

It is noted we are not interested in measuring the reduction of
computational time in this section because in 1D the LO equations can be directly solved efficiently.
We are just interested in ensuring that the acceleration methods can reduce the
number of scattering iterations sufficiently, including cases where inconsistencies
in the LO equations are present.

We test our acceleration methods with one simple test
problem.  This problem is a modification of the two material problem described in
Sec.~\ref{sec:two_mat}. The problem specifications are the same as before, except for
modifications to the material properties for $x>0.5$ cm.  

\subsection{Results for LD Spatial Discretization}

We first test this problem with the source iteration using DSA to accelerate and compare
to an unaccelerated SI solution.  3 batches of 10,000 particles are ran for each HO
solve, one HO solver per time step.  We tally the total number of source iterations per
time step, over the two solves.  We initialize the solution for the scattering iterations to zero at the beginning of
each LO solve. 







