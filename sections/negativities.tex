
\section{Resolving Negative Intensities in Optically Thick Cells}

On difficult issue to overcome in thermal radiative transfer problems is 
in the optically thick regions of the problem.  The LDFE representation in the HO solver
becomes negative, and in the LO equations the values can also be negative.


\subsection{Resolving Issues in Optically Thick Cells}
\label{sec:negs}

The linear-discontinuous (LD) closure with upwinding is not strictly positive.  In particular, for
optically thick cells with a steep intensity gradient, the intensity becomes
negative. In typical TRT problems (e.g., the Marshak wave problems above), this negativity occurs at the wave-front of the
radiation intensity in optically thick materials.
These negativities are not physical and can propagate to adjacent cells. In thick regions of
TRT problems, reasonably fine spatial cells can still be on the order of millions of mean
free paths; negativities with an LD representation are unavoidable in practice for
such cells, and mesh refinement is of minimal use.  We will explore several methods
for resolving negativities.  Ideally the solutions in
such cells should be as consistent as possible for the HO and LO equations.  However,
the differences between the solution methods of the two equations, as well as the
fact that some terms are lagged, have lead to the development of independent
approaches for the LO and HO systems thus far.

Typically, for a standard LDFE method,
the equations are lumped to produce a strictly positive solution (for 1D)~\cite{morel_newton}. However, standard FE lumping
procedures would introduce difficulties in computing the consistency terms from the
HO solution.    The results in Sec.~\ref{results} used a lumping-equivalent closure for the LO equations and $S_2$ equivalent angular terms..
This approach will not produce
accurate results for general problems in higher spatial dimensions. Also, a lumping equivalent closure in $x$ does not guarantee positivity for the space-angle LDFE
representation of the HO intensity.
Alternatively, the equations within a cell can be modified to ensure the outflow is not
below the floor value (the initial temperature of the problem), and energy balance is
conserved.  The LD shape is reconstructed by extrapolating from the outflow back
through the average.  This closure has been implemented and is more promising for extendability into higher
dimensions.

For the HO solver, after an ECMC batch, we detect cells that produced a negative
intensity. In these cells, we scale the linear representation of the intensity (in
$x$ and $\mu$) to be greater than the floor
value.  This scaling process of the two moments is underdefined, so there is not a unique way to enforce
positivity.  The scaling procedure is not an emphasis of the research, so we apply the simple
approach of scaling the slopes such that the ratio of the slope in $x$ and $\mu$ is
unchanged.  The scaled intensity will not satisfy the original residual equation accurately because we have
modified the first moments.   Thus, the ECMC error estimates will rapidly stagnate,
and produce relatively inaccurate solutions.  This scaling can also lead to
negativities in down-stream cells that is unphysical in later batches.  To mitigate
stagnation and improve accuracy, we will add an artifical source $\tilde\delta^{m+1}(x,\mu)$ to the HO equation.
This source is estimated iteratively as
\begin{equation*}
    \tilde\delta(x,\mu)^{(m+1)} = \mathbf{L}(\tilde{I}^{n+1,(m)} -
    \tilde{I}^{n+1,(m)}_{\text{pos}}),
\end{equation*}
where $\tilde{I}_{pos}^{n+1}$ is the modified positive solution.
Essentially, this source is modifying the first moments of the transport equation to
ensure a positive solution within the negative cells.   If necessary, we can estimate a
new source again in the next batch.  The source
$\tilde\delta$ lies in the same functional space as the residual, and thus can use
the existing code for computing the residual and will be straight forward to extend
to higher dimensions.  

We have implemented this artificial source and found it to reduce the
magnitude of the MC estimated error, producing a positive solution. We still need to
investigate the accuracy and robustness of the
added source method, and whether it should be recomputed every batch, once it has been
triggered, or only when the
solution becomes negative again.   We will also
investigate adding the source to the LO equations for greater consistency and accuracy in
the LO equations.  This will not
necessarily guarantee positivity in the LO equations due to the lagged estimate of the emission source in
used in the previous HO solution.  We will investigate altering the  method for computing the added source computation to ensure the zeroth moment
of the equations is not changed, ensuring energy conservation.
