

\chapter{Resolving Negative Intensities in Optically Thick Cells}

The linear-discontinuous (LD) spatial closure with upwinding is not
strictly positive.  In particular, for optically thick cells with a steep intensity
gradient, the linear representation of the intensity can become negative at the edge of the cells.
A common example in 1D is for the Marshak Wave
problem where negative
intensities in the representation occurs at the foot of the radiation wave front. These negativities are not physical and typically propagate to
adjacent cells. In thick regions of
TRT problems, reasonably fine spatial cells can still be on the order of millions of mean
free paths; negativities with an LD representation are unavoidable in practice for
such cells, and mesh refinement is of minimal use.
The HO solver is prone to additional negativities near $\mu=0$ where the intensity
cannot be accurately represented by a linear projection in angle.  These negativities near
$\mu=0$ can occur for modest optical thicknesses and in multiple adjacent cells.
Because of the different solution methods for the HO and LO solvers, indepedent fixups
have been developed for each. 
In the remainder of this chapter, we discuss the fixup methods for each the HO and LO
solver.  Methods are then compared for statistical efficiency and accuracy for several test problems.
      We will explore several methods
for resolving negativities.  Ideally the solutions in
such cells should be as consistent as possible for the HO and LO equations.  However,
the differences between the solution methods of the two equations, as well as the
fact that the modifications made to one solver would be lagged in the next nonlinear LO solve, there
is no guarantee of positivity.  

\section{Fixup for Negative Intensities in the LO Equations}
\label{sec:negs}

In LO equations, the linear representation for $\phi(x)$ can go below the floor
temperature or negative on the interior of the cell.
The floor temperature is defined as the initial
temperature of the material and radiation in problems where boundary sources are
applied at each of the boundaries.  In such problems the radiation and material should
continue to heat on the interior of the domain, and should physically not fall below
the initial temperature. 

Typically, for a standard LDFE Galerkin spatial discretization,
the equations are lumped to produce a discretization that is strongly resistant to
negative values (for 1D)~\cite{morel_ldtrt}. However, standard FE lumping
procedures would introduce difficulties in computing the consistency terms from the
HO solution.  Alternatively, we have derived a modified spatial closure that produces
unknowns equivalent to those from a lumped LD method in 1D.  The modified spatial closure
is
\begin{align}
    \phi_{i+1/2}^+ &= \mom{\phi}_{i,R}^+ \\
    \phi_{i-1/2}^- &= \mom{\phi}_{i,L}^- 
\end{align}
The system is fully defined with upwinding and the assunmption of a linear relationship on
interior of the element.  This modified closure produces a linear
representation that preserves the zeroth moment of the solution, but has modified the
first spatial moment.

We also investigated an alternative closure of the equations based on energy conservation.  
The equations within cells that produce a negativity are modified to ensure the edge
intensities are not below the floor temperature, and energy balance is
conserved.  This fixup is only applied in cells where a intensity has occured during
inversion of the LO streaming plus removal operator.  In the modified equation, the $L$
and $R$ moment equations are summed to produce a
balance equation.  The auxillary equation is then defined by defining the closure
relation such that the appropriate edge
value is the floor intensity.  Because our solution method directly inverts the LO system,
negative edge intensities must be detected, the fixup applied, and then that Newton solve
repeated.

In practice, this approach was observed to produce a positive answer, but was not as robust as performing
lumping in all cells.  In general, as the time step size is increased, this fixup led to the Newton
solve diverging (i.e., damping is required to converge the iterations), more rapidly than if lumping is
just used in all cells.  A similar effect was observed when attempting to only lump the
equations in cells where negativities were observed and restarting that Newton solve.

\subsection{Fixup for the Linear Doubly-Discontinuous Trial Space}

The doubly discontinuous trial space presents an additional difficulty in that the outflow
is now unhinged from the linear relationship.  For this case, we use the lumped
representation on the interior of the solution for all cells. The outflow can still be driven negative
due to non-linearities, which leads to negative values in down-stream cells. 
This is a result of the HO estimation of the spatial closure using
and the moments was based on lagged source terms.  When negativities occur, we force the
outflow to be continous, using the lumping-equivalent LD closure in those cells.  The
Newton solve must then be resolved.


\section{Artificial Source Method for Negativities in the HO Intensity}

The HO solver requires a different fixup approach for negative values of the intensity.
At the end of any particular batch, a LDFE projection of the intensity $\tilde I(x,\mu)$
has been determined.  This projection is based on a statistical estimate of moments of the
exact intensity, based on the truncated representation of sources.  However, when these moments are
projected onto a linear space, the representation becomes negative.  The first
moments can easily be modified to produce a positive representation $\tilde I_{\pos}$.
However, this modified solution will
not satisfy the residual equation as accurately as the original solution, which leads to
rapid error stagnation.  Additionally, the next MC batch based on the residual source from
$\tilde I_{\pos}$ will tend to produce negative cell averages in down stream cells.

Thus, we have devised
a method to modify the transport equation such that $\tilde I_{\pos}$ will satisify the
residual equation more accurately.  We do this in such a manner that the
modified source will lead to the solution converging towards a solution with the same
zeroth moment, but with a first moment in $x$ and $\mu$ that are modified.  This does not
guarantee exponential convergence in this cell, because convergence is still limited by
the overall accuracy of the trial space.  However, now the error will not stagnate as rapidly
and the solution will converge towards the positive representation $\tilde I_{\pos}$.

\subsection{Calculating a Positive LDFE Representation}

First, we define the procedure for obtaining $\tilde I_{\pos}$. We produce a positive
representation $\tilde I_{\text{\pos}}$ over a cell by scaling the first moment
in $x$ and $\mu$ uniformly.  The process of modifying
the first moments to produce a positive solution is under defined, so there is not a unique way to enforce
positivity.  This choice is not an emphasis of this research, so we apply the
simple and efficient approach of scaling the slopes such that the ratio $I_x/I_\mu$, for
each modified cell, is unchanged. 
After an ECMC batch, we detect cells where the linear
representation produces a value below the floor.    The modified representation for the
$ij$-th cell in such cells is
\begin{equation}
    \tilde I_{\pos} = I_{a} + C\left[\frac{2}{h_x}I_x(x-x_i) +
    \frac{2}{h_\mu}I_\mu(\mu-\mu_j)\right],\quad     (x,\mu) \in \mathcal{D}_{ij},
\end{equation}
noting that the average has not been modified.
The constant $C$ is calculated as
\begin{equation}
    C =  \frac{I_{a} - I_{\min}}{\lvert I_x \rvert + \lvert I_\mu \rvert}
\end{equation}
for values where $I_{a} > I_{\min}$, where $I_{\min}$ is the isotropic intensity
corresponding to equilibrium with the floor temperature.  When $I_a$ is below the floor, it is set to
the floor value and $I_x$ and $I_\mu$ set to zero.  It is been noted that in application
the difference between $I_{a}$ and $I_{\min}$ can be on the order of numerical roundoff for
double precision variables.

\subsection{Adding an Artificial Source}

To mitigate stagnation and improve accuracy, we must now add an artificial source
$\tilde\delta^{m+1}(x,\mu)$ to the HO transport equation to 
This source is estimated iteratively as
\begin{equation*}
    \tilde\delta(x,\mu)^{(m+1)} = \mathbf{L}(\tilde{I}^{n+1,(m)} -
    \tilde{I}^{n+1,(m)}_{\text{pos}}),
\end{equation*}
where $\tilde{I}_{pos}^{n+1}$ is the modified positive solution. The source $\tilde
\delta$ is added to all later batches.  If necessary, we can estimate a
new source again in later batches where negative values occur once more. The residual for the
modified transport problem will have the same residual magnitude as the original $\tilde
I$, which will have lower magnitude than the modified solution which does not have the MC
estimated first moments (this is only true for the first application of the modified
source).  Care must be
taken to modify the source on the interior and exterior of the solution, particularly when
the solutions in adjacent cells has been modified.  The source
$\tilde\delta$ lies in the same functional space as the residual and can thus use
the existing code infrastructure to compute the source.  This will also make this approach
straight forward to extend to higher dimensions.  

To provide insight into this choice of source, consider the modified transport problem
that will be solved with ECMC, where the fixup has been applied at batch $m$:
\begin{equation}
   \B L I^{n+1} = q + \mathbf{L}(\tilde{I}^{n+1,(m)} -
    \tilde{I}^{n+1,(m)}_{\text{pos}})
\end{equation}
Application of $\B L^{-1}$ to both sides of the equation produces
\begin{equation}
    I^{n+1} = \B L^{-1} q + (\tilde{I}^{n+1,(m)} -
    \tilde{I}^{n+1,(m)}_{\text{pos}}).
\end{equation}
Because $\tilde I$ and $\tilde I_\pos$ have the same zeroth moment, we have not modified
the zeroth moment of the solution overall.  Monte Carlo transport is used to estimate $L^{-1}$, thus 
we are estimating the solution to a transport problem that has a positive LDFE projection but preserves the
zeroth moment of the original solution.  The estimate of the modification to the first
moments of the solution has statistical noise, and thus may under- or over-predict the
necessary change in the solution.  We make the conservative choice of preserving $\delta$
across batches, and modifying the source only when negative values occur again. 


