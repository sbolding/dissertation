
%\subsection{Alternative time discretization}
%\label{sec:time}
%
%A final area of research to be investigated is in an alternative time discretization
%of the system.  This is the least developed portion of the research.  The goal of
%an alternative time discretization is to produce a more accurate solution in
%optically thin regions where particles transport a long distance.  This is an area
%where the MC treatment of the time-variable by IMC can produce greater accuracy,
%whereas a BE discretization will result in artificially fast propagation of energy.
%This may be important in applications such as stellar atmosphere calculations.  
%
%First, the ECMC
%solution is considered. The emission source
%is still treated as fully implicit, however the intensity is allowed to be continuous
%in time.  In inverting the $\B L$ operator, particles are born with some specific
%time, and their time is tracked until they reach the end of the time step.  Tallies are adjusted
%to account for the averaging over the time step.
%in terms of MC, letting . As with the spatial and angular variables, we assume a trial
%space for the time variable as well.  Two different options are considered.
%The first, is a doubly-discontinous treatement, where teh outflow in time is solved
%for with the MC.  Fig.~\ref{fig:dd_time} demonstrates the basic principle of this
%closure in time.  The second options is a linear discontinuous trial space in time.
%The first approach requires typical census type tallies~\cite{fig}, which adds
%significant complexity in terms of tallies.  The second will
%require the first moment in time to be tallied.  
%Again, ECMC will be computing the projection of the exact solution onto the trial
%space, so this will have the accuracy of IMC for the radiation.    Neither of these cases require the
%addition of Adds one extra set of unknowns to be stored. 
%
%Next, consistency with the LO system is
%necessary.  The goal is to not add additional equations to solve for the
%time-dependent unknowns, but to   Previous work has attempted to simply subtract the continuous HO solution
%from a BE discretziation of the discretized time-derivatives to add an artificial
%term, with the addition of extra terms from hydrodynamics~\cite{holo_rh}.  This has the added benefit that the LO solver exclusively deals in
%time-averaged unknowns.  As an alternative, we will estimate a parameter based on the
%ratio of the outflow in time to the average over the time step. One possible closure
%is
%\begin{equation}
%    I^{n+1} = 2\gamma_t \overline{I} - I^{n}
%\end{equation}
%where $\gamma_t$ is the closure factor and $\overline{I}$ is the time-averaged
%intensity.  A spatial and angle discretized version of the above equation can be used to
%eliminate the extra unknowns from the LO system.  The value of $\gamma_t$ is
%estimated by solving the above equation based on the latest HO solution for all
%parameters.  The stability and accuracy of these methods will need
%to be investigated. As a first step, a truly BE time discretization in the HO can be
%used in conjunction with the LO solver.

%
%  There
%is some difficulty in making the LO time-averaged quantities match with the HO
%solution, which may require some approximation.  However, this is not that different
%than the discrete case because the representation of $\tilde I^{n}$ is lagged an
%iteration anyways. 
%It is noted that the SIMC method explored 
%some similar alternate time discretizations for the temperature in the residual
%treatment, but second-order accurate time treatment.  However, the second-order approximation
