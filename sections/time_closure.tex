
\chapter{ \uppercase {Residual Monte Carlo Treatment of the Time Variable}}
\label{sec:time}
Another area of potential improvement for the HOLO method is in the time discretization
of the system.  We have investiaged using MC integration of the time derivative in the HO
solver and introducing extra consistency terms into the LO equations.  The goal is to produce a more accurate solution in optically
thin regions where particles transport a long distance.  A potential application is in 
stellar atmosphere calculations.  In optically thin regions, the MC
integration of the time-variable by IMC can produce greater accuracy, whereas an implicit
Euler discretization will result in artificially fast propagation of energy.  We hope to improve
the efficiency of MC calculations in thin regions with the ECMC method while still preserving the
accuracy of a MC treatment in the time variable.  The time variable is included in
the trial space for ECMC and the LO equations are closed consistently.  We
will not be performing adaptive refinement in time, so maintaining exponential convergence
may not be possible.  However, we still expect the residual MC formulation of the ECMC method
to show improvement over standard MC.

\section{Modifications to the HO equations}
\label{sec:time_ho}

Inclusion of the time variable $t$ in the trial space used by ECMC allows for no discretization of the
transport operator $\B L$.  The transport operator, applied to the continues intensity $I$
becomes,
\begin{equation}
   \B LI(x,\mu,t) = \frac{1}{c}\pderiv{I}{t} + \sigma_t I + \mu \pderiv{I}{x}
\end{equation}
The emission source is still treated with an implicit Euler discretization, which is
similar to the approximation made in IMC.  The ECMC algorithm is still specified the same as in Sec.~\ref{sec:???}, however the residual
source and MC tracking are now modified.   Each batch is still estimating the error in the
current estimate of $\tilde I(x,\mu,t)$, but now the time variable must be included in the inversion of the $\B L$
operator.  The process of sampling and tracking in time
step is detailed in literature\cite{wollaber_review,fnc,wollaber_thesis,cj_thesis}, but a
brief outline is given in the remainder of this section, as well as the definition for the
trial space and associated tallies. 

\subsection{The Doubly-Discontinuous Trial Space}

It is necessary to define the time trial space so that we can explicitly define the
residual for sampling.  This trial space is similar to the LDD trial space used for the space variable
in Sec.~\ref{sec:lo_closure_LDDTRIAL???}, however the solution is a constant value over
the interior of the time step. This step, doubly-discontinuous trial space is defined as
\begin{equation}\label{eq:time_space}
    \tilde I(x,\mu,t) = \left \{ \begin{array}{cl}
        \tilde I^{n}(x,\mu)  & \quad t = t^n \\ 
        \overline I(x,\mu)  & \quad t \in (t^{n},t^{n+1}) \\               
        &  \quad        t = t^{n+1}
    \end{array}           \right.
\end{equation}
where we have used $\overline I$ to denote the time-averaged value of the intensity over
the interior of the time step and $\tilde I(x,\mu)$ is an LDFE projection in $x$ and
$\mu$.  An illustration of the time variable for this trial space
is depicted in Fig.~\ref{fig:dd_time}.    There is a projection error in that we have used the LDFE space-angle
projection to represent the intensity from the previous time step.  However, with
sufficient noise reduction and mesh resolution this should not be a large error as
compared to IMC.
\begin{figure}[H]
    \centering
    \begin{center}
%        \resizebox{0.4\textwidth}{!}{
        \begin{tikzpicture}[scale=0.882, every node/.style={transform shape}]
            \draw (1.0,4.0) node[fill,circle,inner sep=0pt,minimum
            size=4.2pt] {};
            \draw [->] (1.6,4.25) -- (2.4,4.25) node[anchor=west] {$t$};
            \draw (1.0,0.4) -- (1.0,0.6) node[below, pos=0.4] {$t^{n}$};
            \draw (5.90,0.4) -- (5.90,0.6) node[below, pos=0.4] {$t^{n+1}$};
            \node at (3.6,3.06) {$\overline{I}_{HO}(x,\mu)$};
            \draw [thick] (1.0,0.5) -- (5.9,0.5) node[anchor=north west] {};
            \filldraw[color=black, fill=white] (1,2.450) circle (2.1pt);
            \draw (1.0,2.45) -- (5.90,2.45);
            \filldraw[color=black, fill=white] (5.9,2.450) circle (2.1pt);
            \draw (5.9,1.6) node[black,fill,circle,inner sep=0pt,minimum size=4.2pt] {};
            \node[anchor=west] at (5.9,1.6) {${\tilde{I}_{HO}^{n+1}}$};
        \end{tikzpicture}
 %   }
    \end{center}
    \caption{Step doubly-discontinuous representation of $t$ for the HO solution.}
    \label{fig:dd_time}
\end{figure}

The choice of this trial space will provide a projection for all the desired unknowns to exactly produce the moment
equations, i.e., the time-averaged, end of time step, and previous time step values for
the intensity.  Temporally, these are the only unknowns that appear in equations that have
been integrated over a time step to produce a balance statement.  Another reason for this
choice is that it allows for infrastructure for computing the residual from the
time-discrete case.  This trial space has one major drawback, in that some particles must
reach the end of the time step, which can lead to poor statistics in optically thick
problems.  This is troubling, because in such problems you don't need much correction in
the time variable.  Alternatively, an LDFE representation could be used in the time
variable. The linear representation would produce less noise because all particle tracks
contribute to the slope, rather than just those that reach the end of the time step,
although it would produce an approximate projection error for the end of time step
intensity that is not produced with a discontinuity at the end of the time step.  The
linear representation in time would also produce a more accurate reconstruction of the
scattering source in time.  However, a linear representation requires the sampling
algorithm to be significantly modified because the L$_1$ integral for computing the
residual magnitude is now significantly complicated by the tri-linear function.  A
possible way to sample this source is discussed in Appendix for completeness, but it has
not been rigorously investigated.

\subsection{Residual Source Definition and Sampling}

The residual is defined as $r = q^{n+1} - \B L \tilde I(x,\mu,t)$, where
\begin{equation}
    q^{n+1}=\left(\sigma_a a c (T_{LO}^{n+1})^4(x) + \sigma_s\overline\phi\right)
\end{equation}
is a constant in time. We have assumed a constant reconstruction for the scattering source
in time.  If the system is scattering dominated, this may be a poor approximation. 
Substitution of Eq.~\eqref{eq:time_space} for $I$
produces a uniform source in time, as well as a $\delta$-function source at the
beginning and end of the time step, which is written as
\begin{equation}
    r(x,\mu,t) = \overline r(x,\mu)  + r^{n}(x,\mu)\delta^+(t-t^{n}) +
    r^{n+1}(x,\mu)\delta^-(t - t^{n+1}),
    \quad t\in[t^{n},t^{n+1}]
\end{equation}

We will look at each component individually.  
The first residual term is a constant in time with representation
\begin{equation}
    \overline r(x,\mu) = q  - \mu \pderiv{\overline I(x,\mu)}{x} - \sigma_t \overline
    I(x,\mu)
\end{equation}
Evaluation of the above function produces both face and volumetric sources, similar to in
the discrete case.  To sample $x$ and $\mu$ from the face and volume distributions, the same rejection procedure
can be used as for Eq.~\eqref{eq:???} and detailed in~\cite{jake}.    The
time variable can then be sampled uniformly over the time step, i.e., $t=t^n + \eta \Delta
t$, where $\eta$ is a uniform random variable with support $(0,1)$.

The second source has definition
\begin{equation}
    r^{n}(x,\mu) = -\frac{1}{c}\pderiv{\overline{I}(x,\mu)}{t}\bigg|_{t=t^{n}}
    =-\frac{1}{c}\left(\overline I(x,\mu) - \tilde I^{n}(x,\mu)\right)
\end{equation}
This source is a LDFE space and angle volumetric source.
The rejection sampling procedure is used to sample $x$ and $\mu$.
All particles sampled from this source begin tracking with $t=t^{n}$.

The final source term is
\begin{equation}
    r^{n+1}(x,\mu) = -\frac{1}{c}\pderiv{\overline{I}(x,\mu)}{t}\bigg|_{t=t^{n+1}}
    =-\frac{1}{c}\left(\tilde I^{n+1}(x,\mu) - \overline I(x,\mu)\right).
\end{equation}
The source $r^{n+1}$ can be treated using the
same analytic treatment as the outflow face source in the LDD
trial space, detailed in Sec.~\ref{sec:???}; the source at the end of the time step is never sampled because its
contribution to $I^{n+1}$ can be analytically computed.  To treat the sources this way, the solution for $\tilde I^{n+1}(x,\mu)$ is
initialized to the value of $\overline I(x,\mu)$ before a batch of particles begins.
Then, error particles that reach the end of the time step, referred to as ``census''
particles, contribute a standard score to the projection $\tilde I^{n+1}(x,\mu)$.

With these definitions, it is thus only necessary to sample from the two sources.  Using
composite-rejection sampling~\cite{shultis_mc}.  A discrete probability distribution is
sampled to determine which source is sampled.  The algorithm is
\begin{enumerate}
    \item Sample uniform random number $\eta$
    \item If $\eta < \|r^{n}\|_1/(\|r^{n}\|_1 + \|r^{n+1}\|_1)$:
    \begin{itemize}
        \item Sample from $r^{n}$ source using rejection sampling
        \item Sample $t$ uniformly over $(t^{n},t^{n+1})$.
    \end{itemize}
\item Else:
    \begin{itemize}
        \item Sample from $\overline r$ source
    \end{itemize}
\end{enumerate}
All L$_1$ integrals can be analytically integrated using the same numerics as in the
time-discrete case.  Using the systematic sampling algorithm, as described in
Sec.~\ref{sec:???} is performed similarly.  However, the choice of source is only make
locally over that space-angle element. In that case, the element is chosen systematically,
then the choice of $r^{n}$ or $\overline r$ is made.

\subsection{Importance Sampling on Interior of Time Step}

As an attempt to reduce variance in the estimate of $\tilde \epsilon^{n+1}(x,\mu)$, we use
important sampling in the time variable.  Systematic sampling is still used for
determining the cell of interest, and sampling as described above is used to determine
which source is sampled, based on the appropriate probabilities described in the previous
section.  However, when the interior source $\overline r(x,\mu)$ is sampled, we use
importance sampling for the conditional sampling of the uniform time step.  The goal is to ensure that some
histories reach the end of the time step.  In order to do this, we sample from a modified
PDF such that a fraction $p_{end}$ of particles sampled from $\overline r(x,\mu)$ are born
with $t\in(t^{end},t^{n+1})$.  We define $t^{end}=M/(c\sigma_t)$, where $M$ is the desired
number of MFP of travel the particle will undergo from the end of
the time step (e.g., 2 or 3).  The weights of particles sampled from this
distribution must be modified to prevent source biasing.

The new PDF to be sampled from is
\begin{equation}
    f^*(t) = \left\{ \begin{array}{cl}
        (1 - p_{end}) &\quad 0 < t < t^{end} \\ 
        p_{end} & \quad t^{end} \leq t < t^{n+1}  \\
        0 & \quad \text{elsewhere}\end{array}  \right.
\end{equation}
The original PDF is $f(t)= 1/\Delta t$, for $t\in(t^{n},t^{n+1})$.  Thus, using the
standard procedure for importance sampling\cite{shultis_mc}, the starting time $t_{\text{start}}$ is sampled from
$f^*(t)$, and then weights are multiplied by the factor
$f(t_{\text{start}})/f*(t_{\text{start}})$.  This procedure is not perfect in that if a
particle is moving from an optically thin to an optically thick
region, it is not guaranteed to reach census. However, this case does not introduce bias.

\subsection{Tracking and Tallying in Time}

Because our LO equations will be integrated over the time step, we only need to
perform MC tracking for $t\in[t^{n},t^{n+1}]$.  
The initial time for the particle is
sampled as described in the previous section. In inverting the $\B L$ operator, particles
are tracked until they reach the end of the time step.  Path lengths are sampled or the
weight is exponentially attenuated as before (e.g., Sec.~\ref{sec:???}).  As a particle
travels from position $x_{o}$ to $x_{f}$, with direction $\mu$, the time is updated as 
\begin{equation}
    t^{f} = t^{0} + \frac{|x_{f} - x_{o}|}{c \mu}
\end{equation}
where $c$ is the speed of light. For analog path-length sampling, if $t^{f}>t^{n+1}$ then $t^{f}$ is adjusted to $t^{n+1}$
and the path length is adjusted accordingly.  For continuous weight deposition, particles
are only tracked until they reach $t^{n+1}$.  A proof that this process of tracking
particles is a MC solution to an integral equation that is exactly inverse to the $\B L$ operator is
detailed in~\cite{cj_thesis,shultis_paper_cj_cites???}.  

Tallies must be adjusted to account for the averaging over the time step, and to compute the
intensity at the end of time step.  To produced the time-averaged representation
$\overline I(x,\mu)$, requires estimators for the average, $x$, and $\mu$ moments of the
error, e.g.,
\begin{equation}
    \overline\epsilon_{x,ij} = \frac{1}{\Delta t} \frac{6}{h_j}
    \int\limits_{t^{n}}^{t^{n+1}} \!\!\dd t \!\!\!
    \int\limits_{\xl}^{\xr} \!\!\!\!\! \dd x
    \hspace{-0.081in}\int\limits_{\mu_{j-1/2}}^{\mu_{j+1/2}} \hspace{-0.107in} \dd \mu
    \;\left(\frac{x - x_j}{h_{i}}\right) \epsilon(x,\mu,t)
\end{equation}
with a similar definition for the average and $\mu$ moments.  In implementation, we have
included the $\frac{1}{\Delta t}$ factor in the definition of the residual source, so
the estimators are
\begin{equation}
    \hat{\overline \epsilon}_{x,ij} =\frac{1}{N_{hist}} \frac{6}{h_j} \sum_{n=1}^{N_{hist}}
    \frac{s_n}{h_{i}h_{j}} w_j \left(x_c - x_i\right)
\end{equation}
where $\sum\limits_{n=1}^{N} w_n = \| r(x,\mu,t) \|_1$, $x_c$ is the center of the $n$-th
pathlength, and $s_{n}$ is the path length for the $n$-th path length in the $x-\mu$ cell.

Moments of $I^{n+1}(x,\mu)$ must be estimated to represent the end of time step intensity.
For example, the moment of the error at the end of time step is
\begin{equation}
    \epsilon^{n+1}_{x,ij} = \frac{6}{h_i} \iint\limits_{\mathcal{D}_{ij}} \left(\frac{x
    - x_i}{h_i}\right) \epsilon(x,\mu,t^{n+1}) \dd x \dd \mu
\end{equation}
The estimators for these moments are a generalization of the census
tallies used in IMC~\cite{wollaber_review,wollaber_thesis}.  The tallies are based on the
definition of the intensity as $I(x,\mu,t) = c h \nu N(x,\mu,t)$ given in
Eq.~\eqref{???}, similar to collision estimators~\cite{shultis_mc,mcnp}.  Based on the 
definition of the residual, the estimator for the $x$ moment is
\begin{equation}
    \hat\epsilon^{n+1}_{x,ij} = \frac{1}{N_{hist}} \frac{6}{h_j h_i} \sum_{n=1}^{N_{hist}}
    c w_j \Delta_t \left(x_{c} - x_{i}\right)
\end{equation}
Similar tallies are defined for the other space-angle moments. These tallies can be
exceptionally noisy because only particles that reach the end of the time step contribute.

REWRITE The factors of delta t actually have to do with the fact that when I compute the
L1 norm of the residual, I dont integrate over time. If i did, then this is all fine.

\section{LO Closure}

The LO equations must be modified to have a closure in time for consistency with the HO
equations. The goal is to not add additional equations to solve for the
time-dependent unknowns, rather than just using closure.   Previous work has enforced consistency by subtract the continuous HO solution
from a BE discretization of the discretized time-derivatives to add an artificial
term~\cite{holo_rh}.  This has the added benefit that the LO solver exclusively deals in
time-averaged unknowns for radiation terms in the equations.  We will alternatively use a
parameteric closure in the time variable, similar to the spatial closures discussed in the
Sec.~\ref{???}.  The time-integrated equations will have primarily time-averaged
values, which is desired. Additionally, the closure
produces LO equations that have the same numerical difficulty to solve as the fully-discrete LO equations, but
have the potential to preserve the accuracy of the MC integration in time, upon non-linear
convergence of the system.   A closure relation is used to eliminate
the end of time step moments present from the time derivative term.   We will investigate different parameteric forms of the closure for robustness.   Once the time-averaged unknowns have been calculated,
the time closures can be used to advance to the end of time step values for the next
time step.

\subsection{Derivation of Time-Averaged Moment Equations}

The time-continuous radiation equations are integrated in space and angle the same as
before.  For example, the $L$ and $+$ moment equation is
\begin{multline}
    \frac{1}{c}   \pderiv{ }{t} \mom{\phi}^+_L - 2\left({\mu}_{i-1/2} I_{i-1/2}\right)^+ + \mom{\mu I}^+_{L,i} 
    + \mom{\mu I}^+_{R,i} +  \sigma_{t,i} h_i \mom{\phi}_{L,i}^{+} -  \frac{\sigma_{s,i} h_i}{2} \left( \mom{\phi}_{L,i}^{+} +
  \mom\phi_{L,i}^{-}\right) \\ = \frac{h_i}{2} \mom{\sigma_a a c T^4}_{L,i} 
\end{multline}
This equation is then integrated over the time step, and the emission source is assumed
implicit.  The same manipulations can be
performed on the streaming term to form angular consistency terms, but the weighting fluxes are now
time-averaged values.  Thus, the angular consistency terms are computed with $\overline I(x,\mu)$.  
The resulting equations are
\begin{multline}
    \frac{\mom{\phi}_L^{+,n+1} - \mom{\phi}_L^{+,n}}{c \Delta t}
    -2\overline {\mu}_{i-1/2}^{\,+} \overline \phi_{i-1/2}^{\,+} + \cur {\mu}_{L,i}^{n+1,+}
  \mom{\phi}_{L,i}^{n+1,+}
  +  \cur\mu_{R,i}^{n+1,+}
  \mom{\phi}_{R,i}^{n+1,+} +  \left(\sigma_{t,i}^{n+1}+\frac{1}{c \Delta t} \right) h_i 
  \mom{\phi}_{L,i}^{n+1,+} \\-  \frac{\sigma_{s,i} h_i}{2} \left( \mom{\phi}_{L,i}^{n+1,+} +
  \mom\phi_{L,i}^{n+1,-}\right) = \frac{h_i}{2} \mom{\sigma_a^{n+1} a c T^{n+1,4}}_{L,i} +
  \frac{h_i}{c\Delta t}\mom{\phi}_{L,i}^{n,+},
\end{multline}
These equations are exact at this point, written in terms of primarily time-averaged unknowns

\subsection{Closure}
%\subsection{Alternative time discretization}
%\label{sec:time}
%

The closure relations in time are different than the closure relations for the spatial
variable because we do not have a time slope.  The following closure is a modified diamond
relation:
\begin{equation}
    I^{n+1} = 2\gamma_t \overline{I} - I^{n}
\end{equation}
where $\gamma_t$ is the closure factor and $\overline{I}$ is the time-averaged
intensity.  A spatial and angle discretized version of the above equation can be used to
eliminate the extra unknowns from the LO system.  The value of $\gamma_t$ is
estimated by solving the above equation based on the latest HO solution for all
parameters. 

At the end of a time step, we need to advance the LO solver 

