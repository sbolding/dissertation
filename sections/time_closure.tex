
\subsection{Inclusion of the Time Variable in ECMC Trial Space}

\label{sec:time}
Another area of investigation for closing the LO equations is in the time discretization
of the system.  The goal of
an alternative time discretization is to produce a more accurate solution in optically
thin regions where particles transport a long distance.  A potential application is in 
stellar atmosphere calculations.  In optically thin regions, the MC
integration of the time-variable by IMC can produce greater accuracy, whereas an implicit
Euler discretization will result in artificially fast propagation of energy.  We hope to improve
the efficiency of MC calculations in thin regions with the ECMC method while still preserving the
accuracy of a MC treatment in the time variable.  The time variable will be included in
the trial space for ECMC and the LO equations will be closed in time consistently.  We
will not be performing adaptive refinement in time, so maintaining exponential convergence
may not be not possible.  However, we still expect the residual MC formulation of the ECMC method
to show improvement over standard MC.

Inclusion of time $t$ in the MC trial space allows for no discretization of the
transport operator $L$.  The transport operator becomes
\begin{equation}
    LI = \frac{1}{c}\pderiv{I}{t} + \sigma_t I + \mu \pderiv{I}{x}
\end{equation}
The emission source
is still treated as fully implicit, similar to IMC.
  In inverting the $\B L$ operator, particles are born with some specific
time, and their time is tracked until they reach the end of the time step.  Tallies are adjusted
to account for the averaging over the time step, and to compute the intensity at the end
of time step as necessary, depending on the chosen trial space.

The choice of trial space in time for the intensity will also be investigated.  A doubly-discontinuous
trial space that is constant over the interior of the time step allows for much of the
residual machinery from the discrete case to be reused.  However, this trial space may produce problems in estimating the end of
time-step intensity if few particles reach the end of time step time.  We will explore the
option of using importance sampling in the time variable to ensure some fraction of
histories reach the end of the time step.  If there is too much statistical noise in the
end of time step values, a
linear-discontinuous (LD) treatment in time could be implemented, which would make all
particle tracks contribute to the estimation of the slope in $t$.  The LD time
representation has a couple downsides.  This approach has a projection error for computing
the end of time step intensity.  Additionally, the third linear variable would
substantially complicate computation and sampling of the residual.  This would require
implementation of an alternative sampling method.

The LO equations must now have a closure in time to be consistent with the HO equations. 
Previous work has attempted to simply subtract the continuous HO solution
from a BE discretziation of the discretized time-derivatives to add an artificial
term, with the addition of extra terms from hydrodynamics~\cite{holo_rh}.  This has the added benefit that the LO solver exclusively deals in
time-averaged unknowns.  We will alternatively use a parameteric closure in the time
variable similar to the spatial closures discussed in the previous section.  All
consistency terms will exist as before, but now as time-averaged quantities.  All that
remains is to eliminate the end of time step unknowns in terms of time-averaged
unknowns and previous time step solution, as estimated by the previous HO solution.  We
will investigate different parameteric forms of the closure for robustness.  The closure
produces LO equations that have the same numerical difficulty to solve as the fully-discrete LO equations, but
have the potential to preserve the accuracy of the MC integration in time.   Once the time-averaged unknowns have been calculated,
the time closures can be used to advance to the end of time step values for the next
time step.
%\subsection{Alternative time discretization}
%\label{sec:time}
%
%A final area of research to be investigated is in an alternative time discretization
%of the system.  This is the least developed portion of the research.  The goal of
%an alternative time discretization is to produce a more accurate solution in
%optically thin regions where particles transport a long distance.  This is an area
%where the MC treatment of the time-variable by IMC can produce greater accuracy,
%whereas a BE discretization will result in artificially fast propagation of energy.
%This may be important in applications such as stellar atmosphere calculations.  
%
%First, the ECMC
%solution is considered. The emission source
%is still treated as fully implicit, however the intensity is allowed to be continuous
%in time.  In inverting the $\B L$ operator, particles are born with some specific
%time, and their time is tracked until they reach the end of the time step.  Tallies are adjusted
%to account for the averaging over the time step.
%in terms of MC, letting . As with the spatial and angular variables, we assume a trial
%space for the time variable as well.  Two different options are considered.
%The first, is a doubly-discontinous treatement, where teh outflow in time is solved
%for with the MC.  Fig.~\ref{fig:dd_time} demonstrates the basic principle of this
%closure in time.  The second options is a linear discontinuous trial space in time.
%The first approach requires typical census type tallies~\cite{fig}, which adds
%significant complexity in terms of tallies.  The second will
%require the first moment in time to be tallied.  
%Again, ECMC will be computing the projection of the exact solution onto the trial
%space, so this will have the accuracy of IMC for the radiation.    Neither of these cases require the
%addition of Adds one extra set of unknowns to be stored. 
%
%Next, consistency with the LO system is
%necessary.  The goal is to not add additional equations to solve for the
%time-dependent unknowns, but to   Previous work has attempted to simply subtract the continuous HO solution
%from a BE discretziation of the discretized time-derivatives to add an artificial
%term, with the addition of extra terms from hydrodynamics~\cite{holo_rh}.  This has the added benefit that the LO solver exclusively deals in
%time-averaged unknowns.  As an alternative, we will estimate a parameter based on the
%ratio of the outflow in time to the average over the time step. One possible closure
%is
%\begin{equation}
%    I^{n+1} = 2\gamma_t \overline{I} - I^{n}
%\end{equation}
%where $\gamma_t$ is the closure factor and $\overline{I}$ is the time-averaged
%intensity.  A spatial and angle discretized version of the above equation can be used to
%eliminate the extra unknowns from the LO system.  The value of $\gamma_t$ is
%estimated by solving the above equation based on the latest HO solution for all
%parameters.  The stability and accuracy of these methods will need
%to be investigated. As a first step, a truly BE time discretization in the HO can be
%used in conjunction with the LO solver.

%
%  There
%is some difficulty in making the LO time-averaged quantities match with the HO
%solution, which may require some approximation.  However, this is not that different
%than the discrete case because the representation of $\tilde I^{n}$ is lagged an
%iteration anyways. 
%It is noted that the SIMC method explored 
%some similar alternate time discretizations for the temperature in the residual
%treatment, but second-order accurate time treatment.  However, the second-order approximation
